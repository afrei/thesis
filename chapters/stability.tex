In this section, our goal is to define stability (and \(\omega\)-stability in particular) and show that theories which aren't \(\omega\)-stable cannot be uncountably categorical for any uncountable cardinality. 

In order to define stability we need to recall a few definitions and facts about types. 
Let \(\mcM\) be an \(\mcL\)-structure and \(A \subset M\). 
It will often be natural to want to add constant symbols for every element of \(A\) to our language. 
We will call the resulting language \(\mcL_A\). \(\mcM\) has a natural definition as an \(\mcL_A\)-structure. 
In general, we call the set of \(\mcL_A\)-sentences true in \(\mcM\) ``\(\Th_A(\mcM)\)'' and, in the special case where \(A = M\) we call it \(\eldiag(\mcM)\) or the ``elementary diagram of \(\mcM\)''. 

\subsection{Defining Types}

\textbf{Definition.} Let \(A, \mcM, \mcL\) as above. 
An \(n\)-type, \(p\), is a set of \(\mcL_A\)-formulae in \(n\) free variables (\(v_1, \ldots, v_n\)) for which \(p \cup \Th_A(\mcM)\) is satisfiable.
We say that \(p\) is complete if, for every \(\mcL_A\)-formula \(\phi(v_1, \ldots, v_n)\), either \(p \ni \phi\) or \(p \ni \neg \phi\). 
We call the set of all complete \(n\) types (containing \(\mcL_A\) formulae satisfiable with \(\Th_A(\mcM)\)) ``\(S_n^\mcM(A)\)''. 

Let's consider a few simple examples of types when  \(\mcM = (\N,s)\), the natural numbers with successor and \(A = \{0\}\) (we've added a constant symbol for 0). 
One example or an incomplete type is when we take \(p = \{v_1 \neq 0\}\). 
As \(\N \models \phi(2)\) for all \(\phi \in p\), \(p \cup \Th_\emptyset(\N)\) must be satisfiable, and \(p\) is a type. 
Still, neither \(v_1 = s(s(s(0)))\) or its negation is in \(p\) so it is not complete. 

There is one clear source of complete types. 
Every element of \(\N\) has an associated complete type. 
Let \(p(v) = \{\phi(v) \mid \N \models \phi(1)\}\). 
As \(\N\) is a model realizing \(p \cup \Th_\emptyset(\N)\) when \(v_1\) is interpreted as 1, this set of formulae is a type. 
As every formula in one free variable is either true or false for 1, this type is complete. 
We say \(p\) is realized by 1. 
More generally, we say \(\bar{x} \in M^n\) realizes an \(n\)-type \(p\) if \(\mcM \models \phi(\bar{x})\) for all \(\phi \in p\).  

This, though, is not the only source of types. 
If we take \(p = \{v_1 \neq 0, v_1 \neq s^1(0), v_1 \neq s^2(0), \ldots\}\) we get a set of \(\mcL_{\{0\}}\)-formulae in one free variable. 
We note that \(p \cup \Th_{\{0\}}(\N)\) is satisfiable (as it is finitely satisfiable). 
We can extend \{p\} to a complete type, but that type cannot be realized by any natural number, since an element realizing \(p\) is greater than every natural number by construction. 
It isn't always the case that a type is realized. 

\subsection{Defining Stability}

\textbf{Definition.} Let \(T\) be a complete theory in a countable language and \(\kappa\) be an infinite cardinal. 
\(T\) is \(\kappa\)-stable if \(|S_n^\mcM(A)| = \kappa\) whenever \(\mcM \models T\) and \(|A| = \kappa\). 
Note: \(\aleph_0\)-stable theories are called \(\omega\)-stable for historical reasons. 

First, we will show that the theory of dense linear orders without endpoints is not \(\omega\)-stable. 
\(\mcM = (\R, <)\) is a model of this theory and \(A = \Q\) is a subset of size \(\aleph_0\). 
Still, every real number realizes a different type in \(S_1^\R(\Q)\) (as every two distinct real numbers have a rational number between them). 
So there must be at least as many types as there are real numbers; there are uncountably many types.  

Recall that we mentioned in the proof outline of Morley's Categoricity theorem that all uncountably categorical theories are \(\omega\)-stable. 
We mentioned the theory of infinite sets and the theory of (\(\Z, s\)) as uncountably categorical theories. 
Let's observe that both of these theories are \(\omega\)-stable. 

\textbf{Theorem.} The theory of infinite sets is \omst. 

First, consider the complete 1-types, \(p(v_1)\). \(p(v_1)\) must assert either that \(v_1 = a\) for exactly \(a \in A\) or that \(v_1 \neq a\) for all \(a \in A\). 
Moreover, as the theory of infinite sets admits \qe all formulae are equivalent to boolean combinations of such formulae. 
We have now identified all 1-types (there is one for each element in \(A\) and one more for when \(v_1\) isn't in \(A\)). 
This can be easily extended to \(n\)-types for \(n > 1\)

\textbf{Theorem.} The theory of \((\Z, s)\) is \omst. 

We will again consider 1-types and leave the general argument to the reader. 
We can assume, WLOG, that no two elements of \(A\) are in the same \(\Z\)-chain. 
Our 1-type either asserts that \(v_1\) is in the same \(\Z\)-chain as a single element of \(A\) and determines its position relative to that element, or asserts that \(v_1\) isn't in the same \(\Z\) chain as any element of \(A\). 
Every formula is equivalent to a \qf one, so as these are all of the possible complete atomic types, they are also all of the possible types.    

\subsection{A Theorem about \(\omega\)-stability}

We now aim to show that all theories, \(T\), which aren't \(\omega\)-stable cannot be uncountably categorical.
It suffices to show that for all uncountable \(\kappa\) we can present two non-isomorphic models of \(T\) of cardinality \(\kappa\). 
We will do so by finding one model realizing only countably many types, and one realizing uncountably many. 
Clearly, two such models will not be isomorphic, so \(T\) will not be categorical. 

\subsubsection{Realizing Uncountably Many Types Over a Countable Set}
As \(T\) isn't \omst, for some countable \(A\) we have that \(|S_n^\mcM(A)|\) is uncountable. 
We can realize all of these in some elementary extension of \(\mcM\) which can be taken to be of cardinality exactly \(\kappa\) for any desired uncountable \(\kappa\). 
Say \(\{p_i\}\) is a set of uncountably many but at most \(\kappa\) many types indexed by some set \(I\). 
Let \(c_i\) similar index that same number of constants in a new language \(\mcL^* = \mcL_\mcM \cup \{c_i\}\). 
Let \(\Gamma = T \cup \eldiag(\mcM) \cup \{p_i(c_i)\}\) (we assert each type \(p_i\) is realized by \(c_i\)). 
We claim that every finite subset of \(\Gamma\) is satisfiable (indeed, every finite subset is true in \(\mcM\)).
By compactness, we have a model of \(\Gamma\) of cardinality \(|\mcL^*| \leq \kappa\) and by the upward L\"owenheim-Skolem theorem, a model of cardinality \(\kappa\).
\(\Gamma\) asserts that uncountably many types are realized over \(A\), as desired. 

\subsubsection{Indiscernables}
Now, we must show the existence of a model of size \(\kappa\) which realizes only countably many types. To do so, we will have to construct some machinery.
This will be our guiding intuititon: to ensure our model is of size \(\kappa\), we will start out with \(\kappa\) many virtually identical (indiscernable) elements. 
To ensure our model is indeed a model of \(T\), we will close it under existentials by adding the right elements. 
We will be able to show that these ``right elements'' do not change the cardinality of the model and do not realize any types not found in out countable model.
Only countably many types can be realized in a countable model, so our constructed model will realize countably many types over any countable set, as desired. 

\textbf{Definition.} We say that a \(\{x_i \mid i \in I\}\) where \(I\) is an ordered set is a sequence of \textit{order indiscernibles (of order type \(I\))} if, given two increasing sequences of length \(n\) from \(I\), 
\(i_1 < i_2 < \ldots < i_n\) and \(j_1 < j_2 < \ldots < j_n\), we will have \(\mcM \models \phi(i_1, \ldots, i_n) \iff \mcM \models \phi(j_1, \ldots, j_n)\). 
Intuitively, these elements cannot be distinguised based on first order properties, modulo their relative positions.  
Recall that the theory of dense linear orders without endpoints admits \qe and it's easy to see that any increasing sequence from such an order is a sequence of indiscernibles. 

% Perhaps add something about how < need not be present in the language?

Our first theorem about order indiscernibles is that they always exist in a theory with infinite models.
Moreover, a set of order-indiscernibles exists of every order type.

We claim that any infinite model \(\mcM\) of a theory \(T\), for any finite set \(\Phi\) of formulae in \(n\) free-variables, we can present an \(X \subset M\) such that no two increasing senquences of length \(n\) taken from \(X\) can be distinguiched by a formula from \(\Phi\). 
(The actual order of \(M\) will not matter, we fix an arbitrary one).
We note that by compactness, this will suffice to show the existence of order indiscernibles of any order type. 
Also, observe that this is a straightforward consequence of Ramsey's theorem (the infinite version).
Ramsey's theorem states that if \(S\) is an infinite set whose subsets of size \(n\) are colored with \(c\) different colors, there exists a \(T \subset S\) for which every subset of size \(n\) of \(T\) has the same color. 
In our case, subsets of \(\Phi\) color subsets of size \(n\) taken from \(M\) by seeing which formulae of \(\Phi\) are true of the (ordered) elements of the subset.

\subsubsection{A Proof of Ramsey's Theorem}
%We will therefore prove Ramsey's theorem. 
Assume without loss of generality that \(S = \N\).
When \(n=1\), Ramsey's Theorem states we cannot partition \(\N\) into finitely many finite sets, which is clear.
Now, we will induct on \(n\). 
We will constuct a \(T \subset \N\) with the following property: the color of every subset of size \(n\) of \(T\) is determined by its least element. 
Once we have that set \(T\), we can color the elements \(t\) of \(T\) by the color of all \(n\)-element sets with \(t\) as their element. 
By the pidgeonhole principle, there will be one color which appears infinitely often. 
If we take \(T' \subset T\) consisting of those elements, every \(n\) element subset of \(T'\) will have a least element and by construction, that subset will have the color \(c\) we chose. 
\(T'\) will be the desired set. 

We now construct a \(T \subset \N\) such that the color of every subset of size \(n\) of \(T\) is determined by its least element. 
We can color subsets, \(A\) of size \(n-1\) of \(\N \setminus \{0\}\) by the color assigned to \(A \cup \{0\}\). 
By the inductive hypothesis, there is \(T_0 \subset \N \setminus \{0\}\) such that every \(n\)-element subset of \(T_0 \cup \{0\}\) with 0 as its least element is the same color. 
We now take the least element \(t_0 \in T_0\) and similarly apply the inductive typothesis there to get an infinite \(T_1 \subset T_0\) such that every subset of \(T_1 \cup \{0, t_0\}\) with \(t_0\) as its least element has the same color. 
As \(T_1 \subset T_0\) the same is still true for subsets with \(0\) as their least element. 
Continuing inductively, we take \(\bigcap_{<\omega}T_i = \{0, t_0, t_1, \ldots\}\) which will have this property. 
By what we've said above, we've proven Ramsey's Theorem. 
This is also enough to complete our compactness argument showing that order-indiscernibles of every order type exist.  
It is natural to identify this set of order indiscernibles with \(I\), so that if \(\mcM \models \Gamma \cup T\), then \(I \subset M\).

\subsubsection{Realizing Only Countably Many Types Over Every Countable Set}
Now that we are familiar with order indiscernibles, we can use them to build \textit{Ehrenfeucht-Mostowski Models}, which are constructed as \textit{Skolem hulls} of order indiscernibles. 
We recall that an \(\mcL\)-theory is said to have \textit{built-in Skolem functions} when for all \(\mcL\)-formulae, \(\phi(v, \bar{w})\), there is a function symbol \(f_\phi\) for which 
\(T \models \A \bar{w} [(\E v \phi(v, \bar{w})) \to \phi(f_\phi(\bar{w}), \bar{w})]\), in other words, the Skolem functions provide whitnesses for all existentials. 
Given any \(\mcL\)-theory \(T\), we can extend \(\mcL\) to \(\mcL^*\) so that it includes Skolem-functions and we also extend \(T\) to \(T^*\) which asserts that \(\mcL^*\) has Skolem-functions. 
Moreover, there is a natural way to interpret any \(\mcM \models T\) as an \(\mcL^*\) structure modeling \(T^*\).
Given, then \(I \subset M\), we can look at the substructure of \(\mcM\) whose universe will be the smallest set containing \(I\) and closed under our Skolem functions. 
We will call this model ``\(\mathcal{H}(I)\)'' or the Skolem hull of \(I\).
Note that \(\mathcal{H}(I) \prec \mcM\).

We can now achieve our goal of constructing a model of any \(\omega\)-unstable theory which realizes only countably many types over any countable set. 
Let \(\mcM = \mathcal{H}((\kappa, <))\) be our model of cardinality \(\kappa\) and let \(A\) be an arbitrary countable subset of \(M\). 
As every element of \(M\) (and of \(A\)) must be a Skolem term (i.e. of the form \(f_\phi(\bar{x})\) for some vector of indiscernables), it suffices to look at the types over countable subsets of \(\kappa\).
Let \(X\) be this countable subset of \(\kappa\). 
It suffices to show that for each fixed \(n, \phi\), terms of the form \(f_\phi(y_1, \ldots, y_n)\) realize only countably many types. 
(We then have divided \(M\) into countably many sets each realizing at most countably many types).
We see that \(f_\phi(y_1, \ldots, y_n)\) and \(f_\phi(z_1, \ldots, z_n)\) realize the same type if, for each \(i \leq n\) there is no \(x \in X\) between \(y_i\) and \(z_i\) or that \(y_i\) and \(z_i\) determine the same cut in \(X\).  
This is because indiscernability tells us \(\mcM \models \psi(f_\phi(\bar{y}), \bar{x}) \iff \mcM \models \psi(f_\phi(\bar{z}, \bar{x})\) when the relative orders of the indiscernibles in \(\bar{y}\bar{x}\) and in \(\bar{z}\bar{x}\) are the same. 
It now suffices to note that, as \(\kappa\) is well ordered, we get only countably many cuts of \(X\). 
This deals with the 1-types. Dealing with the \(n\)-types only requires dealing with \(n\)-tuples of formulae instead of the single formula \(\phi\) we adressed here. 

We have succeded in showing that there is a model of any \(\omega\)-unstable theory realizing only countably many types over any countable set.
This model cannot be isomorphic to one realizing uncountably many types over a countable set, which we showed must exist as well. 
Therefore, any \(\omega\)-unstable theory cannot be categorical.
