Morley's Categoricity Theorem has been called the ``beginning of modern model theory''.
To understand its statement requires only familiarity with some basic concepts from mathematical logic, specifically theories, models, cardinality and categoricity. 
A student familiar with just these notions, and not much else, can see the importance of this result.
These notions are often introduced in a first course in mathematical logic.
Such a course might use Enderton's book. %cite

This paper is aimed specifically at such a student, who is familiar with basic logic including models, compactness and the L\"owenheim-Skolem theorem.
Unlike popular introductory model theory texts, such as Marker's, %cite
this thesis keeps focused on Morley's Categoricity Theorem.
Unlike Marker's text, this thesis will introduce notions and immediately point out their relation to Morley's Categoricity. 
In the process, the student will learn about many important model theory concepts, such as stability, types and elementary extensions. 

As this thesis isn't a general introduction to model theory, it doesn't adress many important model theory concepts in detail, such as quantifier elimination.
The interested reader is directed to Marker's text for more information. 
With this thesis as background, the reader will find that they have motivation for model theory concepts they may later encounter in Marker.

We begin by stating Morley's Categoricity Theorem and recalling some basic definitions, then we immediately give examples which will set the tone for the rest of the text. 
When notions are introduced, seeing how they apply to these theories will be useful to the reader. 

The rest of the thesis focuses on the only two obstacles to categoricity, \(\omega\)-instability and Vaughtian pairs. 
Showing that both of these prevent categoricity is the main goal of sections ``\(\omega\)-stability'' and ``Vaughtian pairs''. 
In these sections we also introduce theorems which will be useful in the final proof. 

The next section is the one entitled ``Proof of Morley's Categoricity Theorem'' which will assume our aforementioned obstacles to categoricity are not present and prove categoricity for all such theories. 
Specifically, we show that there is a good notion of ``dimension'' for models, which determines a model up to isomorphism. 
All uncountable models of the same cardinality will have the same dimension, proving categoricity. 

To end the thesis, we adress a few harder questions which aim to extend this result in natural ways. 
To adress many of these extensions would require much more machinery, so we present some of the main ideas and state results without full proof. 