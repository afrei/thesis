\subsection{Defining Categoricity}

First, I recall the defition of categoricity in some cardinal \(\lambda\):

\begin{definition}\label{def_categoricity}
A theory \(T\) is \(\lambda\)-categorical if \(\mcM \isom \mcN\) whenever \(\mcM, \mcN \models T\) and \(|M| = |N| = \lambda\).
\end{definition}

\begin{example}\label{example_categoricity_sets}
Consider the theory of infinite sets in the pure language of equality. 
This theory is \(\lambda\)-categorical for all \(\lambda\).
Consider two models \(\mcM, \mcN\) such that \(|M|=|N|=\lambda\). 
By hypothesis we have a bijection \(f: M \to N\). \(f\) is an isomorphism of sets, so \(M \isom N\).
\end{example}

Recall that a theory, \(T\), is complete if for all sentences \(\phi\) we have either \(T \models \phi\) or \(T \models \neg \phi\). 
Note that this theory must also be complete. 
More generally, Vaught's test tells us that theories without finite models that are categorical in some infinite cardinal \(\kappa\) are complete. 
(If such a theory were not complete, it would have models of cardinality of \(\lambda\) disagreeing on some \(\phi\) which couldn't then be isomorphic)

\begin{example}\label{example_categoricity_equiv}
Consider the theory of an equivalence relation with exactly two infinite classes. 
This theory is ``\(\aleph_0\)-categorical''.
We will often call an \(\aleph_0\)-categorical theory ``countably categorical''.
These terms are synonomyous.  
Consider two countable models \(\mcM, \mcN\). 
There must be countably many members of each equivalence class in any countable model. 
We can present a bijection \(f_1\) between members of the first equivalence class in  \(\mcM\) and those in \(\mcN\) and a similar bijection \(f_2\) for the second equivalence class. 
We see that \(f = f_1 \cup f_2\) is a bijection and actually an isomorphism. 

On the other hand, if \(\lambda > \aleph_0\) is an infinite cardinal, then this theory isn't \(\lambda\)-categorical. 
It suffices to note that we can have the two non-isomorphic cases: where the first equivalence class has cardinality \(\lambda\) and the second is countable 
and where both classes are of cardinality \(\lambda\). Vaught's theorem tells us that even though this theory isn't \(\aleph_1\) categorical, it is still complete. 
\end{example}

\begin{example}\label{example_categoricity_integers}
Consider the theory of (\(\Z, s\)), the integers with successor. 
Included in this theory is the fact that every element has a unique successor and predecessor and that there are no cycles (i.e. no number is its own successor, or the sucessor of its successor, etc.). 
Using this fact, we observe that each model of this theory must consist of some number of \(\Z\)-chains. 
For uncountable cardinals \(\kappa\), we know that, up to isomorphism, the only model of cardinality \(\kappa\) consists of \(\kappa\) many \(\Z\)-chains. 
So this theory is categorical for all uncountable cardinals and must be complete. 
That being said, there are countably many non-isomorphic countable models, which are the familiar \(\Z\), \(n\) copies of \(\Z\) for any natural \(n\) or countably many copies of \(\Z\). 
This theory is not countably categorical, but it is \(\kappa\) categorical for all uncountable \(\kappa\). 
\end{example}

\begin{example}\label{example_categoricity_sequence}
We consider the language \(\mcL = \{<, c_1, c_2, c_3, \ldots\}\) which has countably many constant symbols indexed by the natural numbers. 
In our theory \(<\) is a dense linear order and \(c_i\) is a strictly increasing sequence. 
If we look at any finite sublanguage, we can see that this theory is complete (by constructing an explicit isomorphism). % put in more details or reference?
This theory is never categorical as we can present models in which \(c_i\) is unbounded, has a least upper bound or is bounded without a least upper bound. 
\end{example}

Morley's categoricity theorem will state that these are the only four cases. 
That theorem will be the main focus of this thesis. 

\subsection{Stating Morley's Categoricity Theorem}

\begin{theorem}\label{theorem_morleys_categoricity}
If a first order theory in a countable language is categorical in some uncountable cardinal, it is categorical in all uncountable cardinals. 
\end{theorem}

I present a very general sketch of the proof of this theorem which I will flesh out throughout the rest of this work. 

We will define two properties of theories: being \(\omega\)-stable and having Vaughtian pairs. 
It turns out that lacking \(\omega\)-stability or having Vaughtian pairs will prevent uncountable categoricity, so all uncountably categorical theories lack Vaughtian pairs and are \(\omega\)-stable. 

We then observe that these two conditions, lacking Vaughtian pairs and being \(\omega\)-stable, are sufficient to show categoricity in all uncountable cardinals. 
Hence, the categoricity of a theory in some uncountable cardinal must imply \(\omega\)-stability and the absence of Vaughtian pairs, which, in turn, imply categoricity in all other uncountable cardinals.

% Once I complete this proof, I will elaborate more on issues of stability in general, morley rank and forking.  
