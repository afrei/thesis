Morley's Categoricity Theorem has been called the ``beginning of modern model theory'' \cite{mar}. 
To understand its statement requires only familiarity with some basic concepts from mathematical logic, specifically theories, models, cardinality and categoricity. 
All of these notions will likely be covered in a first course in logic.
They are, for example, covered in Enderton's ``A Mathematical Introduction to Logic'' \cite{end}.
A student familiar with just these notions, and not much else, can see the importance of this result.

This paper is aimed specifically at such a student -- one who is familiar with basic logic including models, compactness and the L\"owenheim-Skolem theorem.
Unlike popular introductory model theory texts, such as Marker's \cite{mar},
this thesis is solely focused on Morley's Categoricity Theorem.
Unlike Marker's text, this thesis will introduce notions and immediately point out their use in the proof of Morley's Categoricity Theorem. 
In the process, the student will learn about many important model theory concepts, such as stability, types and elementary extensions. 

As this thesis isn't a general introduction to model theory, it doesn't address many important model theory concepts, such as quantifier elimination, in detail.
The interested reader is directed to Marker's text for more information. 
With this thesis as background, we hope the reader will be motivated to further explore model theory.

We begin by stating Morley's Categoricity Theorem and recalling some basic definitions.
We immediately give examples which will set the tone for the rest of the text. 
When notions are introduced, seeing how they apply to these theories will be useful to the reader. 

The rest of the thesis focuses on the only two obstacles to categoricity, \(\omega\)-instability and Vaughtian pairs. 
Showing that both of these prevent categoricity is the main goal of the sections ``\(\omega\)-stability'' and ``Vaughtian pairs''. 
In these sections we also introduce theorems which will be essential in the following section.

The next section is entitled ``Proof of Morley's Categoricity Theorem'' which will assume our aforementioned obstacles to categoricity are not present and prove categoricity for all such theories. 
Specifically, we show that there is a good notion of ``dimension'' for models, which determines a model up to isomorphism. 
This will be sufficient to prove the main theorem.

To end the thesis, we address a few harder questions which aim to extend this result in natural ways. 
A full resolution of many of the issues in this section would require more definitions than can be adequately covered in this thesis. 
We will limit ourselves to asking these questions and stating their answers without proof. 
Additionally, we will point interested readers to references where these questions are given a fair treatment. 
