\subsection{Defining Categoricity}

We begin by recalling the defition of categoricity in a cardinal \(\lambda\):

\begin{definition}\label{def_categoricity}
A theory \(T\) is \(\lambda\)-categorical if \(\mcM \isom \mcN\) whenever \(\mcM, \mcN \models T\) and \(|M| = |N| = \lambda\).
Alternatively, \(T\) is \(\lambda\)-categorical if it has a single model of cardinality \(\lambda\) up to isomorphism. 
\end{definition}

\begin{example}\label{example_categoricity_sets}
Consider the theory of infinite sets in the pure language of equality. 
This theory is \(\lambda\)-categorical for all \(\lambda\).
Consider two models \(\mcM, \mcN\) such that \(|M|=|N|=\lambda\). 
By hypothesis we have a bijection \(f: M \to N\). 
\(f\) is an isomorphism of sets, so \(M \isom N\).
\end{example}

Recall that a theory, \(T\), is complete if for all sentences \(\phi\) we have either \(T \models \phi\) or \(T \models \neg \phi\). 
Note that this theory must also be complete. 
More generally, Vaught's test tells us that theories without finite models that are categorical in some infinite cardinal \(\kappa\) are complete. 
(If such a theory were not complete, it would have models of cardinality of \(\lambda\) disagreeing on some \(\phi\) which couldn't then be isomorphic.)

\begin{example}\label{example_categoricity_equiv}
Consider the theory of an equivalence relation with exactly two infinite classes. 
This theory is ``\(\aleph_0\)-categorical''.
We will often call an \(\aleph_0\)-categorical theory ``countably categorical''.
(These terms are synonomyous.)  
Consider two countable models \(\mcM, \mcN\) of this theory. 
There must be countably many members of each equivalence class in any countable model. 
We can present a bijection \(f_1\) between members of the first equivalence class in  \(\mcM\) and those in \(\mcN\) and a similar bijection \(f_2\) for the second equivalence class. 
We see that \(f = f_1 \cup f_2\) is a bijection and actually an isomorphism. 

On the other hand, if \(\lambda > \aleph_0\) is an infinite cardinal, then this theory isn't \(\lambda\)-categorical. 
It suffices to note that we can have the two non-isomorphic cases: where the first equivalence class has cardinality \(\lambda\) and the second is countable and where both classes are of cardinality \(\lambda\). 
Vaught's theorem tells us that even though this theory isn't \(\aleph_1\) categorical, it is still complete. 
\end{example}

\begin{example}\label{example_categoricity_integers}
Consider the theory of (\(\Z, s\)), the integers with successor. 
Included in this theory is the fact that every element has a unique successor and predecessor and that there are no cycles (i.e. no number is its own successor, or the sucessor of its successor, etc.). 
Using this fact, we observe that each model of this theory must consist of some number of \(\Z\)-chains (i.e. submodels isomorphic to \(\Z\) itself). 
For uncountable cardinals \(\kappa\), we know that, up to isomorphism, the only model of cardinality \(\kappa\) consists of \(\kappa\) many \(\Z\)-chains. 
Making this theory categorical for all uncountable cardinals and complete. 
That being said, there are countably many non-isomorphic countable models, which are the familiar \(\Z\), \(n\) copies of \(\Z\) for any natural \(n\) or countably many copies of \(\Z\). 
This theory is not countably categorical, but it is \(\kappa\) categorical for all uncountable \(\kappa\). 
\end{example}

\begin{example}\label{example_categoricity_sequence}
We consider the language \(\mcL = \{<, c_1, c_2, c_3, \ldots\}\) which has countably many constant symbols indexed by the natural numbers. 
In our theory, \(T\), \(<\) is a dense linear order (this is a first order property) and \(c_i\) is a strictly increasing sequence. 

We show that \(T\) is complete.
Consider a sentence \(\phi\), we will show that either \(T \models \phi\) or \(T \models \neg \phi\). 
\(\phi\) can only include finitely many of the constant symbols, so assume without loss of generality that \(\phi\) is an \(\mcL_n\) formula where \(\mcL_n = \{<, c_1, \ldots, c_n\}\).
It would suffice to show that \(T_0 = \{\phi \in T \mid \phi \text{ is an } \mcL_n \text{ formula}\}\) is countably categorical and therefore complete. 
If \(\mcN, \mcM\) are countable models of this theory, we can build an isomorphism between them with the ``back and forth'' method. 
We define an isomorphism as a union of partial functions (``partial isomorphisms'') respecting the order. 
Let \(f_0\) be the function mapping \(c_i\) in \(\mcM\) to \(c_i\) in \(\mcN\) and undefined everywhere else. 
We can enumerate the elements of \(\mcM, \mcN\) by the natural numbers making \(M = \{m_1, \ldots\}, N = \{n_1, \ldots\}\). 
We will define \(f_i\) inductively for \(i > 0\). 
If \(i\) is even, we extend \(f_{i-1}\) to include \(m_{i/2}\) in its domain if \(m_{i/2} \notin \text{dom}(f_{i-1})\) and otherwise, let \(f_i = f_{i-1}\).
Note that as \(\text{dom}(f)\) is finite and doesn't include \(m_{i/2}\) there is a greatest element in \(\text{dom}(f)\) less than \(m_{i/2}\) and a least element greater than it, call them \(p\) and \(q\) respectively.
We can map \(m_{i/2}\) to an arbitrary element in \(N\) which is greater than \(f(p)\) but less than \(f(q)\). 
For odd \(i\) we similarly extend the range of \(f_{i-1}\). 
Letting \(f = \bigcup_{i \in \N}f(i)\) we will find that we have an isomorphism. 
Using the fact that dense linear orders have quantifier elimination \cite{mar} % specific in reference
makes verifying this fact rather straightforward, and the details are left to the reader.
 
This theory is never categorical.
Consider two models of this theory, both resemble \(\Q\) in their underlying set and order.
In \(\mcM\) we let \(c_i = i\), making our increasing sequence the natural numbers. 
In a second model, \(\mcN\), we let \(c_i = 1 - \frac{1}{i}\).
It suffices to note that any isomorphism \(f: \mcN \to \mcM\) must send \(1\) to some element of \(\Q\).
Inevitably \(f(1)\) will be less than \(n\) for some natural number \(n\) and therefore, \(\mcN \models f(1) < c_i\) for some \(i\).
As \(\mcM \models c_i < 1\) for all \(i\), \(f\) cannot be an isomorphism. 
We've shown two countable models of \(T\) which aren't isomorphic, showing that \(T\) isn't countably categorical. 
Taking these two models, and adding \(\kappa\) many elements greater than every element of \(\Q\) will give us two non-isomorphic models of cardinality \(\kappa\) by a similar argument. 

\(T\) is complete but never categorical. 
\end{example}

Morley's categoricity theorem will state that these are the only four cases.
That is, a theory \(T\) is never categorical, always categorical, \(\kappa\) categorical for all \(\kappa > \aleph_0\) or countably categorical but categorical in no uncountable cardinal.  
That theorem will be the main focus of this thesis. 

\subsection{Stating Morley's Categoricity Theorem}

\begin{theorem}\label{theorem_morleys_categoricity}
If a first order theory in a countable language is categorical in some uncountable cardinal, it is categorical in all uncountable cardinals. 
\end{theorem}

I present a very general sketch of the proof of this theorem which I will flesh out throughout the rest of this work. 

We will define two properties of theories: being \(\omega\)-stable and having Vaughtian pairs. 
It turns out that lacking \(\omega\)-stability or having Vaughtian pairs will prevent uncountable categoricity, so all uncountably categorical theories lack Vaughtian pairs and are \(\omega\)-stable. 

We then observe that these two conditions, lacking Vaughtian pairs and being \(\omega\)-stable, are sufficient to show categoricity in all uncountable cardinals. 
Hence, the categoricity of a theory in some uncountable cardinal must imply \(\omega\)-stability and the absence of Vaughtian pairs, which, in turn, imply categoricity in all other uncountable cardinals.

Once I complete this proof, I will seek to answer some related questions about how many non-isomorphic models there can be of a theory. 

%I will elaborate more on issues of stability in general, morley rank and forking.  