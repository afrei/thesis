\section{Definition}
The (only) obstacle to uncountable categoricity aside from \(\omega\)-instability is the existence of Vaughtian pairs. %Introduce V-pairs first
We will be able to use a Vaughtian pair of models of a theory to construct a model of size \(\kappa\) with a countable definable subset.  
Such models prevent uncountable categoricity because given any model of cardinality \(\kappa\) we can construct an elementarily equivalent model in which all infinite definable sets are of cardinality \(\kappa\).
We simply add \(\kappa\) many vectors of constant symbols for each formula \(\phi\) defining an infinite set, assert that none of the vectors of constant symbols are equal. This is clearly finitely satisfiable and its model has the desired property.
First, let's define what it means for two models to form a Vaughtian pair. % Possible Addition: Definition of kappa lambda models (definition in previous commits)

\begin{definition}\label{definition_vaughtian_pairs}
\((\mcN, \mcM)\) form a Vaughtian pair if \(\mcM \prec \mcN\) and there is an \(\mcL_\mcM\)-formula \(\phi(\bar{v})\) (in some number of free variables) for which
\(\{\bar{m} \in N^k \mid \mcN \models \phi(\bar{n})\}\) is an infinite subset of \(M^k\). %7-1 m? I don't get it (I don't get what's not to get?)
In other words, a formula with parameters in \(M\) defines an infinite subset of \(N^k\) containing no elements of \(N^k \setminus M^k\).
\end{definition}

Even though we write \(\mcM \prec \mcN\), we write the Vaughtian pair in the other order, that is as \((\mcN, \mcM)\), because we can think of this pair of models as a single model (\(\mcN\)) with a distinguished subset corresponding to \(\mcM\).
This will often be useful in proving theorems and lemmas about Vaughtian pairs, so we will flesh out the construction here. 
If \(\mcM, \mcN\) are \(\mcL\)-structures, let  \(\mcL^* = \mcL \cup \{U\}\). 
We look at (\(\mcN, \mcM\)) as an \(\mcL^*\)-structure which shares its underlying set and interpretations of \(\mcL\) with \(\mcN\) but where \(U\) is the subset \(M\). 
This predicate allows us to state that a formula \(\phi(\bar{v})\) is true in \(\mcM\) with slight modification to \(\phi\).
Specifically, we will have \((\mcN, \mcM) \models \phi_U(\bar{v}) \iff \mcM \models \phi(\bar{v})\).
For \(\phi(\bar{v})\) \qf we let \(\phi_U(\bar{v}) = \bigwedge U(v_i) \land \phi(\bar{v})\) and for \(\phi(\bar{v}) = \exists v \psi, \phi_U(\bar{v}) = \E v (U(v) \land \psi_U)\). 
In the other cases, do the natural thing. %Say this better
% We might want to show how we can also state that this \mcL^* structure is a Vaughtian pair as a first order thing (mostly). 

\begin{theorem}\label{thm_countable_vaughtian_pairs}
Assuming a theory \(T\) has a Vaughtian pair \(\mcN, \mcM\), it has a Vaughtian pair \((\mcN_0, \mcM_0)\) of countable models. 
\end{theorem}

\begin{proof}\label{proof_countable_vaughtian_pairs}
In order to show the existence of a countable model, we will want to use the L\"owenheim-Skolem theorm.
To do so, we will have to view the pair of models as a single model. 

Our goal is to demonstrate an \(\mcL^*\)-theory in a countable language asserting that \((\mcN, \mcM)\) is Vaughtian pair.
By the L\"owenheim-Skolem theorem, we will get out countable pair of models.
This comes in two parts: asserting \(\mcM \prec \mcN\) and asserting that we can isolate an infinite subset of \(M\). 
We assert the former with the formulae \((\bigwedge U(v_i) \land \psi(\bar{v})) \to \psi_U(\bar{v})\). 
Let \(\phi(\bar{v})\) be the \(\mcL_\mcM\)-formula which isolates an infinite subset of \(M^k\). 
We can assert that \(\phi\) has infinitely many realizations by \(\E \bar{v}_1, \E \bar{v}_2, \ldots \E \bar{v}_n \bigwedge \bar{v}_i \neq \bar{v}_j \land \bigwedge \phi(\bar{v}_i)\). %This argument got out of hand
We can assert that \(\phi\) only holds for vectors from \(M^k\) by \(\A \bar{v} (\phi(\bar{v}) \to U(\bar{v}))\).
Finally, we assert that \(\mcN\) is a proper extension with \(\E x \neg U(x)\). 
By construction, the countable model of this theory which is satisfied by \((\mcN, \mcM)\) will be our countable model.
\end{proof}

For the next few claims we will need to use a property called homogeneity which will allow us to show that certain countable models are isomorphic.

\begin{definition}\label{definition_homogeneity}
We call a countable model \(\mcM\) is homogeneous if all finite \(A \subset M\) and functions \(f: A \to M\) for which \(\mcM \models \phi(\bar{v}) \iff \mcM \models \phi(f(\bar{v}))\) -- we call this property being ``partial elementary'' -- and every \(m \in M\), there is a \(f': A \cup \{m\} \to M\) extending \(f\) which is partial elementary. % Examples?
\end{definition}

The first interesting thing we will note about countable homogeneous models is that any partial elementary map \(f: A \to M\) (where \(A \subset M\) is finite) can be extended to an automorphism. 
To see this, note that partial elementary maps have partial elementary inverses. And \(f_0 = f\) %this bad
We will alternatively extend the domain and range of \(f_i\) by elements in \(M\). 
By construction, \(f^* = \bigcup_{<\omega}f_i\) has domain and range equal to all of \(M\). %something here
Additionally, it is easy to verify that \(f^*\) is a homomorphism. 

This is why homogeneity is a useful property. We can use it to build isomorphisms, which will be useful soon. 

\begin{lemma}\label{lemma_types_isomorphism}
The next thing we note about countable homogeneous models, \(\mcM\) and \(\mcN\), of the same complete theory, \(T\), is that if \(\mcM\) and \(\mcN\) realize the same types in \(S_n(T)\), they are isomorphic.
\end{lemma}

\begin{proof}\label{proof_types_isomorphism}
To show this, we will build a map in a way very similar to the way we built \(f^*\) above. 
Note that we need \(T\) to be complete in order for \(f_0=\emptyset\) to be partial elementary.
We will enumerate the elements in the universe of both models, and iteratively add elements to the domain and range of a partial elementary map. 
If we can do this, the union of this sequence of maps will be an isomorphism.
As the inverse of a partial elementary map is also partial elementary, it will suffice to show that we can extend a partial elementary map to include an arbitrary element in its range. 
Let \(dom(f) = \{m_1, \ldots, m_i\}, rng(f) = \{n_1, \ldots, n_i\}, f(m_j) = n_j\).
Now, choose \{m\} an arbitrary element of \(M\), our goal is to find \(g \supset f\) which is partial elementary and \(m \in dom(g)\).
Note that we can find a partial elementary \(h\) for which \(dom(f) \cup \{m\} \subset dom(h)\) by merely finding the realization of \(tp(\bar{m}, m)\) in \(\mcN\), which is guarunteed by \(\mcM, \mcN\) realizing the same types. 
Our one remaining obstacle is that \(h(m_i)\) may not be equal to \(n_i\). 
This can be rectified with homogeneity. We can easily see that the function which takes \(h(m_i) \mapsto n_i\) can be extended to an automorphism. 
By simply composing this automorphism with our \(h\) from above to form \(g\) we have found our desired function. 
\end{proof}

% Possible Clarification: Moving from types over models to types over theories (p.125)

\begin{theorem}\label{theorem_vaughtian_pairs_countable_isomorphic}
A Vaughtian pair of countable models, (\(\mcN_0, \mcM_0\)) can be extended to a Vaughtian pair of isomorphic countable models.
\end{theorem}

\begin{proof}
By above, it suffices to show a vaughtian pair \((\mcN, \mcM)\) can be extended to a vaughtian pair \((\mcN^*, \mcM^*)\) where \(\mcN^*, \mcM^*\) are homogeneous and realize the same types. 

We aim to construct a chain of Vaughtian pairs indexed by the natural numbers such that each pair is an elementary extension of the previous one and their union will be homogeneous and realize the same types. 
The zeroth pair is the countable pair (\(\mcN_0, \mcM_0\)) from above. 
For \(i > 0\) we construct (\(\mcN_i, \mcM_i\)) as follows (the details of constructions in each of these stages will follow):
\begin{enumerate}
\item  If \(i\) is a multiple of 3, then (\(\mcN_i, \mcM_i\)) is an elementary extension of (\(\mcN_{i-1}, \mcM_{i-1}\)) such that for every \(\bar{a} \in M\) every type in \(S_n(\bar{a})\) realized in \(\mcN_{i-1}\) is realized in \(\mcM_i\). %make sure vector notation is good. 
\item  If \(i\) is one more than a mutliple of 3, then (\(\mcN_i, \mcM_i\)) is an elementary extension of (\(\mcN_{i-1}, \mcM_{i-1}\)) such that if \(\bar{a}, \bar{b}\) realize the same type in \(\mcM_{i-1}\) and \(c \in \mcM_{i-1}\), then there is a \(d \in \mcM_i\) such that \(\bar{a}c\) and \(\bar{b}d\) realize the same type in \(\mcM_i\). 
\item If \(i\) is two more than a mutliple of 3, then (\(\mcN_i, \mcM_i\)) is an elementary extension of (\(\mcN_{i-1}, \mcM_{i-1}\)) such that if \(\bar{a}, \bar{b}\) realize the same type in \(\mcN_{i-1}\) and \(c \in \mcN_{i-1}\), then there is a \(d \in \mcN_i\) such that \(\bar{a}c\) and \(\bar{b}d\) realize the same type in \(\mcN_i\).
\end{enumerate}
If we let \((\mcN, \mcM) = \bigcup_{i \in \N}(\mcN_i, \mcM_i)\), we see that every type realized in \(\mcN\) is realized in some \(\mcN_i\) and therefore in \(\mcM_{i+3}\) (as we must have done the first stage of our construction once in the interim) and thus in \(\mcM\). 
Similarly, any type realized in \(\mcM\) must be realized in some \(\mcM_i\) and in \(\mcN_i\) as well (as \(\mcM \prec \mcN\)) and thus in \(\mcN\).
Thus, \(\mcM\) and  \(\mcN\) realize the same types.  
Given a partial elementary map \(f:\mcM \to \mcM\) with finite domain, we have \(\text{dom}(f) \in \mcM_i\) for some \(i\), and stage two of our construction guaruntees that it can be extended. 
Similarly for homogeneity of \(\mcN\) by the third stage of our construction. 

At this point it would suffice to show how to construct the elementary extensions mentioned in our three stages. 
\begin{enumerate}
\item As we can enumerate the countably many types over finite sets realized in any countable \(\mcN\) (there are countably many finite sequences from a countable set and countably many \(\bar{a}\)), it would suffice to show that if \(p(\bar{v})\) is an \(\bar{a}\)-type and \((\mcN, \mcM)\) is a Vaughtian pair with \(\mcN\) realizing \(p(\bar{v})\) that there is a \((\mcN, \mcM) \prec (\mcN', \mcM')\) with \(\mcM'\) realizing \(p(\bar{v})\). 
We can see that \(\Gamma = \eldiag((\mcN, \mcM)) \cup \{\phi_U(\bar{v}, \bar{a}) \mid \phi(\bar{v}, \bar{a}) \in  p(\bar{v})\}\) is satisfiable.  
Specifically, if \(\phi_1(\bar{v}, \bar{a}), \ldots, \phi_k(\bar{v}, \bar{a}) \in p\) are a finite subset of this type, we know that \(\mcN \models \bigwedge \E \bar{x} \phi_k(\bar{x},\bar{a})\) and therefore that \(\mcM \models \bigwedge \E \bar{x} (\phi_k)_U(\bar{x},\bar{a})\). 
This finite subset if realized in \((\mcN, \mcM)\) itself. 
Any model of \(\Gamma\) is an elementary extension of \((\mcN, \mcM)\) which we may call \((\mcN', \mcM')\).
As \((\mcN', \mcM') \models \Gamma\) we also have that \(p(\bar{v})\) is realized in \((\mcN', \mcM')\). 
\item The construction for this stage is very similar to the one covered just above for the first stage.
% This stage

\item
% This stage

% TODO 1: Proof of 4.3.38
\textcolor{red}{Insert Proof Here \ldots}

\end{proof}

% claim?
\begin{theorem}\label{theorem_aleph_one_vaightian_pairs}
Given a Vaughtian pair of countable isomorphic models, \((\mcN_1, \mcN_0) \models T\), we can construct a model \(\mcN^*\models T\) which is of cardinality \(\aleph_1\) and has a definable subset which is countable.  
\end{theorem}

\begin{proof}
% We have \(\mcN_0 \prec \mcN_1\) so \(\iota: \mcN_0 \to \mcN_1\) is the map including \(\mcN_0\) into \(\mcN_1\). 
% Additionally, we have an isomorphism \(f: \mcN_0 \to \mcN_1\). 

% TODO 2: Proof of 4.3.34
\textcolor{red}{Insert Proof Here \ldots}
\end{proof}

\begin{theorem}\label{theorem_uncountable_vaightian_pairs}
Given a Vaughtian pair \((\mcN, \mcM)\), \(\mcN, \mcM \models T\) where \(|N| = \aleph_1\), \(\mcM\) is countable and \(T\) is \(\omega\)-stable, for every uncountable cardinal \(\kappa\), there is a model of \(T\) of cardinality \(\kappa\) with a countable definable subset.  
\end{theorem}

\begin{proof}
% TODO 2b: Proof of 4.3.41 
Let \(\phi\) be the formula defining a countable subset. 
It suffices to construct an elementary chain of models indexed by ordinals less than \(\kappa\) such that \(\mcM_\alpha \prec \mcM_{\alpha+1}\) and \(\mcM_\alpha \neq \mcM_{\alpha+1}\) but \(\phi(\mcM_\alpha) = \phi(\mcM_{\alpha+1}\). 
The model given by \(\bigcup_{\alpha<\kappa} \mcM_\alpha\) will be of size \(\kappa\) (as there are \(\kappa\) many stages, each adding fewer than \(\kappa\) many elements but at least one).
By construction, \(\phi\) will still define a countable subset. 

\textcolor{red}{Insert Proof Here \ldots} 
\end{proof}

\begin{theorem}\label{theorem_vaughtian_pairs_categoricity}
Any theory which has Vaughtian pairs cannot be \(\kappa\)-categorical for any uncountable \(\kappa\).
\end{theorem}

\begin{proof}
As we stated above, we can also create a model of \(T\) of size \(\kappa\) where all infinite definable sets are of size \(\kappa\). 
Such a model wouldn't be isomorphic to the model we just created which has a countable definable subset. 
\end{proof}
