\subsection{Definition}
Like \(\omega\)-instability, lacking Vaughtian pairs is a property of theories which prevents uncountable categoricity (we will prove this in Theorem \ref{theorem_vaughtian_pairs_categoricity}).
The content of Morley's Categoricity theorem (Theorem \ref{theorem_morleys_categoricity}) will be that these are the only two obstacles to uncountable categoricity. 

\begin{definition}\label{definition_definable_subset}
Let \(\mcM\) be an \(\mcL\)-structure and \(\phi\) and \(\mcL\) formula in \(n\) free variables. 
We define \(\phi(\mcM) = \{\bar{m} \in M^n \mid \mcM \models \phi(\bar{m})\}\).
If \(S \subseteq M^n = \phi(\mcM)\) for some \(\phi\) we say that \(S\) is a definable subset of \(\mcM\) and that it is defined by \(\phi\).
\end{definition}

\begin{definition}\label{definition_vaughtian_pairs}
Two distinct \(\mcL\)-structures \((\mcN, \mcM)\) form a Vaughtian pair if \(\mcM \prec \mcN\) and for some \(\mcL_\mcM\)-formula \(\phi\), \(\phi(\mcM) = \phi(\mcN)\) is infinite. 
\end{definition}

Even though we write \(\mcM \prec \mcN\), we write the Vaughtian pair in the other order, that is as \((\mcN, \mcM)\), because we can think of this pair of models as a single model (\(\mcN\)) with a distinguished subset corresponding to \(\mcM\).
This will often be useful in proving theorems and lemmas about Vaughtian pairs, so we will flesh out the construction here. 

If \(\mcM, \mcN\) are \(\mcL\)-structures, let  \(\mcL^U = \mcL \cup \{U\}\). 
We look at (\(\mcN, \mcM\)) as an \(\mcL^U\)-structure which shares its underlying set and interpretations of \(\mcL\) with \(\mcN\) but where \(U\) is the subset \(M\). 
Note that, if \(\phi\) is an \(\mcL\)-formula \((\mcN, \mcM) \models \phi \iff \mcN \models \phi\).
Adding the unary predicate, \(U\), to our language allows us to find an \(\mcL^U\)-formula \(\phi_U\) corresponding to any \(\mcL\)-formula \(\phi\) such that \((\mcN, \mcM) \models \phi \iff \mcM \models \phi_U\).

We define \(\phi_U\) as follows:
\begin{enumerate}
\item If \(\phi(\bar{v})\) is atomic, we let \(\phi_U(\bar{v}) = \bigwedge_{v \in \bar{v}}U(v) \land \phi(\bar{v})\)
\item If \(\phi(\bar{v}) = \neg \psi(\bar{v})\) then \(\phi_U(\bar{v}) = \neg \psi_U(\bar{v})\)
\item If \(\phi(\bar{v}) = (\psi \land \theta)(\bar{v})\) then \(\phi_U(\bar{v}) = \psi_U(\bar{v}) \land \theta_U(\bar{v})\)
\item If \(\phi(\bar{v}) = \E x \psi(x, \bar{v})\) then \(\phi_U(\bar{v}) = \E x (U(x) \land \psi_U(x, \bar{v}))\)
\end{enumerate}
A simple induction shows that \((\mcN, \mcM) \models \phi \iff \mcM \models \phi_U\).

Before we proceed, it will be useful to develop some intuition about why Vaughtian pairs prevent uncountable categoricity for theories in countable languages, \(T\).
\begin{theorem}\label{theorem_vaughtian_pair_categoricity}
If \(\mcM \models T\), \(|\mcM| = \kappa > \aleph_0\) and \(\phi(\mcM)\) is countably infinite then \(T\) is not \(\kappa\)-categorical and \(T\) has a Vaughtian pair. 
\end{theorem}

\begin{proof}
We first show that then there is a Vaughtian pair for \(T\).
It suffices to let \(\mcM^* \models T^*\) be as in Theorem \ref{theorem_skolem_function_extension}.
We see that \((\mathcal{H}(\phi(\mcM)), \mcM)\) form a Vaughtian pair. 
Additionally, \(T\) cannot be \(\kappa\) categorical as we can construct an \(\mcN \models T\) with \(|\mcN| = \kappa\) and \(|\phi(\mcN)| = \kappa\) (an isomorphism \(f: \mcM \to \mcN\) would be bijection when restricted to \(\phi(\mcM)\) and by definition there is no bijection from a countable set to any uncountable \(\kappa\)).
We simple add constant symbols \(\{\bar{c}_i \mid i \in \kappa\}\) to our language. 
We can easily see that \(\Gamma = \text{Th}(\mcM) \cup \{\bar{c}_i \neq \bar{c}_j \mid i \neq j\} \cup \{\phi(\bar{c}_i) \mid i \in \kappa\}\) is finitely satisfiable (it suffices to find \(n\) distinct realizations of \(\phi\) in \(\mcM\) for all \(n \in \N\) which is possible as \(\phi(\mcM)\) is infinite). 
Any \(\mcN \models \Gamma\) will have \(|\phi(\mcN)| = \kappa\). 
\end{proof}

We see the connection between some Vaughtian pairs and lacking \(\kappa\)-categoricity, specifically Vaughtian pairs \((\mcN, \mcM)\) for which \(|\mcN| = \kappa\) and \(|\mcM| = \aleph_0\).
It would suffice to show that whenever there is any Vaughtian pair for an \omst theory \(T\), we can construct a model of \(T\) of any cardinality \(\kappa\) which has a countable definable subset. 
If \(T\) is not \omst then Theorem \ref{theorem_omega_stability_categoricity} tells us \(T\) is not \(\kappa\)-catgeorical.
Otherwise, this construction, coupled with Theorem \ref{theorem_vaughtian_pair_categoricity} tells us \(T\) isn't \(\kappa\)-categorical.
The rest of this chapter will show this construction. 

\begin{theorem}\label{theorem_countable_vaughtian_pairs}
If \(\mcL\) is countable and an \(\mcL\)-theory \(T\) has a Vaughtian pair \(\mcN, \mcM\), it has a Vaughtian pair \((\mcN_0, \mcM_0)\) of countable models. 
\end{theorem}

\begin{proof}
It will suffice to show that there is an \(\mcL^U\)-theory, \(T'\), for which \((\mcA, \mcB) \models T'\) iff \((\mcA, \mcB)\) forms a Vaughtian pair for \(T\). 
As \((\mcN, \mcM)\) forms a Vaughtian pair, \(T'\) is satisfiable.
As \(\mcL^U\) is countable, it has a countable model, \((\mcN_0, \mcM_0)\), by the downward L\"owenheim-Skolem theorem.
This is the desired countable Vaughtian pair. 

Our goal is to demonstrate an \(\mcL_\mcM^U\)-theory, \(\Gamma\), asserting that \((\mcN, \mcM)\) is Vaughtian pair.

\(\Gamma\) contains the following sets of formulae:
\begin{itemize}
\item \(T\) and \(\{\phi_U | \phi \in T\}\)
\item \(\A \bar{v} (U(\bar{v}) \land \E x \phi(x, \bar{v})) \to \E x U(x) \land \phi(x, \bar{v})\) for all \(\phi\). (By Theorem \ref{theorem_tarski_vaught_test} this shows \(\mcM \prec \mcN\))) 
\item If \(\psi(\bar{v})\) is the \(\mcL_\mcM\)-formula which isolates an infinite subset of \(\mcM\), for all \(n \in \N\), \(\Gamma\) includes \(\E \bar{v}_1, \E \bar{v}_2, \ldots \E \bar{v}_n \bigwedge \bar{v}_i \neq \bar{v}_j \land \bigwedge \phi(\bar{v}_i)\). That is \(\psi\) defines an infinite subset of \(\mcN\).
\item \(\A \bar{v} (\psi(\bar{v}) \to U(\bar{v}))\). That is, \(\psi(\mcN) \subset \psi(\mcM)\). As the other direction is trivial, this ensures \(\psi(\mcN) = \psi(\mcM)\). 
\item Finally, we assert that \(\mcN\) is a proper elementary extension of \(\mcM\) with \(\E x \neg U(x)\). 
\end{itemize}

By construction, \(\Gamma\) is an \(\mcL^U\)-theory, \(T'\), for which \((\mcA, \mcB) \models T'\) iff \((\mcA, \mcB)\) forms a Vaughtian pair for \(T\). 
The L\"owenheim-Skolem theorem gives us our desired model. 
\end{proof}

For the next few claims we will need to use a property called homogeneity which will allow us to show that certain countable models are isomorphic.

\begin{definition}\label{definition_partial_elementary}
Let \(\mcM\) be an \(\mcL\) function and \(A \subseteq M\).
A function \(f:A \to M\) for which \(\mcM \models \phi(\bar{v}) \iff \mcM \models \phi(f(\bar{v}))\) is called ``partial elementary''.
\end{definition}

\begin{definition}\label{definition_homogeneity}
A countable model \(\mcM\) is homogeneous if all finite \(A \subseteq M\), partial elementary functions \(f: A \to M\) and \(m \in M\), there is a partial elementary \(f': A \cup \{m\} \to M\) extending \(f\) which is partial elementary. 
% Examples?
\end{definition}

\begin{theorem}\label{theorem_partial_elementary_automorphism}
Any partial elementary map \(f: A \to M\) (where \(A \subseteq M\) is finite) can be extended to an automorphism. 
\end{theorem}

\begin{proof}
We define an isomorphism extending \(f\) inductively. 
Let \(f_0 = f\) be the first function in our chain.
We will define \(f_i\) for \(i > 0\) such that \(f_{i-1} \subseteq f_i\).
We will alternatively extend the domain and range of \(f_i\) by elements in \(M = \{m_1, m_2, \ldots\}\).
That is, if \(i\) is odd, \(m_i \in \text{dom}(f_{\frac{i-1}{2}})\) and if \(i\) is even \(m_i \in \text{rng}(f_{\frac{i}{2}})\).
The former is possible as homogeneity allows us to expand the domain of \(f_{i-1}\) to include \(m_i\).
For the case of \(i\) even, note that partial elementary maps have partial elementary inverses (if \(f\) is partial elementary, it is injective). 
That partial elementary functions have partial elementary inverses allows us to extend the range of \(f_{i-1}^{-1}\) to include \(m_i\). 
Taking the inverse of the resulting function gives us \(f_i\).
Let \(f^* = \bigcup_{i \in \N}f_i\).
\(f^*\) has domain and range equal to all of \(M\) and is partial elementary, making \(f^*\) our desired automorphism.
\end{proof}



\begin{lemma}\label{lemma_types_isomorphism}
The next thing we note about countable homogeneous models, \(\mcM\) and \(\mcN\), of the same complete theory, \(T\), is that if \(\mcM\) and \(\mcN\) realize the same types in \(S_n(T)\), they are isomorphic.
\end{lemma}

\begin{proof}
To show this, we will build a map in a way very similar to the way we built \(f^*\) in Theorem \ref{theorem_partial_elementary_automorphism}
We start by letting \(f_0 = \emptyset\), which is partial elementary if \(T\) is complete. . 
We will enumerate the elements in the universe of both models, and iteratively add elements to the domain and range of a partial elementary map as above.
If we can do this, the union of this sequence of maps will be an isomorphism.

As the inverse of a partial elementary map is also partial elementary, it will suffice to show that, for an arbitrary  partial elementary map \(f\) and element \(n \in N\), we can find \(f'\subseteq f\) with \(m \in \text{dom}(f')\).
Let \(\text{dom}(f) = \{m_1, \ldots, m_i\}, \text{rng}(f) = \{n_1, \ldots, n_i\}, f(m_j) = n_j\).
Note that we can find a partial elementary \(h\) for which \(dom(f) \cup \{m\} =  dom(h)\) by merely finding the realization of \(\text{tp}(\bar{m}, m)\) in \(\mcN\), which must exist by hypothesis.
Our one remaining obstacle is that \(h(m_i)\) may not be equal to \(n_i\) for all \(i\). 
This can be rectified with homogeneity. 
We can easily see that the function \(g: h(m_i) \mapsto n_i\) and which is undefined everywhere else, is a partial elementary map which can be extended to an automorphism by Theorem \ref{theorem_partial_elementary_automorphism}. 
Call this automorphism \(g'\). 
We take \(f' = g' \circ h\).
\(f'\) is partial elementary, extends \(f\) and has \(m\) in its domain.   

We alternate adding elements of \(M\) and \(N\) to domain and range of \(f_i\) as above and take the union of these functions.
Their union will be a partial elementary bijection defined on all of \(M\) and therefore an isomorphism
\end{proof}

\begin{theorem}\label{theorem_countable_isomorphic_vaughtian_pair}
A Vaughtian pair of countable models, (\(\mcN_0, \mcM_0\)), can be extended to a Vaughtian pair of isomorphic countable models.
\end{theorem}

% TODO A: Revise this proof
%%%%%%%%%%%%%%%%%%%%%%%%%%%%%%%%%%%%%%%%%%%%%%%%%%%%%%%%%%%%%%%%%%%%%%%%%%%%%%%%
\begin{proof}%{\color{red} I think this proof will get a rewrite.}
By above, it suffices to show a vaughtian pair \((\mcN, \mcM)\) can be extended to a vaughtian pair \((\mcN^*, \mcM^*)\) where \(\mcN^*, \mcM^*\) are homogeneous and realize the same types. 

We aim to construct a chain of Vaughtian pairs indexed by the natural numbers such that each pair is an elementary extension of the previous one and their union will be homogeneous and realize the same types. 
The zeroth pair is the countable pair (\(\mcN_0, \mcM_0\)) from above. 
For \(i > 0\) we construct (\(\mcN_i, \mcM_i\)) as follows (the details of constructions in each of these stages will follow):

\begin{enumerate}

  \item  If \(i\) is a multiple of 3, then (\(\mcN_i, \mcM_i\)) is an elementary extension of (\(\mcN_{i-1}, \mcM_{i-1}\)) such that for every \(\bar{a} \in M\) every type in \(S_n(\bar{a})\) realized in \(\mcN_{i-1}\) is realized in \(\mcM_i\). %make sure vector notation is good. 

  \item  If \(i\) is one more than a mutliple of 3, then (\(\mcN_i, \mcM_i\)) is an elementary extension of (\(\mcN_{i-1}, \mcM_{i-1}\)) such that if \(\bar{a}, \bar{b}\) realize the same type in \(\mcM_{i-1}\) and \(c \in \mcM_{i-1}\), then there is a \(d \in \mcM_i\) such that \(\bar{a}c\) and \(\bar{b}d\) realize the same type in \(\mcM_i\). 

  \item If \(i\) is two more than a mutliple of 3, then (\(\mcN_i, \mcM_i\)) is an elementary extension of (\(\mcN_{i-1}, \mcM_{i-1}\)) such that if \(\bar{a}, \bar{b}\) realize the same type in \(\mcN_{i-1}\) and \(c \in \mcN_{i-1}\), then there is a \(d \in \mcN_i\) such that \(\bar{a}c\) and \(\bar{b}d\) realize the same type in \(\mcN_i\).

\end{enumerate}

If we let \((\mcN, \mcM) = \bigcup_{i \in \N}(\mcN_i, \mcM_i)\), we see that every type realized in \(\mcN\) is realized in some \(\mcN_i\) and therefore in \(\mcM_{i+3}\) (as we must have done the first stage of our construction once in the interim) and thus in \(\mcM\). 
Similarly, any type realized in \(\mcM\) must be realized in some \(\mcM_i\) and in \(\mcN_i\) as well (as \(\mcM \prec \mcN\)) and thus in \(\mcN\).
Thus, \(\mcM\) and  \(\mcN\) realize the same types.  
Given a partial elementary map \(f:\mcM \to \mcM\) with finite domain, we have \(\text{dom}(f) \in \mcM_i\) for some \(i\), and stage two of our construction guaruntees that it can be extended. 
Similarly for homogeneity of \(\mcN\) by the third stage of our construction. 

At this point it would suffice to show how to construct the elementary extensions mentioned in our three stages. 

\begin{enumerate}

\item As we can enumerate the countably many types over finite sets realized in any countable \(\mcN\) (there are countably many finite sequences from a countable set and countably many \(\bar{a}\)), it would suffice to show that if \(p(\bar{v})\) is an \(\bar{a}\)-type and \((\mcN, \mcM)\) is a Vaughtian pair with \(\mcN\) realizing \(p(\bar{v})\) that there is a \((\mcN, \mcM) \prec (\mcN', \mcM')\) with \(\mcM'\) realizing \(p(\bar{v})\). 
We can see that \(\Gamma = \eldiag((\mcN, \mcM)) \cup \{\phi_U(\bar{v}, \bar{a}) \mid \phi(\bar{v}, \bar{a}) \in  p(\bar{v})\}\) is satisfiable.  
Specifically, if \(\phi_1(\bar{v}, \bar{a}), \ldots, \phi_k(\bar{v}, \bar{a}) \in p\) are a finite subset of this type, we know that \(\mcN \models \bigwedge \E \bar{x} \phi_k(\bar{x},\bar{a})\) and therefore that \(\mcM \models \bigwedge \E \bar{x} (\phi_k)_U(\bar{x},\bar{a})\). 
This finite subset if realized in \((\mcN, \mcM)\) itself. 
Any model of \(\Gamma\) is an elementary extension of \((\mcN, \mcM)\) which we may call \((\mcN', \mcM')\).
We can also take \((\mcN', \mcM')\) to be countable. 
As \((\mcN', \mcM') \models \Gamma\) we also have that \(p(\bar{v})\) is realized in \((\mcN', \mcM')\). 

\item The construction for this stage is very similar to the one covered just above for the first stage.
Again, it will suffice to show that just one such type can be realized (namely the type of \(c\) over \(\bar{a}\)) as we can add realizations for our countably many types one at a time. 
Here, we take \(\Gamma = \eldiag((\mcN, \mcM)) \cup \{\phi(\bar{b}, d) \mid \phi(\bar{a}, c)\}\). 
If some finite subset, \(\Phi\), were not realizable, we would have \(\mcM \models \exists x \Phi(\bar{a})\) but this formuale wouldn't hold for \(\bar{b}\), giving a contradiction. 

\item Here we let \(\Gamma = \eldiag((\mcN, \mcM)) \cup \{\phi(\bar{b}, d) \mid \mcN \models \phi(\bar{a}, c)\}\) where \(d\) is a new constant symbol.
If \(\Gamma\) is not finitely satisfiable that must mean that for some \(\phi\) we have \(\mcN \models \phi(\bar{a}, c) \land \neg \E x \phi(\bar{b}, x)\), 
so \(\bar{a}, \bar{b}\) cannot have the same type, a contradiction.

\end{enumerate}

\end{proof}
%%%%%%%%%%%%%%%%%%%%%%%%%%%%%%%%%%%%%%%%%%%%%%%%%%%%%%%%%%%%%%%%%%%%%%%%%%%%%%%%

\begin{lemma}\label{lemma_extend_isomorphically}
Let \((\mcN, \mcM)\) be a vaughtian pair of countable models, \(\mcN'\) a countable model and \(f:\mcN' \to \mcM\) is an isomorphism. 
We claim that there must exist an \(\mcN''\) such that \((\mcN'', \mcN') \isom (\mcN, \mcM)\). 
\end{lemma}

\begin{proof}
Let \(N'' = N' \cup (N \setminus M)\).
Define \(f'': N'' \to N\) as follows: if \(x \in N'\) then \(f'(x) = f(x)\). 
Otherwise, \(f'(x) = x\). 
By construction, \(f'\) is a bijection between \(N''\) and \(N\) which, when restricted to \(N'\) is a bijection between  \(N'\) and \(M\). 
For atomic \(\mcL^U\)-formulae \(\phi\), let \((\mcN'', \mcN') \models \phi(\bar{v}) \iff (\mcN, \mcM) \models \phi(f'(\bar{v}))\).
This gives us interpretations of \(\mcL\) for \(\mcN''\) such that \(f'\) is an \(\mcL\)-isomorphism by construction. 
\end{proof}

\begin{theorem}\label{theorem_aleph_one_vaughtian_pair}
Given a Vaughtian pair of countable isomorphic models, \((\mcN, \mcM) \models T\), we can construct a model \(\mcN^* \models T\) which is of cardinality \(\aleph_1\) and has a definable subset which is countable.  
\end{theorem}

% Consider discussing elementary chains

\begin{proof}
We will construct \(\mcN^*\) as the union of an elementary chain of models \(\mcN_\alpha\) indexed by \(\omega_1\).

Let \(\mcN_0 = \mcN\). 
We'd like to maintain \(\mcN_\alpha \isom \mcN\) and \(\phi(\mcN_\alpha) = \phi(\mcN)\) and \(\mcN_\alpha\) is countable. 
This is true by hypothesis for \(\mcN\).

If \(\alpha + 1\) is not a limit ordinal, let \(\mcN_{\alpha + 1}\) be a countable model of \(T\) such that \((\mcN_{\alpha+1}, \mcN_\alpha) \isom (\mcN, \mcM)\) which exists by Lemma \ref{lemma_extend_isomorphically}. 
As \((\mcN_{\alpha+1}, \mcN_\alpha)\) is a Vaughtian pair, we've ensured that \(\mcN_\alpha \prec \mcN_{\alpha+1}\) and \(\mcN_\alpha \neq \mcN_{\alpha+1}\).
We have \(\phi(\mcN_{\alpha + 1}) = \phi(\mcN_\alpha)\) and by hypothesis \(\phi(\mcN_\alpha) = \phi(\mcM)\). 
Similarly,  \(\mcN_{\alpha+1} \isom \mcN_\alpha \isom \mcN\). 

To define \(\mcN_\alpha\) where \(\alpha\) is a limit ordinal, let \(\mcN_\alpha = \bigcup_{\beta < \alpha}\mcN_\beta\). 
As \(\mcN_\alpha\) is the union of countably many countable sets, it is countable. 
If \(\bar{a} \in \mcN_\alpha\) then \(\bar{a} \in \mcN_\beta\) for some \(\beta < \alpha\). 
As \(\mcN_\beta \prec \mcN_\alpha\), \(\text{tp}^{\mcN_\beta}(\bar{a}) = \text{tp}^{\mcN_\alpha}(\bar{a})\). 
As \(\mcN_\beta \isom \mcM\), this type is realized in \(\mcN\).
Thus, every from \(S_n(T)\) realized in \(\mcN_\alpha\) is realized in \(\mcN\). 
The converse follows from \(\mcN \prec \mcN_\alpha\). 
Thus \(\mcN \isom \mcN_\alpha\).

If \(\mcN^* = \bigcup_{\alpha < \omega_1}\mcN_\alpha\) we know that \(\phi(\mcN^*) = \phi(\mcN) \subset \mcM\) which is countable.
As \(\mcN \prec \mcN^*\) we have \(\mcN^* \models T\).  
As we add at most countably many but at least one element at each of the \(\aleph_1\) stages, we have \(|\mcN^*| = \aleph_1\).
\end{proof}

\begin{theorem}\label{theorem_uncountable_vaughtian_pairs}
Given \(\mcM \models T\) where \(|M| = \aleph_1\), \(T\) is \(\omega\)-stable and \(\phi(\mcM)\) is countably infinite, for every uncountable cardinal \(\kappa\), there is a model of \(T\) of cardinality \(\kappa\) with a countably infinite definable subset.  
\end{theorem}

\begin{proof}
As in Theorem \ref{theorem_aleph_one_vaughtian_pair}, it suffices to show that if \(\mcM\) is uncountable has a countably infinite definable subset defined by \(\phi\), that there is an \(\mcM \prec \mcN\) which is a strict elementary extension such that \(\phi(\mcN) = \phi(\mcM)\) and \(|\mcM| = |\mcN|\).

First, we show there is an \(\mcL_\mcM\)-formula, \(\psi\) such that there are uncountably many realizations of \(\psi(v)\) in \(\mcM\) and for all \(\mcL_\mcM\)-formulae, \(\theta\), either \(\psi \land \theta(v)\) or \(\psi \land \neg\theta(v)\) has only countably many realizations. 

For contradiction, we build an infinite binary tree of formulae such that all paths in the tree are types over a countably set, and all are satisfiable, contradicting \(\omega\)-stability.  
% as in reference
The root of the tree is \(v = v\), which has uncountably many realizations. 
As this formula is not our desired \(\psi\), there is a \(\theta\) such that \((v = v) \land \theta(v)\) and \((v = v) \land \neg \theta(v)\) have uncountably many realizations. 
Every path in this tree is a finitely satisfiable type over the countable set of constants from \(\mcM\) appearing in the tree as in %reference. 
.
This contradicts \(\omega\) stability. 

Given \(\psi\) as above, we now construct a type over \(M\), \(p\), the type of an element not in any of the countable definable subsets of \(\psi(\mcM)\). 
That is \(p = \{\theta(v) \mid |(\theta \land \psi (\mcM))| \geq \aleph_1\}\).
This type is finitely satisfiable. If we take \(\theta_1, \ldots, \theta_n\) from \(p\) then each fails in on at most countably many elements. 
Their conjunction also fails to hold of at most countably many elements. 
This is actually a complete type as, if \(\theta \in p\) exactly when it is true of uncountably many elements of \(\mcM\).
Note that \(\mcM\) cannot contain a realization of \(p\) (for each element \(m \in M\), \(p\) contains \(v \neq m\)). 
We realize \(p\) in \(\mcM'\) such that \(\mcM \prec \mcM'\). 
Let \(c\) be the realization of \(p\). 
By theorem % to be proven later
we can find a \(\mcM_0 \prec \mcM'\) prime over \(M \cup \{c\}\) (specifically, \(\mcM \prec \mcM_0\)).

All that remains is to show that \(\mcM_0\) has the desired property, that \(\phi(\mcM) = \phi(\mcM_0)\).   
It suffices to show that the type \(\Gamma(v) = \{\phi(v)\} \cup \{v \neq m \mid m \in \phi(\mcM)\}\) is not realized. 
Assume for the purposes of contradiction that it is realized in \(\mcN\) by some \(v\), and we will show \(\Gamma\) must be realized in \(\mcM\) as well. 
There must be an \(\mcL_\mcM\) formula \(\psi(x, y)\) such that \(\psi(v, c)\) isolates \(\text{tp}(v/M \cup \{c\})\).% Why? Do something with the prime models
Note that the formula \(\E x \psi(x, c)\) is true of \(c\) and is therefore in \(p(c)\). 
Moreover, if \(\gamma(v) \in \Gamma(v)\) then \(\A v (\psi(v, c) \to \gamma(v)) \in p(c)\). 
Let \(\Delta(c) = \{\A v (\psi(v, c) \to \gamma(v)) \mid \gamma \in \Gamma\} \cup \{\E v \psi(v, c)\}\).
\(\Delta\) is countable and a subset of \(p\) so it is realized by \(c\). 
If another \(d\) realizes \(\Delta\) then \(\E w \psi(w, d)\) and \(w\) realizes \(\Gamma\).
Let \(\{\delta_0(c), \delta_1(c), \ldots\} = \Delta(c)\). 
\(\{x \in M \mid \psi(x)\}\) must be uncountable. 
For every \(n\), though, \(|\{x \in M \mid \psi(x) \land \neg (\bigwedge\limits_{i=0}^n\delta_i)\}|\) is countable (by the definition of \(p \ni \delta_i\)).
So only countably many \(x\) realizing \(\psi\) fail to realize \(\Delta\), let \(d\) be one of the uncountably many realizing \(\Delta\) in \(M\).
As above, \(w\) realizes \(\Gamma\) in \(\mcM\). This is our contradiction.  

\end{proof}

\begin{theorem}\label{theorem_vaughtian_pairs_categoricity}
Any theory, \(T\), which has Vaughtian pairs cannot be \(\kappa\)-categorical for any uncountable \(\kappa\).
\end{theorem}

\begin{proof}
If \(T\) is not \omst it cannot be categorical for any uncountable \(\kappa\) by Theorem \ref{theorem_omega_stability_categoricity}.
Otherwise, as \(T\) has a Vaughtian pair, by Theorem \ref{theorem_countable_vaughtian_pairs} it has a Vaughtian pair of countable models. 
By Theorem \ref{theorem_countable_isomorphic_vaughtian_pair} it has a Vaughtian pair of countable isomorphic models.
By Theorem \ref{theorem_aleph_one_vaughtian_pair} it has a model of size \(\aleph_1\) with a countable definable subset. 
By Theorem \ref{theorem_uncountable_vaughtian_pairs} it has a model of size \(\kappa\) which has a countable subset.  
Theorem \ref{theorem_vaughtian_pair_categoricity} tells us that \(T\) cannot be \(\kappa\)-categorical. 
\end{proof}
