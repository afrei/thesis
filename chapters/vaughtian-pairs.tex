\subsection{Definition}
Like \(\omega\)-instability, lacking Vaughtian pairs is a property of theories which prevents uncountable categoricity (we will prove this in Theorem \ref{theorem_vaughtian_pairs_categoricity}).
The content of Morley's Categoricity theorem (Theorem \ref{theorem_morleys_categoricity}) will be that these are the only two obstacles to uncountable categoricity. 

\begin{definition}\label{definition_definable_subset}
Let \(\mcM\) be an \(\mcL\)-structure and \(\phi\) an \(\mcL\) formula in some number of free variables.
We define \(\phi(\mcM) = \{\bar{m} \in M \mid \mcM \models \phi(\bar{m})\}\).
If \(S = \phi(\mcM)\) for some \(\phi\) we say that \(S\) is a definable subset of \(\mcM\) and that it is defined by \(\phi\).
\end{definition}

\begin{definition}\label{definition_vaughtian_pairs}
Two distinct \(\mcL\)-structures \((\mcN, \mcM)\) form a Vaughtian pair if \(\mcM \prec \mcN\) and for some \(\mcL_\mcM\)-formula \(\phi\), \(\phi(\mcM) = \phi(\mcN)\) is infinite. 
\end{definition}

Even though we write \(\mcM \prec \mcN\), we write the Vaughtian pair in the other order, that is as \((\mcN, \mcM)\), because we can think of this pair of models as the model \(\mcN\) with a distinguished subset corresponding to \(\mcM\), that is a subset defined by a unary predicate.
This will often be useful in proving theorems and lemmas about Vaughtian pairs, so we will flesh out the construction here. 

If \(\mcM, \mcN\) are \(\mcL\)-structures, let  \(\mcL^U = \mcL \cup \{U\}\). 
We look at (\(\mcN, \mcM\)) as an \(\mcL^U\)-structure which shares its underlying set and interpretations of \(\mcL\) with \(\mcN\) but where \(U\) is the subset \(M\). 
Note that, if \(\phi\) is an \(\mcL\)-formula \((\mcN, \mcM) \models \phi \iff \mcN \models \phi\).
Adding the unary predicate, \(U\), to our language allows us to find an \(\mcL^U\)-formula \(\phi_U\) corresponding to any \(\mcL\)-formula \(\phi\) such that \((\mcN, \mcM) \models \phi_U \iff \mcM \models \phi\).

We define \(\phi_U\) as follows:
\begin{enumerate}
\item If \(\phi(\bar{v})\) is atomic, we let \(\phi_U(\bar{v}) = \bigwedge\limits_{v \in \bar{v}}U(v) \land \phi(\bar{v})\)
\item If \(\phi(\bar{v}) = \neg \psi(\bar{v})\) then \(\phi_U(\bar{v}) = \neg \psi_U(\bar{v})\)
\item If \(\phi(\bar{v}) = (\psi \land \theta)(\bar{v})\) then \(\phi_U(\bar{v}) = \psi_U(\bar{v}) \land \theta_U(\bar{v})\)
\item If \(\phi(\bar{v}) = \E x \, \psi(x, \bar{v})\) then \(\phi_U(\bar{v}) = \E x\, (U(x) \land \psi_U(x, \bar{v}))\)
\end{enumerate}
A simple induction shows that \((\mcN, \mcM) \models \phi_U \iff \mcM \models \phi\).

The following is now an easy consequence of the downward L\"owenheim-Skolem theorem:

\begin{theorem}\label{theorem_countable_vaughtian_pairs}
If \(\mcL\) is countable and an \(\mcL\)-theory, \(T\), has a Vaughtian pair \(\mcN, \mcM\), it has a Vaughtian pair \((\mcN_0, \mcM_0)\) of countable models. 
\end{theorem}

\begin{proof}
It will suffice to show that there is an \(\mcL^U\)-theory, \(T'\), for which \((\mcA, \mcB) \models T'\) iff \((\mcA, \mcB)\) forms a Vaughtian pair for \(T\) with \(\phi(\mcB) = \phi(\mcB)\) an infinite set. 
As \((\mcN, \mcM)\) forms a Vaughtian pair, \(T'\) is satisfiable.
As \(\mcL^U\) is countable, it has a countable model, \((\mcN_0, \mcM_0)\), by the downward L\"owenheim-Skolem theorem.
This is the desired countable Vaughtian pair. 

Our goal is to demonstrate an \(\mcL_\mcM^U\)-theory, \(\Gamma\), asserting that \((\mcN, \mcM)\) is Vaughtian pair.

\(\Gamma\) is the union of the following sets of formulae:
\begin{itemize}
\item \(T\) and \(\{\phi_U \mid \phi \in T\}\). (That is \(\mcM, \mcN \models T\).)
\item \(\A \bar{v} \, (U(\bar{v}) \land \E x \, \phi(x, \bar{v})) \to \E x \, U(x) \land \phi(x, \bar{v})\) for all \(\phi\). 

(By Theorem \ref{theorem_tarski_vaught_test} this shows \(\mcM \prec \mcN\)) 

\item If \(\psi(\bar{v})\) is the \(\mcL_\mcM\)-formula which isolates an infinite subset of \(\mcM\), we take the set \(\{\psi_n \mid n \in \N\}\) where \(\psi_n = |\{\bar{v} \mid \psi(\bar{v})\}| > n\).\footnote{\(\psi_n\) is first order.
Specifically, \(\psi_n = \E x_1 \ldots \E x_n \, \left(\bigwedge\limits_{0 <i < j \leq n} x_i \neq x_j\right) \land \left(\bigwedge\limits_{0 < i \leq n} \psi(x_i)\right)\). Similarly, we can assert \(|\{\bar{v} \mid \psi(\bar{v})\}| = n\) as \((\psi_n \land \neg \psi_{n+1})\).}
That is \(\psi\) defines an infinite subset of \(\mcN\).

\item \(\A \bar{v} \, (\psi(\bar{v}) \to U(\bar{v}))\). That is, \(\psi(\mcN) \subseteq \psi(\mcM)\). As the other direction is trivial, this ensures \(\psi(\mcN) = \psi(\mcM)\). 
\item Finally, we assert that \(\mcN\) is a proper elementary extension of \(\mcM\) with \(\E x \, \neg U(x)\). 
\end{itemize}

By construction, \(\Gamma\) is an \(\mcL^U\)-theory, \(T'\), for which \((\mcA, \mcB) \models T'\) iff \((\mcA, \mcB)\) form a Vaughtian pair for \(T\) (where the required infinite set is defined by \(\psi\)). 
The L\"owenheim-Skolem theorem gives us our desired countable model. 
\end{proof}

\subsection{Intuition for Vaughtian Pairs and Categoricity}

Before we proceed, it will be useful to develop some intuition about why Vaughtian pairs prevent uncountable categoricity for theories in countable languages.

\begin{theorem}\label{theorem_vaughtian_pair_categoricity}
If \(\mcM \models T\), \(|\mcM| = \kappa > \aleph_0\) and \(\phi(\mcM)\) is countably infinite then \(T\) is not \(\kappa\)-categorical and \(T\) has a Vaughtian pair. 
\end{theorem}

\begin{proof}
We first show that there is a Vaughtian pair for \(T\).
We have established that we can assume without loss of generality that \(T\) has built in Skolem functions. 
If we take the submodel of \(\mcM\) generated by \(\phi(\mcM)\) this will be a countable (and therefore proper) elementary submodel of \(\mcM\) in which \(\phi\) will define the same subset, \(\phi(\mcM)\).
This provides the desired Vaughtian pair. 
Additionally, \(T\) cannot be \(\kappa\) categorical as we can construct an \(\mcN \models T\) with \(|\mcN| = \kappa\) and \(|\phi(\mcN)| = \kappa\) (an isomorphism \(f: \mcM \to \mcN\) would be bijection when restricted to \(\phi(\mcM)\) and by definition there is no bijection from a countable set to any uncountable \(\kappa\)).
We simply add constant symbols \(\{\bar{c}_i \mid i \in \kappa\}\) to our language. 
We can easily see that \(\Gamma = \text{Th}(\mcM) \cup \{\bar{c}_i \neq \bar{c}_j \mid i \neq j\} \cup \{\phi(\bar{c}_i) \mid i \in \kappa\}\) is finitely satisfiable (it suffices to find \(n\) distinct realizations of \(\phi\) in \(\mcM\) for all \(n \in \N\) which is possible as \(\phi(\mcM)\) is infinite). 
Any \(\mcN \models \Gamma\) will have \(|\phi(\mcN)| = \kappa\). 
\end{proof}

We see the connection between some Vaughtian pairs and lacking \(\kappa\)-categoricity, specifically Vaughtian pairs \((\mcN, \mcM)\) for which \(|\mcN| = \kappa\) and \(|\mcM| = \aleph_0\).
It will suffice to show that whenever there is a Vaughtian pair for an \omst theory \(T\), we can construct a model of \(T\) which has a countable definable subset of any uncountable cardinality \(\kappa\). 
If \(T\) is not \omst then Theorem \ref{theorem_omega_stability_categoricity} tells us \(T\) is not \(\kappa\)-categorical.
Otherwise, this construction, coupled with Theorem \ref{theorem_vaughtian_pair_categoricity} tells us \(T\) isn't \(\kappa\)-categorical.
The rest of this chapter will show this construction. 

\subsection{Homogeneity}

For the next few claims we will need to use a property called homogeneity which will allow us to show that certain countable models are isomorphic.

\begin{definition}\label{definition_homogeneity}
A countable model \(\mcM\) is homogeneous if, for all finite \(A \subseteq M\) and partial elementary functions \(f: A \to M\) and \(m \in M\), there is a partial elementary \(f' \supseteq f\) with domain \(A \cup \{m\}\) (taking values in \(M\)). 
\end{definition}

\begin{lemma}\label{theorem_partial_elementary_automorphism}
Any partial elementary map \(f: A \to M\) (where \(A \subseteq M\) is finite) can be extended to an automorphism. 
\end{lemma}

\begin{proof}
We define an isomorphism extending \(f\) inductively. 
Let \(f_0 = f\) be the first function in our chain.
We will define \(f_i\) for \(i > 0\) such that \(f_{i} \supseteq f_{i-1}\).
We will alternatively extend the domain and range of \(f_i\) by elements in \(M = \{m_1, m_2, \ldots\}\).
That is, if \(i\) is odd, \(m_i \in \text{dom}(f_{\frac{i-1}{2}})\) and if \(i\) is even \(m_i \in \text{rng}(f_{\frac{i}{2}})\).
The former is possible as homogeneity allows us to expand the domain of \(f_{i-1}\) to include \(m_i\).
For the case of \(i\) even, note that partial elementary maps have partial elementary inverses (if \(f\) is partial elementary, it is injective). 
That partial elementary functions have partial elementary inverses allows us to extend the range of \((f_{i-1})^{-1}\) to include \(m_i\). 
Taking the inverse of the resulting function gives us \(f_i\).
Let \(f^* = \bigcup\limits_{i \in \N}f_i\).
\(f^*\) has domain and range equal to all of \(M\) and is partial elementary, making \(f^*\) our desired automorphism.

This is a ``back and forth'' argument, much like the one we saw in Example \ref{example_categoricity_sequence}. 
In retrospect, we see that the construction in that example relied on homogeneity. 
\end{proof}

\begin{lemma}\label{lemma_types_isomorphism}
The next thing we note about countable homogeneous models, \(\mcM\) and \(\mcN\), of the same complete theory, \(T\), is that if \(\mcM\) and \(\mcN\) realize the same types in \(S_n(T)\), they are isomorphic.
\end{lemma}

\begin{proof}
To show this, we will build a map in a way very similar to the way we built \(f^*\) in the previous Lemma \ref{theorem_partial_elementary_automorphism}.
We start by letting \(f_0 = \emptyset\), which is partial elementary if \(T\) is complete.
We will enumerate the elements in the universe of both models, and iteratively add elements to the domain and range of a partial elementary map as above.
If we can do this, the union of this sequence of maps will be an isomorphism.

As the inverse of a partial elementary map is also partial elementary, it will suffice to show that, for an arbitrary  partial elementary map \(f\) and element \(n \in N\), we can find \(f'\supseteq f\) with \(m \in \text{dom}(f')\).
Let \(\text{dom}(f) = \{m_1, \ldots, m_i\}, \text{rng}(f) = \{n_1, \ldots, n_i\}, f(m_j) = n_j\).
Note that we can find a partial elementary \(h\) for which \(dom(f) \cup \{m\} =  dom(h)\) by merely finding the realization of \(\text{tp}(\bar{m}, m)\) in \(\mcN\), which must exist by hypothesis.
Our one remaining obstacle is that \(h(m_i)\) may not be equal to \(n_i\) for all \(i\), that is, it is quite possible that \(f \not \subseteq h\). 
This can be rectified with homogeneity. 
We can easily see that the function \(g: h(m_i) \mapsto n_i\) and undefined elsewhere, is a partial elementary map which can be extended to an automorphism by Lemma \ref{theorem_partial_elementary_automorphism}. 
Call this automorphism \(g'\). 
We take \(f' = g' \circ h\).
\(f'\) is partial elementary, extends \(f\) and has \(m\) in its domain.   

We alternate adding elements of \(M\) and \(N\) to domain and range of \(f_i\) as above and take the union of these functions.
Their union will be a partial elementary bijection defined on all of \(M\) and therefore an isomorphism.
\end{proof}

\subsection{Uncountable Models with Definable Subsets}

\begin{theorem}\label{theorem_countable_isomorphic_vaughtian_pair}
A Vaughtian pair of countable models, (\(\mcN_0, \mcM_0\)), can be extended to a Vaughtian pair of isomorphic countable models.
\end{theorem}

\begin{proof}
By Lemma \ref{lemma_types_isomorphism}, it suffices to show a countable Vaughtian pair \((\mcN, \mcM)\) can be extended to a countable Vaughtian pair \((\mcN^*, \mcM^*)\) where \(\mcN^*, \mcM^*\) are homogeneous and realize the same types. 

We aim to construct a chain of Vaughtian pairs \((\mcN_i, \mcM_i)\) such that \((\mcN_i, \mcM_i) \prec (\mcN_{i+1}, \mcM_{i+1})\) and their union, \(\bigcup\limits_{i\in \N} (\mcN_i, \mcM_i)\) will be homogeneous and realize the same types. 
The first pair in our chain is \((\mcN_0, \mcM_0) = (\mcN, \mcM)\) from above. 
For \(i > 0\) we construct (\(\mcN_i, \mcM_i\)) as follows (the details of constructions in each of these stages will follow):

\begin{enumerate}

  \item  \(i\) is a multiple of 3: then (\(\mcN_i, \mcM_i\)) is an elementary extension of (\(\mcN_{i-1}, \mcM_{i-1}\)) such that every type in \(S_n(T)\) realized in \(\mcN_{i-1}\) is realized in \(\mcM_i\). (This ensures \(\mcN^*, \mcM^*\) realize the same types.)

  \item  If \(i = 3k+1\) , then (\(\mcN_i, \mcM_i\)) is an elementary extension of (\(\mcN_{i-1}, \mcM_{i-1}\)) such that if \(\bar{a}, \bar{b}\) realize the same type in \(\mcM_{i-1}\) and \(c \in \mcM_{i-1}\), then there is a \(d \in \mcM_i\) such that \(\bar{a}c\) and \(\bar{b}d\) realize the same type in \(\mcM_i\). (This will ensure \(\mcM^*\) is homogeneous.) 

  \item If \(i = 3k + 2\): then (\(\mcN_i, \mcM_i\)) is an elementary extension of (\(\mcN_{i-1}, \mcM_{i-1}\)) such that if \(\bar{a}, \bar{b}\) realize the same type in \(\mcN_{i-1}\) and \(c \in \mcN_{i-1}\), then there is a \(d \in \mcN_i\) such that \(\bar{a}c\) and \(\bar{b}d\) realize the same type in \(\mcN_i\). (This will ensure \(\mcN^*\) is homogeneous.) 

\end{enumerate}

If we let \((\mcN^*, \mcM^*) = \bigcup\limits_{i \in \N}(\mcN_i, \mcM_i)\), we see that every type realized in \(\mcN^*\) is realized in some \(\mcN_i\) and therefore in \(\mcM_{i+3}\) (as we must have done the first stage of our construction once in the interim) and thus in \(\mcM^*\). 
Similarly, any type realized in \(\mcM^*\) must be realized in some \(\mcM_i\) and in \(\mcN_i\) as well (as \(\mcM_i \prec \mcN_i\)) and thus in \(\mcN^*\).
Thus, \(\mcM^*\) and  \(\mcN^*\) realize the same types.  
Given a partial elementary map \(f:\mcM^* \to \mcM^*\) with finite domain, we have \(\text{dom}(f) \in \mcM_i\) for some \(i\), and stage two of our construction guarantees that it can be extended. 
Similarly for homogeneity of \(\mcN^*\) by the third stage of our construction. 

At this point it will suffice to show how to construct the elementary extensions mentioned in our three stages. 

\begin{enumerate}
\item As we can enumerate the countably many types realized in any countable \(\mcN\) (there are countably many finite sequences from a countable set and each realizes just one type), it would suffice to show that if \(p(\bar{v})\) is a single type and \((\mcN, \mcM)\) is a Vaughtian pair with \(\mcN\) realizing \(p(\bar{v})\) that there is a \((\mcN, \mcM) \prec (\mcN', \mcM')\) with \(\mcM'\) realizing \(p(\bar{v})\). 
If we can realize a single type in an elementary extension, enumerating the types as \(p_1, p_2, \ldots\) and forming an elementary chain where each \(p_i\) is realized in \(\mcM\) in the \(i\)th model in the chain and taking the union will suffice.

Let \(p(\bar{v})\) be a given type. 
We can see that \(\Gamma = \eldiag((\mcN, \mcM)) \cup \{\phi_U(\bar{v}, \bar{a}) \mid \phi(\bar{v}, \bar{a}) \in  p(\bar{v})\}\) is satisfiable.  
Specifically, if \(\phi_1(\bar{v}, \bar{a}), \ldots, \phi_k(\bar{v}, \bar{a}) \in p\) are a finite subset of this type, we know that \(\mcN \models \bigwedge\limits_{j \leq k} \E \bar{x}\, \phi_j(\bar{x},\bar{a})\) and therefore that \(\mcM \models \bigwedge\limits_{j \leq k} \E \bar{x} \, (\phi_j)_U(\bar{x},\bar{a})\). 
This finite subset is realized in \((\mcN, \mcM)\) itself. 
Any model of \(\Gamma\) is an elementary extension of \((\mcN, \mcM)\) which we may call \((\mcN', \mcM')\).
We can also take \((\mcN', \mcM')\) to be countable as we did in Theorem \ref{theorem_countable_vaughtian_pairs}. 
As \((\mcN', \mcM') \models \Gamma\) we also have that \(p(\bar{v})\) is realized in \((\mcN', \mcM')\). 

\item The construction for this stage is very similar to the one covered just above for the first stage.
Again, it will suffice to show that just one such type can be realized (namely the type of \(c\) over \(\bar{a}\)) as we can add realizations for our countably many types one at a time. 
Here, we take \(\Gamma = \eldiag((\mcN, \mcM)) \cup \{\phi(\bar{b}, d) \mid \mcM \models \phi(\bar{a}, c)\}\). 
If some finite subset, \(\Phi\), were not realizable, we would have \(\mcM \models \exists x \bigwedge\limits_{\phi \in \Phi}\phi(\bar{a}, c)\) but this formula wouldn't hold for \(\bar{b}\), giving a contradiction to \(\bar{a}, \bar{b}\) having the same type. 

\item This case is very similar to the previous one. Here we let \(\Gamma = \eldiag((\mcN, \mcM)) \cup \{\phi(\bar{b}, d) \mid \mcN \models \phi(\bar{a}, c)\}\) where \(d\) is a new constant symbol.
If \(\Gamma\) is not finitely satisfiable that must mean that for some \(\phi\) we have \(\mcN \models \phi(\bar{a}, c) \land \neg \E x \, \phi(\bar{b}, x)\), 
so \(\bar{a}, \bar{b}\) cannot have the same type, a contradiction.
\end{enumerate}
This completes the proof. 
\end{proof}

\begin{lemma}\label{lemma_extend_isomorphically}
Let \((\mcN, \mcM)\) be a Vaughtian pair of countable models, \(\mcN'\) a countable model and \(f:\mcN' \to \mcM\) is an isomorphism. 
We claim that there must exist an \(\mcN''\) such that \((\mcN'', \mcN') \isom (\mcN, \mcM)\) and that, moreover, we can take \(N' \subseteq N''\). 
\end{lemma}

\begin{proof}
Let \(N'' = N' \sqcup (N \setminus M)\) (note that this ensures \(N' \subseteq N''\)).
Define \(f'': N'' \to N\) as follows: if \(x \in N'\) then \(f'(x) = f(x)\). 
Otherwise, \(f'(x) = x\). 
By construction, \(f'\) is a bijection between \(N''\) and \(N\) which, when restricted to \(N'\) is a bijection between  \(N'\) and \(M\). 
For atomic \(\mcL^U\)-formulae \(\phi\), let \((\mcN'', \mcN') \models \phi(\bar{v}) \iff (\mcN, \mcM) \models \phi(f'(\bar{v}))\).
This gives us interpretations of \(\mcL\) for \(\mcN''\) such that \(f'\) is an \(\mcL\)-isomorphism by construction. 
\end{proof}

\begin{theorem}\label{theorem_aleph_one_vaughtian_pair}
Given a Vaughtian pair of countable isomorphic models, \((\mcN, \mcM) \models T\), we can construct a model \(\mcN^* \models T\) which is of cardinality \(\aleph_1\) and has a definable subset which is countable.  
\end{theorem}

\begin{proof}
We will construct \(\mcN^*\) as the union of an elementary chain of models \(\mcN_\alpha\) indexed by \(\omega_1\).

Let \(\mcN_0 = \mcN\). 
We'd like to maintain \(\mcN_\alpha \isom \mcN\) and \(\phi(\mcN_\alpha) = \phi(\mcN)\) and \(\mcN_\alpha\) is countable. 
This is true by hypothesis for \(\mcN\).

If \(\alpha\) is not a limit ordinal, let \(\mcN_{\alpha}\) be a countable model of \(T\) such that \((\mcN_{\alpha}, \mcN_{\alpha-1}) \isom (\mcN, \mcM)\) and \(N_{\alpha-1} \subseteq N_\alpha\) which exists by Lemma \ref{lemma_extend_isomorphically}. 
As \((\mcN_{\alpha}, \mcN_{\alpha-1})\) is a Vaughtian pair, we've ensured that \(\mcN_{\alpha-1} \prec \mcN_{\alpha}\) and \(\mcN_{\alpha-1} \neq \mcN_{\alpha}\).
We have \(\phi(\mcN_{\alpha}) = \phi(\mcN_{\alpha-1})\) and by hypothesis \(\phi(\mcN_{\alpha-1}) = \phi(\mcM)\). 
Similarly,  \(\mcN_{\alpha} \isom \mcN_{\alpha-1} \isom \mcN\). 

To define \(\mcN_\alpha\) where \(\alpha\) is a limit ordinal, let \(\mcN_\alpha = \bigcup\limits_{\beta < \alpha}\mcN_\beta\). 
As \(\mcN_\alpha\) is the union of countably many countable sets, it is countable. 
If \(\bar{a} \in \mcN_\alpha\) then \(\bar{a} \in \mcN_\beta\) for some \(\beta < \alpha\). 
As \(\mcN_\beta \prec \mcN_\alpha\), \(\text{tp}^{\mcN_\beta}(\bar{a}) = \text{tp}^{\mcN_\alpha}(\bar{a})\). 
As \(\mcN_\beta \isom \mcM\), this type is realized in \(\mcN\).
Thus, every from \(S_n(T)\) realized in \(\mcN_\alpha\) is realized in \(\mcN\). 
The converse follows from \(\mcN \prec \mcN_\alpha\). 
Thus \(\mcN \isom \mcN_\alpha\).

If \(\mcN^* = \bigcup\limits_{\alpha < \omega_1}\mcN_\alpha\) we know that \(\phi(\mcN^*) = \phi(\mcN) \subseteq \mcM\) which is countable.
As \(\mcN \prec \mcN^*\) we have \(\mcN^* \models T\).  
As we add at most countably many but at least one element at each of the \(\aleph_1\) stages, we have \(|\mcN^*| = \aleph_1\).
\end{proof}

\subsection{Vaughtian Pairs and Categoricity}

\begin{theorem}\label{theorem_uncountable_vaughtian_pairs}
Given \(\mcM \models T\) where \(|M| = \aleph_1\), \(T\) is \(\omega\)-stable and \(\phi(\mcM)\) is countably infinite, for every uncountable cardinal \(\kappa\), there is a model of \(T\) of cardinality \(\kappa\) with a countably infinite definable subset.  
\end{theorem}

\begin{proof}
As in Theorem \ref{theorem_aleph_one_vaughtian_pair}, it suffices to show that if \(\mcM\) is uncountable and has a countably infinite definable subset defined by \(\phi\), that there is an \(\mcM \prec \mcN\) which is a strict elementary extension such that \(\phi(\mcN) = \phi(\mcM)\) and \(|\mcM| = |\mcN|\).

First, we show there is an \(\mcL_\mcM\)-formula, \(\psi\) such that there are uncountably many realizations of \(\psi(v)\) in \(\mcM\) and for all \(\mcL_\mcM\)-formulae, \(\theta\), either \(\psi \land \theta(v)\) or \(\psi \land \neg\theta(v)\) has only countably many realizations. 
Such a formula is similar to a minimal formula. 
The proof of its existence in a model of an \(\omega\)-stable theory will be similar to the proof of Theorem \ref{lemma_minimal_omst}.

For contradiction, we build an infinite binary tree of formulae such that all paths in the tree are types over a countably set, and all are satisfiable, contradicting \(\omega\)-stability.
The root of the tree is \(v = v\), which has uncountably many realizations. 
As this formula is not our desired \(\psi\), there is a \(\theta\) such that \((v = v) \land \theta(v)\) and \((v = v) \land \neg \theta(v)\) have uncountably many realizations. 
Every path in this tree is a finitely satisfiable type over the countable set of constants from \(\mcM\) appearing in the tree.
This contradicts \(\omega\)-stability. 

Given \(\psi\) as above, we now construct a type over \(M\), \(p\), the type of an element not in any of the countable definable subsets of \(\psi(\mcM)\). 
That is \(p = \{\theta(v) \mid |(\theta \land \psi (\mcM))| \geq \aleph_1\}\).
This type is finitely satisfiable. If we take \(\theta_1, \ldots, \theta_n\) from \(p\) then each fails in on at most countably many elements. 
Their conjunction also fails to hold of at most countably many elements. 
\(p\) is actually a complete type as a formula is in this type  exactly when it is true of uncountably many elements of \(\mcM\) and it's negation is in this type otherwise. 
Note that \(\mcM\) cannot contain a realization of \(p\) (for each element \(m \in M\), \(p\) contains \(v \neq m\)). 
We realize \(p\) in \(\mcM'\) such that \(\mcM \prec \mcM'\). 
Let \(c\) be the realization of \(p\). 
By Theorem \ref{thm_omst_prime} we can find a \(\mcM_0 \prec \mcM'\) prime over \(M \cup \{c\}\). 
Additionally, \(\mcM \prec \mcM_0\). 
This \(\mcM_0\) will be our desired model. 

All that remains is to show that \(\mcM_0\) has the desired property, that \(\phi(\mcM) = \phi(\mcM_0)\).   
\(\mcM_0\) will actually have the more general property that if \(\Delta = \{\delta_i(v)\mid i \in \N\}\) is a countable (incomplete) type which is realized in \(\mcM_0\), then \(\Delta\) is realized in \(\mcM\).
If we take \(\Delta(v) = \{v \neq a \mid a \in \phi(\mcM)\} \cup \{\phi(v)\}\) then we see that \(\Delta\) is not realized in \(\mcM\) and therefore not realized in \(\mcM_0\), giving the desired property. 

Say \(\Delta(v)\) is realized in \(\mcM_0\).
The type of \(v\) over \(M \cup \{c\}\) must be isolated by some formula \(\theta(v, c)\). %cite
Note that \(p(c) \ni \E x \, \theta(x, c)\) as this formula is true of \(c\) in \(\mcM_0\) (as it is witnessed by \(v\)).
Additionally, \(\A v \, (\psi(v, c) \to \delta_i(v)) \in p(c)\) for all \(i\). (As \(\theta\) isolates the type of \(v\)). 
Let \(\Gamma(c) \subseteq p(c)\) be these formulae just mentioned, that is \(\Gamma(c) = \{\A v \,(\theta(v, c) \to \delta_i(v)) \in p(c) \mid i \in \N\} \cup \{\E x \, \theta(x, c)\}\). 
\(\Gamma\) is realized by \(c\) in \(\mcM_0\). 

At this point, it suffices to find a realization of \(\Gamma\) in \(\mcM\).
If \(c'\) realizes \(\Gamma\) in \(\mcM\) then \(\E x \, \theta(x, c')\), call this element \(v'\). 
As \(c'\) realizes \(\Gamma\), \(\mcM \models \A v' \,(\theta(v', c') \to \delta_i(v'))\) giving us that \(v'\) realized \(\Delta\) in \(\mcM\). 
Note that there are countably many formulae in \(\Gamma\), each of which will fail on only countably many elements of \(\psi(\mcM)\) by construction of \(p\). 
There must therefore be only countably many elements of \(M\) for which \(\Gamma\) doesn't hold. 
Therefore, \(\Gamma\) mist be realized in \(\mcM\) and \(\Delta\) must be as well. 
\end{proof}

\begin{theorem}\label{theorem_vaughtian_pairs_categoricity}
Any theory, \(T\), which has Vaughtian pairs cannot be \(\kappa\)-categorical for any uncountable \(\kappa\).
\end{theorem}

\begin{proof}
If \(T\) is not \omst it cannot be categorical for any uncountable \(\kappa\) by Theorem \ref{theorem_omega_stability_categoricity}.
Otherwise, as \(T\) has a Vaughtian pair, by Theorem \ref{theorem_countable_vaughtian_pairs} it has a Vaughtian pair of countable models. 
By Theorem \ref{theorem_countable_isomorphic_vaughtian_pair} it has a Vaughtian pair of countable isomorphic models.
By Theorem \ref{theorem_aleph_one_vaughtian_pair} it has a model of size \(\aleph_1\) with a countable definable subset. 
By Theorem \ref{theorem_uncountable_vaughtian_pairs} it has a model of size \(\kappa\) which has a countable subset.  
Theorem \ref{theorem_vaughtian_pair_categoricity} tells us that \(T\) cannot be \(\kappa\)-categorical. 
\end{proof}
