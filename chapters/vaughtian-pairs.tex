\section{Definition}
Like \(\omega\)-instability, lacking Vaughtian pairs is a property of theories which prevents uncountable categoricity (we will prove this in Theorem \ref{theorem_vaughtian_pairs_categoricity}).
The content of Morley's Categoricity theorem (Theorem \ref{theorem_morleys_categoricity}) will be that these are the only two obstacles to uncountable categoricity. 

\begin{definition}\label{definition_definable_subset}
Let \(\mcM\) be an \(\mcL\)-structure and \(\phi\) and \(\mcL\) formula in \(n\) free variables. 
We define \(\phi(\mcM) = \{\bar{m} \in M^n \mid \mcM \models \phi(\bar{m})\}\).
If \(S \subseteq M^n = \phi(\mcM)\) for some \(\phi\) we say that \(S\) is a definable subset of \(\mcM\) and that it is defined by \(\phi\).
\end{definition}

\begin{definition}\label{definition_vaughtian_pairs}
Two distinct \(\mcL\)-structures \((\mcN, \mcM)\) form a Vaughtian pair if \(\mcM \prec \mcN\) and for some \(\mcL_\mcM\)-formula \(\phi\), \(\phi(\mcM) = \phi(\mcN)\) is infinite. 
\end{definition}

Even though we write \(\mcM \prec \mcN\), we write the Vaughtian pair in the other order, that is as \((\mcN, \mcM)\), because we can think of this pair of models as a single model (\(\mcN\)) with a distinguished subset corresponding to \(\mcM\).
This will often be useful in proving theorems and lemmas about Vaughtian pairs, so we will flesh out the construction here. 

If \(\mcM, \mcN\) are \(\mcL\)-structures, let  \(\mcL^U = \mcL \cup \{U\}\). 
We look at (\(\mcN, \mcM\)) as an \(\mcL^U\)-structure which shares its underlying set and interpretations of \(\mcL\) with \(\mcN\) but where \(U\) is the subset \(M\). 
Note that, if \(\phi\) is an \(\mcL\)-formula \((\mcN, \mcM) \models \phi \iff \mcN \models \phi\).
Adding the unary predicate, \(U\), to our language allows us to find an \(\mcL^U\)-formula \(\phi_U\) corresponding to any \(\mcL\)-formula \(\phi\) such that \((\mcN, \mcM) \models \phi \iff \mcM \models \phi_U\).

We define \(\phi_U\) as follows:
\begin{enumerate}
\item If \(\phi(\bar{v})\) is atomic, we let \(\phi_U(\bar{v}) = \bigwedge_{v \in \bar{v}}U(v) \land \phi(\bar{v})\)
\item If \(\phi(\bar{v}) = \neg \psi(\bar{v})\) then \(\phi_U(\bar{v}) = \neg \psi_U(\bar{v})\)
\item If \(\phi(\bar{v}) = (\psi \land \theta)(\bar{v})\) then \(\phi_U(\bar{v}) = \psi_U(\bar{v}) \land \theta_U(\bar{v})\)
\item If \(\phi(\bar{v}) = \E x \psi(x, \bar{v})\) then \(\phi_U(\bar{v}) = \E x (U(x) \land \psi_U(x, \bar{v}))\)
\end{enumerate}
A simple induction shows that \((\mcN, \mcM) \models \phi \iff \mcM \models \phi_U\).

Before we proceed, it will be useful to develop some intuition about why Vaughtian pairs prevent uncountable categoricity for theories in countable languages, \(T\).
\begin{theorem}\label{theorem_vaughtian_pair_categoricity}
If \(\mcM \models T\), \(|\mcM| = \kappa > \aleph_0\) and \(\phi(\mcM)\) is countably infinite then \(T\) is not \(\kappa\)-categorical and \(T\) has a Vaughtian pair. 
\end{theorem}

\begin{proof}
We first show that then there is a Vaughtian pair for \(T\).
It suffices to let \(\mcM^* \models T^*\) be as in Theorem \ref{theorem_skolem_function_extension}.
We see that \((\mathcal{H}(\phi(\mcM)), \mcM)\) form a Vaughtian pair. 
Additionally, \(T\) cannot be \(\kappa\) categorical as we can construct an \(\mcN \models T\) with \(|\mcN| = \kappa\) and \(|\phi(\mcN)| = \kappa\) (an isomorphism \(f: \mcM \to \mcN\) would be bijection when restricted to \(\phi(\mcM)\) and by definition there is no bijection from a countable set to any uncountable \(\kappa\)).
We simple add constant symbols \(\{\bar{c}_i \mid i \in \kappa\}\) to our language. 
We can easily see that \(\Gamma = \text{Th}(\mcM) \cup \{\bar{c}_i \neq \bar{c}_j \mid i \neq j\} \cup \{\phi(\bar{c}_i) \mid i \in \kappa\}\) is finitely satisfiable (it suffices to find \(n\) distinct realizations of \(\phi\) in \(\mcM\) for all \(n \in \N\) which is possible as \(\phi(\mcM)\) is infinite). 
Any \(\mcN \models \Gamma\) will have \(|\phi(\mcN)| = \kappa\). 
\end{proof}

We see the connection between some Vaughtian pairs and lacking \(\kappa\)-categoricity, specifically Vaughtian pairs \((\mcN, \mcM)\) for which \(|\mcN| = \kappa\) and \(|\mcM| = \aleph_0\).
It would suffice to show that whenever there is any Vaughtian pair for \(T\), we can construct a model of \(T\) of any cardinality \(\kappa\) which has a countable definable subset. 
The rest of this chapter will show this construction. 

\begin{theorem}\label{thm_countable_vaughtian_pairs}
If \(\mcL\) is countable and an \(\mcL\)-theory \(T\) has a Vaughtian pair \(\mcN, \mcM\), it has a Vaughtian pair \((\mcN_0, \mcM_0)\) of countable models. 
\end{theorem}

\begin{proof}\label{proof_countable_vaughtian_pairs}
It will suffice to show that there is an \(\mcL^U\)-theory, \(T'\), for which \((\mcA, \mcB) \models T'\) iff \((\mcA, \mcB)\) forms a Vaughtian pair for \(T\). 
As \((\mcN, \mcM)\) forms a Vaughtian pair, \(T'\) is satisfiable.
As \(\mcL^U\) is countable, it has a countable model, \((\mcN_0, \mcM_0)\), by the downward L\"owenheim-Skolem theorem.
This is the desired countable Vaughtian pair. 

Our goal is to demonstrate an \(\mcL_\mcM^U\)-theory, \(\Gamma\), asserting that \((\mcN, \mcM)\) is Vaughtian pair.

\(\Gamma\) contains the following sets of formulae:
\begin{itemize}
\item \(T\) and \(\{\phi_U | \phi \in T\}\)
\item \(\A \bar{v} (U(\bar{v}) \land \E x \phi(x, \bar{v})) \to \E x U(x) \land \phi(x, \bar{v})\) for all \(\phi\). (By Theorem \ref{theorem_tarski_vaught_test} this shows \(\mcM \prec \mcN\))) 
\item If \(\psi(\bar{v})\) is the \(\mcL_\mcM\)-formula which isolates an infinite subset of \(\mcM\), for all \(n \in \N\), \(\Gamma\) includes \(\E \bar{v}_1, \E \bar{v}_2, \ldots \E \bar{v}_n \bigwedge \bar{v}_i \neq \bar{v}_j \land \bigwedge \phi(\bar{v}_i)\). That is \(\psi\) defines an infinite subset of \(\mcN\).
\item \(\A \bar{v} (\psi(\bar{v}) \to U(\bar{v}))\). That is, \(\psi(\mcN) \subset \psi(\mcM)\). As the other direction is trivial, this ensures \(\psi(\mcN) = \psi(\mcM)\). 
\item Finally, we assert that \(\mcN\) is a proper elementary extension of \(\mcM\) with \(\E x \neg U(x)\). 
\end{itemize}

By construction, \(\Gamma\) is an \(\mcL^U\)-theory, \(T'\), for which \((\mcA, \mcB) \models T'\) iff \((\mcA, \mcB)\) forms a Vaughtian pair for \(T\). 
The L\"owenheim-Skolem theorem gives us our desired model. 
\end{proof}

For the next few claims we will need to use a property called homogeneity which will allow us to show that certain countable models are isomorphic.

\begin{definition}\label{definition_partial_elementary}
Let \(\mcM\) be an \(\mcL\) function and \(A \subseteq M\).
A function \(f:A \to M\) for which \(\mcM \models \phi(\bar{v}) \iff \mcM \models \phi(f(\bar{v}))\) is called ``partial elementary''.
\end{definition}

\begin{definition}\label{definition_homogeneity}
A countable model \(\mcM\) is homogeneous if all finite \(A \subseteq M\), partial elementary functions \(f: A \to M\) and \(m \in M\), there is a partial elementary \(f': A \cup \{m\} \to M\) extending \(f\) which is partial elementary. 
% Examples?
\end{definition}

\begin{theorem}\label{theorem_partial_elementary_automorphism}
Any partial elementary map \(f: A \to M\) (where \(A \subseteq M\) is finite) can be extended to an automorphism. 
\end{theorem}

\begin{proof}
We define an isomorphism extending \(f\) inductively. 
Let \(f_0 = f\) be the first function in our chain.
We will define \(f_i\) for \(i > 0\) such that \(f_{i-1} \subseteq f_i\).
We will alternatively extend the domain and range of \(f_i\) by elements in \(M = \{m_1, m_2, \ldots\}\).
That is, if \(i\) is odd, \(m_i \in \text{dom}(f_{\frac{i-1}{2}})\) and if \(i\) is even \(m_i \in \text{rng}(f_{\frac{i}{2}})\).
The former is possible as homogeneity allows us to expand the domain of \(f_{i-1}\) to include \(m_i\).
For the case of \(i\) even, note that partial elementary maps have partial elementary inverses (if \(f\) is partial elementary, it is injective). 
That partial elementary functions have partial elementary inverses allows us to extend the range of \(f_{i-1}^{-1}\) to include \(m_i\). 
Taking the inverse of the resulting function gives us \(f_i\).
Let \(f^* = \bigcup_{i \in \N}f_i\).
\(f^*\) has domain and range equal to all of \(M\) and is partial elementary, making \(f^*\) our desired automorphism.
\end{proof}

\begin{lemma}\label{lemma_types_isomorphism}
The next thing we note about countable homogeneous models, \(\mcM\) and \(\mcN\), of the same complete theory, \(T\), is that if \(\mcM\) and \(\mcN\) realize the same types in \(S_n(T)\), they are isomorphic.
\end{lemma}

\begin{proof}\label{proof_types_isomorphism}
To show this, we will build a map in a way very similar to the way we built \(f^*\) above. 
Note that we need \(T\) to be complete in order for \(f_0=\emptyset\) to be partial elementary.
We will enumerate the elements in the universe of both models, and iteratively add elements to the domain and range of a partial elementary map. 
If we can do this, the union of this sequence of maps will be an isomorphism.
As the inverse of a partial elementary map is also partial elementary, it will suffice to show that we can extend a partial elementary map to include an arbitrary element in its range. 
Let \(dom(f) = \{m_1, \ldots, m_i\}, rng(f) = \{n_1, \ldots, n_i\}, f(m_j) = n_j\).
Now, choose \{m\} an arbitrary element of \(M\), our goal is to find \(g \supseteq f\) which is partial elementary and \(m \in dom(g)\).
Note that we can find a partial elementary \(h\) for which \(dom(f) \cup \{m\} =  dom(h)\) by merely finding the realization of \(tp(\bar{m}, m)\) in \(\mcN\), which is guarunteed by \(\mcM, \mcN\) realizing the same types. 
Our one remaining obstacle is that \(h(m_i)\) may not be equal to \(n_i\). 
This can be rectified with homogeneity. We can easily see that the function which takes \(h(m_i) \mapsto n_i\) can be extended to an automorphism. 
By simply composing this automorphism with our \(h\) from above to form \(g\) we have found our desired function. 
\end{proof}

% Possible Clarification: Moving from types over models to types over theories (p.125)

\begin{theorem}\label{theorem_vaughtian_pairs_countable_isomorphic}
A Vaughtian pair of countable models, (\(\mcN_0, \mcM_0\)) can be extended to a Vaughtian pair of isomorphic countable models.
\end{theorem}

\begin{proof}
By above, it suffices to show a vaughtian pair \((\mcN, \mcM)\) can be extended to a vaughtian pair \((\mcN^*, \mcM^*)\) where \(\mcN^*, \mcM^*\) are homogeneous and realize the same types. 

We aim to construct a chain of Vaughtian pairs indexed by the natural numbers such that each pair is an elementary extension of the previous one and their union will be homogeneous and realize the same types. 
The zeroth pair is the countable pair (\(\mcN_0, \mcM_0\)) from above. 
For \(i > 0\) we construct (\(\mcN_i, \mcM_i\)) as follows (the details of constructions in each of these stages will follow):
\begin{enumerate}
\item  If \(i\) is a multiple of 3, then (\(\mcN_i, \mcM_i\)) is an elementary extension of (\(\mcN_{i-1}, \mcM_{i-1}\)) such that for every \(\bar{a} \in M\) every type in \(S_n(\bar{a})\) realized in \(\mcN_{i-1}\) is realized in \(\mcM_i\). %make sure vector notation is good. 
\item  If \(i\) is one more than a mutliple of 3, then (\(\mcN_i, \mcM_i\)) is an elementary extension of (\(\mcN_{i-1}, \mcM_{i-1}\)) such that if \(\bar{a}, \bar{b}\) realize the same type in \(\mcM_{i-1}\) and \(c \in \mcM_{i-1}\), then there is a \(d \in \mcM_i\) such that \(\bar{a}c\) and \(\bar{b}d\) realize the same type in \(\mcM_i\). 
\item If \(i\) is two more than a mutliple of 3, then (\(\mcN_i, \mcM_i\)) is an elementary extension of (\(\mcN_{i-1}, \mcM_{i-1}\)) such that if \(\bar{a}, \bar{b}\) realize the same type in \(\mcN_{i-1}\) and \(c \in \mcN_{i-1}\), then there is a \(d \in \mcN_i\) such that \(\bar{a}c\) and \(\bar{b}d\) realize the same type in \(\mcN_i\).
\end{enumerate}
If we let \((\mcN, \mcM) = \bigcup_{i \in \N}(\mcN_i, \mcM_i)\), we see that every type realized in \(\mcN\) is realized in some \(\mcN_i\) and therefore in \(\mcM_{i+3}\) (as we must have done the first stage of our construction once in the interim) and thus in \(\mcM\). 
Similarly, any type realized in \(\mcM\) must be realized in some \(\mcM_i\) and in \(\mcN_i\) as well (as \(\mcM \prec \mcN\)) and thus in \(\mcN\).
Thus, \(\mcM\) and  \(\mcN\) realize the same types.  
Given a partial elementary map \(f:\mcM \to \mcM\) with finite domain, we have \(\text{dom}(f) \in \mcM_i\) for some \(i\), and stage two of our construction guaruntees that it can be extended. 
Similarly for homogeneity of \(\mcN\) by the third stage of our construction. 

At this point it would suffice to show how to construct the elementary extensions mentioned in our three stages. 
\begin{enumerate}
\item As we can enumerate the countably many types over finite sets realized in any countable \(\mcN\) (there are countably many finite sequences from a countable set and countably many \(\bar{a}\)), it would suffice to show that if \(p(\bar{v})\) is an \(\bar{a}\)-type and \((\mcN, \mcM)\) is a Vaughtian pair with \(\mcN\) realizing \(p(\bar{v})\) that there is a \((\mcN, \mcM) \prec (\mcN', \mcM')\) with \(\mcM'\) realizing \(p(\bar{v})\). 
We can see that \(\Gamma = \eldiag((\mcN, \mcM)) \cup \{\phi_U(\bar{v}, \bar{a}) \mid \phi(\bar{v}, \bar{a}) \in  p(\bar{v})\}\) is satisfiable.  
Specifically, if \(\phi_1(\bar{v}, \bar{a}), \ldots, \phi_k(\bar{v}, \bar{a}) \in p\) are a finite subset of this type, we know that \(\mcN \models \bigwedge \E \bar{x} \phi_k(\bar{x},\bar{a})\) and therefore that \(\mcM \models \bigwedge \E \bar{x} (\phi_k)_U(\bar{x},\bar{a})\). 
This finite subset if realized in \((\mcN, \mcM)\) itself. 
Any model of \(\Gamma\) is an elementary extension of \((\mcN, \mcM)\) which we may call \((\mcN', \mcM')\).
We can also take \((\mcN', \mcM')\) to be countable. 
As \((\mcN', \mcM') \models \Gamma\) we also have that \(p(\bar{v})\) is realized in \((\mcN', \mcM')\). 
\item The construction for this stage is very similar to the one covered just above for the first stage.
Again, it will suffice to show that just one such type can be realized (namely the type of \(c\) over \(\bar{a}\)) as we can add realizations for our countably many types one at a time. 
Here, we take \(\Gamma = \eldiag((\mcN, \mcM)) \cup \{\phi(\bar{b}, d) \mid \phi(\bar{a}, c)\}\). 
If some finite subset, \(\Phi\), were not realizable, we would have \(\mcM \models \exists x \Phi(\bar{a})\) but this formuale wouldn't hold for \(\bar{b}\), giving a contradiction. 
\item The argument above can applies again, except we no longer take \(\phi\) to be a formula in \(\mcM\). % TODO: clarify this. 
\end{enumerate}

\end{proof}

% claim?
\begin{theorem}\label{theorem_aleph_one_vaightian_pairs}
Given a Vaughtian pair of countable isomorphic models, \((\mcN_1, \mcN_0) \models T\), we can construct a model \(\mcN^*\models T\) which is of cardinality \(\aleph_1\) and has a definable subset which is countable.  
\end{theorem}

\begin{proof}
% We have \(\mcN_0 \prec \mcN_1\) so \(\iota: \mcN_0 \to \mcN_1\) is the map including \(\mcN_0\) into \(\mcN_1\). 
% Additionally, we have an isomorphism \(f: \mcN_0 \to \mcN_1\). 

% TODO 2: Proof of 4.3.34
\textcolor{red}{Insert Proof Here \ldots}
\end{proof}

\begin{theorem}\label{theorem_uncountable_vaightian_pairs}
Given a Vaughtian pair \((\mcN, \mcM)\), \(\mcN, \mcM \models T\) where \(|N| = \aleph_1\), \(\mcM\) is countable and \(T\) is \(\omega\)-stable, for every uncountable cardinal \(\kappa\), there is a model of \(T\) of cardinality \(\kappa\) with a countable definable subset.  
\end{theorem}

\begin{proof}
% TODO 2b: Proof of 4.3.41 
Let \(\phi\) be the formula defining a countable subset. 
It suffices to construct an elementary chain of models indexed by ordinals less than \(\kappa\) such that \(\mcM_\alpha \prec \mcM_{\alpha+1}\) and \(\mcM_\alpha \neq \mcM_{\alpha+1}\) but \(\phi(\mcM_\alpha) = \phi(\mcM_{\alpha+1}\). 
The model given by \(\bigcup_{\alpha<\kappa} \mcM_\alpha\) will be of size \(\kappa\) (as there are \(\kappa\) many stages, each adding fewer than \(\kappa\) many elements but at least one).
By construction, \(\phi\) will still define a countable subset. 

\textcolor{red}{Insert Proof Here \ldots} 
\end{proof}

\begin{theorem}\label{theorem_vaughtian_pairs_categoricity}
Any theory which has Vaughtian pairs cannot be \(\kappa\)-categorical for any uncountable \(\kappa\).
\end{theorem}

\begin{proof}
As we stated above, we can also create a model of \(T\) of size \(\kappa\) where all infinite definable sets are of size \(\kappa\). 
Such a model wouldn't be isomorphic to the model we just created which has a countable definable subset. 
\end{proof}
