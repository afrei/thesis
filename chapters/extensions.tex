We begin by restating Morley's categoricity theorem in a way which will allow us to generalize or extend it naturally.

\begin{definition}\label{definition_number_models}
If \(T\) be a complete first order theory \(I(T, \kappa)\) is the number of models of \(T\) of size \(\kappa\) up to isomorphism.
\end{definition}

Morley's categoricity theorem stated that if \(T\) is a theory in a countable language and \(\kappa, \lambda > \aleph_0\) then \(I(T, \kappa) = 1 \iff I(T, \lambda) = 1\). 

We'd like to ask some more general questions, some of which we will be able to answer fully in this section and some of which will only adress in a limited sense. 
In the following, we assume that \(T\) is a theory in a countable language. 

\begin{question}\label{question_countable_models_uncountably_categorical}
If \(\lambda > \aleph_0, I(T, \lambda) = 1\) what can we conclude about \(I(T, \aleph_0)\)? (could it be \(2^{\aleph_0}\)?)
\end{question}

\begin{question}\label{question_finite_spectra}
For which natural numbers, \(n\), are there theories for which  \(I(T, \aleph_0) = n\)?
\end{question}

% \item More generally, what possible values can \(I(T, \aleph_0)\) take on?

\begin{question}\label{question_morleys_conjecture}
Is it possible to have \(\kappa > \lambda > \aleph_0\) but \(I(T, \kappa) < I(T, \lambda)\)?
\end{question}

\begin{question}\label{question_los_conjecture_uncountable_languages}
Is there a natural analogue to Morley's Categoricity Theorem when we no longer assume that \(T\) is a theory in a countable language?
\end{question}

% \item What if we no longer assume that \(T\) is a first order theory?

An easy extension of our argument for Morley's Categoricity Theorem shows us that for uncountably categorical \(T\), \(I(T, \aleph_0)\leq \aleph_0\).
Up to isomorphism, these models are determined by their dimension, which must not be greater than \(\aleph_0\) leaving only countably many possibilities. 
This is an answer to question \ref{question_countable_models_uncountably_categorical}.

With regards to question \ref{question_finite_spectra}, we will exhibit theories which have \(n\) models up to isomorphism for all \(n \in \N\) with the exception of \(2\).
Moreover, we can show that a theory with only two countable models up to isomorphism must not exist. 

% TODO 4?
% You have to actually do this

We can answer question \ref{question_morleys_conjecture} in the negative. This is known as Morley's Conjecture. A full proof is beyond the scope of this paper, but we will present some of the ideas from the proof. 

To adress Question \ref{question_los_conjecture_uncountable_languages}, \L o\'s' Conjecture for uncountable languages (of cardinality \(\kappa\)) states that if \(\lambda, \lambda' >\kappa\) then \(I(T, \lambda) = 1 \iff I(T, \lambda') = 1\). 
A complete proof of this theorem is again beyond the scope of this paper, but we reference {\color{red}someplace or other.} %insert some reference here 