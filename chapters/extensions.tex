We begin by restating Morley's categoricity theorem in a way which will allow us to generalize or extend it naturally.

\begin{definition}\label{definition_number_models}
If \(T\) be a complete first order theory \(I(T, \kappa)\) is the number of models of \(T\) of size \(\kappa\) up to isomorphism.
\end{definition}

Morley's categoricity theorem stated that if \(T\) is a theory in a countable language and \(\kappa, \lambda > \aleph_0\) then \(I(T, \kappa) = 1 \iff I(T, \lambda) = 1\). 

We'd like to ask some more general questions, some of which we will be able to answer fully in this section and some of which will only adress in a limited sense. 
In the following, we assume that \(T\) is a theory in a countable language. 

\begin{question}\label{question_countable_models_uncountably_categorical}
If \(\lambda > \aleph_0, I(T, \lambda) = 1\) what could \(I(T, \aleph_0)\) be? (could it be \(2^{\aleph_0}\)?)
\end{question}

%\begin{question}\label{question_finite_spectra}
%For which natural numbers are there theories for which  \(I(T, \aleph_0) = n\)?
%\end{question}

\begin{question}\label{question_los_conjecture_uncountable_languages}
Is there a natural analogue to Morley's Categoricity Theorem when we no longer assume that \(T\) is a theory in a countable language?
\end{question}

\begin{question}\label{question_morleys_conjecture}
Is it possible to have \(\kappa > \lambda > \aleph_0\) but \(I(T, \kappa) < I(T, \lambda)\)?
\end{question}

An easy extension of our argument for Morley's Categoricity Theorem shows us that for uncountably categorical \(T\), \(I(T, \aleph_0)\leq \aleph_0\).
Up to isomorphism, these models are determined by their dimension, which must not be greater than \(\aleph_0\) leaving only countably many possibilities. 
This is an answer to question \ref{question_countable_models_uncountably_categorical}.

% Possible addition: With regards to question \ref{question_finite_spectra}...

To adress Question \ref{question_los_conjecture_uncountable_languages}, \L o\'s' Conjecture for uncountable languages (of cardinality \(\kappa\)) states that if \(\lambda, \lambda' >\kappa\) then \(I(T, \lambda) = 1 \iff I(T, \lambda') = 1\). 
A complete proof of this theorem is again beyond the scope of this paper, but we reference \cite{shelahUncountable}.

We can answer question \ref{question_morleys_conjecture} in the negative. This is known as Morley's Conjecture. A full proof is beyond the scope of this paper, but we will present some of the ideas from the proof. 

The proof relies on dealing with theories of a few different kinds.\cite{hart}
For a few of these kinds of theories, we will be able to compute the spectrum explicitly.

To categorize theories we will need to define a few properties of theories. 

\begin{definition}\label{definition_superstable}
A theory \(T\) is superstable if it is \(\kappa\)-stable for all sufficiently large \(\kappa\). Otherwise, we say \(T\) is unsuperstable. 
\end{definition}

%\begin{definition}\label{definition_ndop}
%A theory \(T\) is said to have \(NDOP\) if 
%\end{definition}
%
%\begin{definition}\label{definition_otop}
%
%\end{definition}
%
%\begin{definition}\label{definition_deep}
%
%\end{definition}

\begin{definition}\label{definition_many_models}
A theory \(T\) is said to have many models if \(I(T, \lambda) = 2^\lambda\) for all uncountable \(\lambda\). 
\end{definition}

Any theory which has many models has non-decreasing spectrum on uncountable cardinals as Morley's Conjecture states. 

Theories which are unsuperstable, deep, have DOP or have OTOP they have many models. 
Our fifth case, which is theories which are superstable, shallow and lacking both DOP and OTOP, also have non-decresing spectra. 
Any theory which is uncountably categorical, as those do not have many models, is in this final category. 





