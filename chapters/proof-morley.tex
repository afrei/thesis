At this point we've made good progress towards the proof of Morley's Categoricity theorem.
It suffices to show that uncountable categoricity (for any uncountable cardinal) is equivalent to \(\omega\)-stability and lacking Vaughtian pairs.
As this property doesn't depend on the uncountable cardinality, this suffices. 

We've already shown that \(\omega\)-stability and lacking Vaughtian pairs are necessary for uncountable categoricity.
To show these conditions are also sufficient we will define a notion of ``dimension'' for elementary extensions.
\(\omega\)-stability will tell us that we can view any two models as elementary extensions of the same \textit{prime model}.
The fact that, because of their cardinalities, any equinumerous models will have the same dimension over this prime model.
Having the same dimension over this prime model will suffice to show that these two models are isomorphic.
This will complete the proof. 

\subsection{Prime Models}

\textbf{Definition.} We say that a theory, \(T\),  has a prime model, \(\mcM \models T\) if, for every \(\mcN \models T\), we have \(\mcM \prec \mcN\). 
Clearly, prime models are unique up to isomorphism. Easy examples of prime models are \(\Q\) for fields of characteristic 0 and \(\N\) for \(\Th(\N, 0, s)\). % Examples?

\textbf{Theorem.} \omst theories have prime models.
 
% Proof: 4.2.20 p 136 (a transfinite induction)
{\color{red}Prove this \(\ldots\)}
% Strongly minimal formula with params from prime model

\subsection{Minimality}

\textbf{Definition.} We say that a \(\mcL_\mcM\)-formula \(\phi\) is \textit{minimal} if \(\phi(\mcM) = \{\bar{m} \in M \mid \mcM \models \phi(\bar{m})\) is infinite and has no infinite and co-infinite definable subsets. 
We may also call \(\phi(\mcM)\) minimal. 
We will call \(\phi\) and the set \(\phi(\mcM) \subset M\) % -- This could use some work 
\textit{strongly minimal} if they are minimal in any elementary extension of \(\mcM\).
Moreover, a thepry is strongly minimal if the set \(M\) (defined by the formula \(v = v\)) is strongly minimal. 

Now, assume we have a theory which is \omst and has no Vaughtian-pairs. 
As it is \omst it has a prime model.
We will show that it's being \omst also tells us that there is a minimal formula in its prime model.
We will also show this formula is strongly minimal since the theory has no vaughtian pairs. 

\textbf{Lemma.} If \(T\) is \omst there is a minimal formula in all \(\mcM \models T\).

We will build an infinite binary tree of formulae such that 
every node corresponds to an infinite definable subset of our model (and the formula defining it), 
and such that every node is a subset (or consequence) of its parent. 
We will see that every (infinite) path in this tree is a distinct type over a countable set and that there are \(2^{\aleph_0}\) such types. 
This contradicts the \(\omega\)-stability of \(T\).

It is easy to construct this tree. Let the root of the tree be the entire universe, \(M\), of our model which is defined by the formula \(v = v\).
Given a node in our tree defined by a formula \(\phi\) which defines an infinite subset, there must be a formula \(\psi\) such that both \(\phi \land \psi\) and \(\phi \land \neg \psi\) define infinite subsets, as \(\phi\) is not minimal. 
These two formulae/definable sets will be the children of \(\phi\) in our tree. 
As no formula is minimal, we can repeat this ad infinitum. 

Consider a path in this tree. We can consider the corresponding countable set of formulae consisting of the nodes on this path. 
Every finite subset contains a node of maximum depth, which defines an infinite subset of \(M\) and is therefore satisfiable. 
Our set of formulae corresponding to this path can therefore be completed to a type.
No two distinct paths can be completed to the same type. 
Specifically, if two paths diverge after a node \(\phi\), then they will disagree on the formula \(\psi\) as defined above. 

Moreover, there are only countably many nodes in this tree. 
Each formula in this tree contains finitely many constants from \(M\).
Therefore, all of these types can be defined over countably many constants from \(M\). 
We have presented \(2^{\aleph_0}\) types over a finite subset, contradicting the \(\omega\)-stability of \(T\).

\textbf{Lemma.} If \(T\) has no Vaughtian pairs and \(\phi\) is a minimal formula for \(\mcM \models T\) then \(\phi\) is strongly-minimal. 

{\color{red}\(\ldots\) PROOF????} % Do This one

With these past few lemmata we can conclude that any theory which lack Vaughtian pairs and is \omst has a strongly minimal formula with parameters from a countable prime model. 
We will develop the theory of dimensions of elementary extensions and see that any two extensions of cardinality \(\kappa > \aleph_0\) will have dimension \(\kappa\) over the prime model.
From this fact, we will be able to show they are isomorphic, providing us, finally, with \(\kappa\) categoricity and completing the proof of Morley's categoricity theorem. 

\subsection{Algebraic Closure and Dimension}

\textbf{Definition.} Given a model \(\mcM\) and an \(A \subset M\), we define the algebraic closure of \(A\), denoted \(\acl(A)\) to be all elements of \(M\) which are in a finite set definable by an \(\mcL_A\)-formula. 
(Note that by quantifier elimination over algebraically closed fields, this notion of ``algebraic closure'' agrees with the familiar one.)

The algebraic closure has the nice properties that \(\acl(\acl(A)) = \acl(A)\), that \(A \subset B \implies \acl(A) \subset \acl(B)\) and that if \(a \in \acl(A)\) then there is a finite subset \(A_0 \subset A\) such that \(a \in \acl(A_0)\).
This almost gives us that algebraic closure is a pregeometry. 
The only othe fact we would have to know is exchange, namely that if \(a, b \notin \acl(A), a \in \acl(A \cup \{b\})\) then \(b \in \acl(A \cup \{a\})\) 
This is only true when we take \(A\) to be a subset of some strongly minimal set.

\textbf{Theorem.} Let \(S\) be a strongly minimal set. If \(A \subset S, a, b \in S\) and \(a, b \notin \acl(A)\) but \(a \in \acl(A \cup \{b\})\) then \(b \in \acl(A \cup \{a\})\) 

% Clarify: Limiting \acl to a set D


% I think there might be a better, more intuititve presentation for this proof. (something about pairs for which \phi(x, y))
As \(a \in \acl(A \cup \{b\})\), we have \(\mcM \models \phi(a, b)\) for some \(\mcL_A\)-formula % Make sure we use this notation
\(\phi\) where \(\{x \in S \mid \phi(x, b)\}\) contains \(n\) elements. 
Consider the set defined by \(\psi(y)\) asserting \(\{x \in S \mid \phi(x, y)\}\) contains exactly \(n\) elements.
% As neither \(a\) nor \(b\) are in \(\acl(A)\), the set defined by \(\psi\) must not be finite. 
% By minimality, we know that \(\psi(\mcM)\) is cofinite. % Make sure we use this notation.   
We would very much like for \(X = \{y \in S \mid \psi(y) \land \phi(a, y)\}\) to be finite, making \(b \in \acl(A \cup \{a\})\). 
We will show this must be the case by contradiction.  % WRONG?: As \(\mcM \models \psi(b)\), it will suffice to exhibit \(a_1, \ldots, a_{n+1}\) for which \(\phi(a_i, b)\) holds. 
If this fails and \(X\) is infinite, then \(X\) is cofinite. 
Specifically, \(\{y \in S \mid \neg(\psi(y) \land \phi(a, y))\}\) contains \(m\) elements.  
We let \(\theta(x)\) assert that \(\{y \in S \mid \neg(\psi(y) \land \phi(x, y))\}\) contains exactly \(m\) elements.
\(\theta\) also defines a cofinite set, as otherwise \(a \in \acl(A)\).   
Let \(a_1, \ldots, a_{n+1}\) be such that for each \(i\), \(\theta(a_i)\) holds.
As \(\theta(a_i)\) holds, each set \(\{y \in S \mid \psi(y) \land \phi(a_i, y)\}\) is cofinite, as is their intersection. 
For any element, \(b'\) in the intersection, we have \(\psi(b')\) and \(a_i \in \{x \in S \mid \phi(x, b')\} for each \(a_i\) by construction.
This is a contradiction as \(\psi(b')\) asserts there are only \(n\) such elements.  


\textbf{Definition.} We say that % ???
Independent % something something
Basis % something something
Dimension

% 6.1.11
\textbf{Theorem.} There is a partial elementary map from the set defined by \(\phi\) in \(\mcM\) to that set in \(\mcN\). % This isn't what I'd seriously suggest as the statement. 

% define ``Prime over''
% 6.1.17
\textbf{Theorem.} \(\mcM\) is prime over \(\phi(\mcM)\).

\subsection{Proof of Morley's Categoricity Theorem}

In the previous sections we have succeeded in showing theories categorical in any uncountable cardinal are \omst and lack Vaughtian pairs. 
We now show the converse: two models \(\mcM, \mcN\)  of a complete \omst theory, \(T\), in a countable language lacking Vaughtian pairs we can show \(\mcM \isom \mcN\). 
As \(T\) is \omst it has a prime model, \(\mcM_0\), in which we have a minimal formula \(\phi\) (again as \(T\) is \(\omega\)-stable).
Moreover, as \(T\) has no Vaughtian pairs, \(\phi\) is strongly minimal. 
We begin by constructing a partial elementary map \(f:\phi(\mcM) \to \phi(\mcN)\) which is a bijection, which exists as per our theorem above.
% Both \(\phi(\mcM), \phi(\mcN)\) have cardinality \(\kappa\) and therefore dimension \(\kappa\). Maybe some more explanation. 

% We could use some work here:
Something something \(\mcM\) prime over something. Extending \(f\). \(f'\) is surjective. \(f'\) is an isomorphism finally giving us that \(\mcM \isom \mcN\). 
