At this point we've made good progress towards the proof of Morley's Categoricity theorem.
It suffices to show that uncountable categoricity (for any uncountable cardinal) is equivalent to \(\omega\)-stability and lacking Vaughtian pairs.
As this property doesn't depend on the uncountable cardinality, this suffices. 

We've already shown that \(\omega\)-stability and lacking Vaughtian pairs are necessary for uncountable categoricity.
To show these conditions are also sufficient we will define a notion of ``dimension'' for elementary extensions.
\(\omega\)-stability will tell us that we can view any two models as elementary extensions of the same \textit{prime model}.
The fact that, because of their cardinalities, any equinumerous models will have the same dimension over this prime model.
Having the same dimension over this prime model will suffice to show that these two models are isomorphic.
This will complete the proof. 

\subsection{Prime Models}

\textbf{Definition.} We say that a theory, \(T\),  has a prime model, \(\mcM \models T\) if, for every \(\mcN \models T\), we have \(\mcM \prec \mcN\). 
Clearly, prime models are unique up to isomorphism. Easy examples of prime models are \(\Q\) for fields of characteristic 0 and \(\N\) for \(\Th(\N, 0, s)\). 


%\subsection{Dimension}

% definition of a basis for \omst 210

% examples?

% ----------------------------------------

% Omega Stability => Prime model exists

% Strongly minimal formula with params from prime model

% 11 - Partial Elementary bijections

% 17 - It's an isomorphism

% ----------------------------------------

% 6.1.13, 15

% 6.1.11 

% 6.1.17

