At this point we've made good progress towards the proof of Morley's Categoricity theorem.
It suffices to show that uncountable categoricity (for any uncountable cardinal) is equivalent to \(\omega\)-stability and lacking Vaughtian pairs.
As this property doesn't depend on the uncountable cardinality, this suffices. 

We've already shown that \(\omega\)-stability and lacking Vaughtian pairs are necessary for uncountable categoricity.
To show these conditions are also sufficient we will define a notion of ``dimension'' for elementary extensions.
\(\omega\)-stability will tell us that we can view any two models as elementary extensions of the same \textit{prime model}.
The fact that, because of their cardinalities, any equinumerous models will have the same dimension over this prime model.
Having the same dimension over this prime model will suffice to show that these two models are isomorphic.
This will complete the proof. 

\subsection{Prime Models}

\textbf{Definition.} We say that a theory, \(T\),  has a prime model, \(\mcM \models T\) if, for every \(\mcN \models T\), we have \(\mcM \prec \mcN\). 
Clearly, prime models are unique up to isomorphism. Easy examples of prime models are \(\Q\) for fields of characteristic 0 and \(\N\) for \(\Th(\N, 0, s)\). % Examples?

\textbf{Theorem.} \omst theories have prime models.
 
% Proof: 4.2.20 p 136 (a crazy induction)
{\color{red}Prove this \(\ldots\)}

% Strongly minimal formula with params from prime model

\textbf{Definition.} We say that a \(\mcL_\mcM\)-formula \(\phi\) is \textit{minimal} if \(\phi(\mcM) = \{\bar{m} \in M \mid \mcM \models \phi(\bar{m})\) is infinite and has no infinite and co-infinite definable subsets. 
We may also call \(\phi(\mcM)\) minimal. 
We will call \(\phi\) % and \phi(\mcM)\) -- This could use some work 
\textit{strongly minimal} if it is  minimal in any elementary extension of \(\mcM\).

Now, assume we have a theory which is \omst and has no Vaughtian-pairs. 
As it is \omst it has a prime model.
We will show that it's being \omst also tells us that there is a minimal formula in its prime model.
We will also show this formula is strongly minimal since the theory has no vaughtian pairs. 

\textbf{Lemma.} If \(T\) is \omst there is a minimal formula in all \(\mcM \models T\).

We will build an infinite binary tree of formulae such that 
every node corresponds to an infinite definable subset of our model (and the formula defining it), 
and such that every node is a subset (or consequence) of its parent. 
We will see that every (infinite) path in this tree is a distinct type over a countable set and that there are \(2^{\aleph_0}\) such types. 
This contradicts the \(\omega\)-stability of \(T\).

It is easy to construct this tree. Let the root of the tree be the entire universe, \(M\), of our model which is defined by the formula \(v = v\).
Given a node in our tree defined by a formula \(\phi\) which defines an infinite subset, there must be a formula \(\psi\) such that both \(\phi \land \psi\) and \(\phi \land \neg \psi\) define infinite subsets, as \(\phi\) is not minimal. 
These two formulae/definable sets will be the children of \(\phi\) in our tree. 
As no formula is minimal, we can repeat this ad infinitum. 

Consider a path in this tree. We can consider the corresponding countable set of formulae consisting of the nodes on this path. 
Every finite subset contains a node of maximum depth, which defines an infinite subset of \(M\) and is therefore satisfiable. 
Our set of formulae corresponding to this path can therefore be completed to a type.
No two distinct paths can be completed to the same type. 
Specifically, if two paths diverge after a node \(\phi\), then they will disagree on the formula \(\psi\) as defined above. 

Moreover, there are only countably many nodes in this tree. 
Each formula in this tree contains finitely many constants from \(M\).
Therefore, all of these types can be defined over countably many constants from \(M\). 
We have presented \(2^{\aleph_0}\) types over a finite subset, contradicting the \(\omega\)-stability of \(T\).

\textbf{Lemma.} If \(T\) has no Vaughtian pairs and \(\phi\) is a minimal formula for \(\mcM \models T\) then \(\phi\) is strongly-minimal. 

% Proof

{\color{red}\(\ldots\) ????} % Do This one

% Define Algebraic Closure, 
% Define Basis, 
% Define dimensions of sets

% 11 - Partial Elementary bijections

% 17 - It's an isomorphism
