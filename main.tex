\documentclass[12pt]{article}
\usepackage[centertags]{amsmath}
\usepackage{amsfonts,amssymb,amsthm,color}

%
%MY SHORTCUTS
%
\newcommand{\mcE}{\mathcal{E}}
\newcommand{\mcC}{\mathcal{C}}
\newcommand{\mcA}{\mathcal{A}}
\newcommand{\mcB}{\mathcal{B}}
\newcommand{\mcM}{\mathcal{M}}
\newcommand{\mcN}{\mathcal{N}}
\newcommand{\mcL}{\mathcal{L}}
\newcommand{\R}{\mathbb{R}}
\newcommand{\C}{\mathbb{C}}
\newcommand{\Z}{\mathbb{Z}}
\newcommand{\N}{\mathbb{N}}
\newcommand{\Q}{\mathbb{Q}}
\newcommand{\satisfies}{\models}
\newcommand{\A}{\forall}
\newcommand{\E}{\exists}
\newcommand{\isom}{\cong}
\newcommand{\diag}{\text{Diag}}
\newcommand{\Th}{\text{Th}}
\newcommand{\eldiag}{\text{Diag}_\text{el}}
\newcommand{\qe}{quantifier elimination }
\newcommand{\qf}{quantifier-free }
\newcommand{\omst}{\(\omega\)-stable }
\newcommand{\acl}{\text{acl}}

%
%PAGE REFORMATTING
%
\setlength{\topmargin}{0 in}
\setlength{\headheight}{0.25 in}
\setlength{\oddsidemargin}{0.5 in}
\setlength{\textwidth}{6 in}
\setlength{\textheight}{8.5 in}
\setlength{\headsep}{0.25 in}
\newcommand{\singlespaced}{\renewcommand{\baselinestretch}{1}\normalfont}
\newcommand{\doublespaced}{\renewcommand{\baselinestretch}{2}\normalfont}
\singlespaced
\doublespaced

%
%THEOREM STYLES
%
\newtheorem{theorem}{Theorem}[section]
\newtheorem{lemma}[theorem]{Lemma}
\newtheorem{question}[theorem]{Question}

\theoremstyle{definition}
\newtheorem{definition}[theorem]{Definition}
\newtheorem{notation}[theorem]{Notation}
\newtheorem{remark}[theorem]{Remark}
\newtheorem{exercise}[theorem]{Exercise}
\newtheorem{example}[theorem]{Example}
\newtheorem{convention}[theorem]{Convention}

\theoremstyle{definition}
\newtheorem{defn}{Definition}[section]
\newtheorem{fig}{Figure}[section]
%
\theoremstyle{plain}
\newtheorem{res}[defn]{Result}
\newtheorem{thm}[defn]{Theorem}
\newtheorem{cor}[defn]{Corollary}
\newtheorem{prop}[defn]{Proposition}
\newtheorem{lem}[defn]{Lemma}
%
\theoremstyle{remark}
\newtheorem*{case}{Case}
\newtheorem{rem}[defn]{Remark}
\newtheorem{ex}[defn]{Example}
\newtheorem{exs}[defn]{Examples}

%
%EQ. NUMBERING
%
\numberwithin{equation}{section}
\renewcommand{\theequation}{\thesection.\arabic{equation}}
%
\def\thetitle{A Proof of Morley's Categoricity Theorem}
\def\theauthor{Adam Freilich}
\def\theadvisor{Advisor: Dr. Henry Towsner}
\def\theyear{2014}
%
\pagenumbering{roman}

\begin{document}
\large\newlength{\oldparskip}\setlength\oldparskip{\parskip}\parskip=.3in
\thispagestyle{empty} \doublespaced
\begin{center}
\vspace*{\fill} \thetitle

\theauthor


A THESIS

in

Mathematics

\singlespaced

\large Presented to the Faculties of the University of
Pennsylvania in Partial
 Fulfillment of the Requirements for the Degree of Master of
 Arts
 
\doublespaced

\large
\theyear
\end{center}

\singlespaced

\noindent\makebox[0in][l]{\rule[2ex]{3in}{.3mm}}
 Henry Towsner \\  \hspace*{.5mm} Supervisor of Thesis

\noindent\makebox[0in][l]{\rule[2ex]{3in}{.3mm}}  David Harbater \\  \hspace*{.5mm} Graduate Group Chairperson \vspace*{\fill}

\normalsize\parskip=\oldparskip

\doublespaced


%\include{Abstract}
%\vspace*{\fill} \begin{center} {\it
Dedicated to the winners and the losers.}
\end{center}
\vspace*{\fill}



\pagenumbering{arabic}
\pagestyle{myheadings} \markright{}

\section{Basic Notions}
\subsection{Defining Categoricity}

First, I recall the defition of categoricity in some cardinal \(\lambda\):

\begin{definition}\label{def_categoricity}
A theory \(T\) is \(\lambda\)-categorical if \(\mcM \isom \mcN\) whenever \(\mcM, \mcN \models T\) and \(|M| = |N| = \lambda\).
\end{definition}

\begin{example}\label{example_categoricity_sets}
Consider the theory of infinite sets in the pure language of equality. 
This theory is \(\lambda\)-categorical for all \(\lambda\).
Consider two models \(\mcM, \mcN\) such that \(|M|=|N|=\lambda\). 
By hypothesis we have a bijection \(f: M \to N\). \(f\) is an isomorphism of sets, so \(M \isom N\).
\end{example}

Recall that a theory, \(T\), is complete if for all sentences \(\phi\) we have either \(T \models \phi\) or \(T \models \neg \phi\). 
Note that this theory must also be complete. 
More generally, Vaught's test tells us that theories without finite models that are categorical in some infinite cardinal \(\kappa\) are complete. 
(If such a theory were not complete, it would have models of cardinality of \(\lambda\) disagreeing on some \(\phi\) which couldn't then be isomorphic)

\begin{example}\label{example_categoricity_equiv}
Consider the theory of an equivalence relation with exactly two infinite classes. 
This theory is ``\(\aleph_0\)-categorical''.
We will often call an \(\aleph_0\)-categorical theory ``countably categorical''.
These terms are synonomyous.  
Consider two countable models \(\mcM, \mcN\). 
There must be countably many members of each equivalence class in any countable model. 
We can present a bijection \(f_1\) between members of the first equivalence class in  \(\mcM\) and those in \(\mcN\) and a similar bijection \(f_2\) for the second equivalence class. 
We see that \(f = f_1 \cup f_2\) is a bijection and actually an isomorphism. 

On the other hand, if \(\lambda > \aleph_0\) is an infinite cardinal, then this theory isn't \(\lambda\)-categorical. 
It suffices to note that we can have the two non-isomorphic cases: where the first equivalence class has cardinality \(\lambda\) and the second is countable 
and where both classes are of cardinality \(\lambda\). Vaught's theorem tells us that even though this theory isn't \(\aleph_1\) categorical, it is still complete. 
\end{example}

\begin{example}\label{example_categoricity_integers}
Consider the theory of (\(\Z, s\)), the integers with successor. 
Included in this theory is the fact that every element has a unique successor and predecessor and that there are no cycles (i.e. no number is its own successor, or the sucessor of its successor, etc.). 
Using this fact, we observe that each model of this theory must consist of some number of \(\Z\)-chains. 
For uncountable cardinals \(\kappa\), we know that, up to isomorphism, the only model of cardinality \(\kappa\) consists of \(\kappa\) many \(\Z\)-chains. 
So this theory is categorical for all uncountable cardinals and must be complete. 
That being said, there are countably many non-isomorphic countable models, which are the familiar \(\Z\), \(n\) copies of \(\Z\) for any natural \(n\) or countably many copies of \(\Z\). 
This theory is not countably categorical, but it is \(\kappa\) categorical for all uncountable \(\kappa\). 
\end{example}

\begin{example}\label{example_categoricity_sequence}
We consider the language \(\mcL = \{<, c_1, c_2, c_3, \ldots\}\) which has countably many constant symbols indexed by the natural numbers. 
In our theory \(<\) is a dense linear order and \(c_i\) is a strictly increasing sequence. 
If we look at any finite sublanguage, we can see that this theory is complete (by constructing an explicit isomorphism). % put in more details or reference?
This theory is never categorical as we can present models in which \(c_i\) is unbounded, has a least upper bound or is bounded without a least upper bound. 
\end{example}

Morley's categoricity theorem will state that these are the only four cases. 
That theorem will be the main focus of this thesis. 

\subsection{Stating Morley's Categoricity Theorem}

\begin{theorem}\label{theorem_morleys_categoricity}
If a first order theory in a countable language is categorical in some uncountable cardinal, it is categorical in all uncountable cardinals. 
\end{theorem}

I present a very general sketch of the proof of this theorem which I will flesh out throughout the rest of this work. 

We will define two properties of theories: being \(\omega\)-stable and having Vaughtian pairs. 
It turns out that lacking \(\omega\)-stability or having Vaughtian pairs will prevent uncountable categoricity, so all uncountably categorical theories lack Vaughtian pairs and are \(\omega\)-stable. 

We then observe that these two conditions, lacking Vaughtian pairs and being \(\omega\)-stable, are sufficient to show categoricity in all uncountable cardinals. 
Hence, the categoricity of a theory in some uncountable cardinal must imply \(\omega\)-stability and the absence of Vaughtian pairs, which, in turn, imply categoricity in all other uncountable cardinals.

% Once I complete this proof, I will elaborate more on issues of stability in general, morley rank and forking.  


\section{\(\omega\)-stability}
In this section, our goal is to define stability (and \(\omega\)-stability in particular) and show that theories which aren't \(\omega\)-stable cannot be uncountably categorical for any uncountable cardinality. 

In order to define stability we need to recall a few definitions and facts about types. 
Let \(\mcM\) be an \(\mcL\)-structure and \(A \subseteq M\). 
It will often be natural to want to add constant symbols for every element of \(A\) to our language. 
We will call the resulting language \(\mcL_A\). \(\mcM\) has a natural definition as an \(\mcL_A\)-structure. 
In general, we call the set of \(\mcL_A\)-sentences true in \(\mcM\) ``\(\Th_A(\mcM)\)'' and, in the special case where \(A = M\) we call it \(\eldiag(\mcM)\) or the ``elementary diagram of \(\mcM\)''. 

\section{Defining Types}

\begin{definition}\label{def_types}
Let \(A, \mcM, \mcL\) as above. 
An \(n\)-type over \(A\), \(p\), is a set of \(\mcL_A\)-formulae in \(n\) free variables (\(v_1, \ldots, v_n\)) for which \(p \cup \Th_A(\mcM)\) is satisfiable.
We say that \(p\) is complete if, for every \(\mcL_A\)-formula \(\phi(v_1, \ldots, v_n)\), either \(p \ni \phi\) or \(p \ni \neg \phi\). 
We call the set of all complete \(n\) types over \(A\) ``\(S_n^\mcM(A)\)''. 

In the event that \(A = \emptyset\), \(T\) is a complete theory and \(\mcM, \mcN \models T\), we observe that \(S_n^\mcM(\emptyset) = S_n^\mcN(\emptyset)\) as both consist of complete \(n\)-types consistent with \(T\). 
We define \(S_n(T) = S_n^\mcM(\emptyset)\).
\end{definition}

Let's consider a few simple examples of types when  \(\mcM = (\N,s)\), the natural numbers with successor and \(A = \{0\}\) (we've added a constant symbol for 0). 
One example or an incomplete type is when we take \(p = \{v_1 \neq 0\}\). 
As \(\N \models \phi(2)\) for all \(\phi \in p\), \(p \cup \Th_\emptyset(\N)\) must be satisfiable, and \(p\) is a type. 
Still, neither \(v_1 = s(s(s(0)))\) or its negation is in \(p\) so it is not complete. 

There is one clear source of complete types. 
Every element of \(\N\) has an associated complete type. 
Let \(p(v) = \{\phi(v) \mid \N \models \phi(1)\}\). 
As \(\N\) is a model realizing \(p \cup \Th_\emptyset(\N)\) when \(v_1\) is interpreted as 1, this set of formulae is a type. 
As every formula in one free variable is either true or false for 1, this type is complete. 
We say \(p\) is realized by 1. 
More generally, we say \(\bar{x} \in M^n\) realizes an \(n\)-type \(p\) if \(\mcM \models \phi(\bar{x})\) for all \(\phi \in p\).  

This, though, is not the only source of types. 
If we take \(p = \{v_1 \neq 0, v_1 \neq s^1(0), v_1 \neq s^2(0), \ldots\}\) we get a set of \(\mcL_{\{0\}}\)-formulae in one free variable. 
We note that \(p \cup \Th_{\{0\}}(\N)\) is satisfiable (as it is finitely satisfiable). 
We can extend \{p\} to a complete type, but that type cannot be realized by any natural number, since an element realizing \(p\) is greater than every natural number by construction. 
We see that it isn't always the case that a type is realized in every model, even though it is consistent with the theory of that model. 

\section{Defining Stability}

\begin{definition}\label{def_stability}
Let \(T\) be a complete theory in a countable language and \(\kappa\) be an infinite cardinal. 
\(T\) is \(\kappa\)-stable if \(|S_n^\mcM(A)| = \kappa\) whenever \(\mcM \models T\) and \(|A| = \kappa\). 
Note: \(\aleph_0\)-stable theories are called \(\omega\)-stable for historical reasons. 
\end{definition}

First, we will show that the theory of dense linear orders without endpoints is not \(\omega\)-stable. 
\(\mcM = (\R, <)\) is a model of this theory and \(A = \Q\) is a subset of size \(\aleph_0\). 
Still, every real number realizes a different type in \(S_1^\R(\Q)\) (as every two distinct real numbers have a rational number between them). 
So there must be at least as many types as there are real numbers; there are uncountably many types.  

Recall that we mentioned in the proof outline of Morley's Categoricity theorem that all uncountably categorical theories are \(\omega\)-stable. 
We mentioned the theory of infinite sets and the theory of (\(\Z, s\)) as uncountably categorical theories. 
Let's observe that both of these theories are \(\omega\)-stable. 

\begin{example}\label{example_omst_sets}
The theory of infinite sets is \omst. 
First, consider the complete 1-types, \(p(v_1)\). \(p(v_1)\) must assert either that \(v_1 = a\) for exactly \(a \in A\) or that \(v_1 \neq a\) for all \(a \in A\). 
Moreover, as the theory of infinite sets admits \qe all formulae are equivalent to boolean combinations of such formulae. % Add Citation 
We have now identified all 1-types (there is one for each element in \(A\) and one more for when \(v_1\) isn't in \(A\)). 
This can be easily extended to \(n\)-types for \(n > 1\)
\end{example}

\begin{example}\label{example_omst_Z}
The theory of \((\Z, s)\) is \omst. 
We will again consider 1-types and leave the general argument to the reader. 
We can assume, WLOG, that no two elements of \(A\) are in the same \(\Z\)-chain. 
Our 1-type either asserts that \(v_1\) is in the same \(\Z\)-chain as a single element of \(A\) and determines its position relative to that element, or asserts that \(v_1\) isn't in the same \(\Z\) chain as any element of \(A\). 
Note that the number of possibilities mentioned above is countable and that these possibilities are exhaustive.
I assert that if we know which of these possibilities is true of an element \(a\), i.e. what atomic formula \(\psi(a, \bar{b})\) hold for \(\bar{b} \in A\), we can determine the type of \(A\). 
It suffices to show that we can determine wether \(\phi(a, \bar{b})\) holds of \(a\) given that atomic information.  
Quantifier elminiation tells us that for all formulae \(\phi(\bar{v})\), \(\Th(\Z, s) \models \A \bar{v} \phi(\bar{v}) \psi(\bar{v})\) for some quantifier free \(\psi\).
Given an arbitrary \(\phi(a, \bar{b})\) with \(\bar{b} \in A\) we know that \(\phi(a, \bar{b}) \iff  \psi(a, \bar{b})\) and by hypothesis we know wether or not \(\psi(a, \bar{b})\) holds. 
This suffices to decide \(\phi(a, \bar{b})\) and show the \(\omega\)-stability of the theory of \((\Z, s)\).
\end{example}

\section{A Theorem about \(\omega\)-stability}
We now aim to show that no theory \(T\) can be uncountably categorical, unless it is \(\omega\)-stable.
It suffices to show that for all uncountable \(\kappa\) we can present two non-isomorphic models of \(T\) of cardinality \(\kappa\), given only that \(T\) is not \(\omega\)-stable.  
We will do so by finding one model realizing only countably many types over any countable set, and one realizing uncountably many types over some countable set. 
Clearly, two such models will not be isomorphic, so \(T\) will not be categorical. 

\subsection{Realizing Uncountably Many Types}
\begin{theorem}\label{theorem_realizing_uncountable_types}
If \(T\) is an \omst theory, there are models of \(T\) of any uncountable cardinality realizing uncountably many types over a countable set.
\end{theorem}

\begin{proof}
As \(T\) isn't \omst, for some countable \(A\) we have that \(|S_n^\mcM(A)|\) is uncountable. 
We can realize at least \(\aleph_1\) of these in some elementary extension of \(\mcM\) which can be taken to be of cardinality exactly \(\kappa\) for any desired uncountable \(\kappa\).
Say \(\{p_i\}\) is a set of \(\aleph_1\) many types from \(S_n^{\mcM}(A)\).
Let \(C = \{c_i| i \in \aleph_1\} \) be \(\aleph_1\) constant symbols in a new language \(\mcL^* = \mcL_\mcM \cup C\). 
Let \(\Gamma = T \cup \eldiag(\mcM) \cup \{p_i(c_i)\}\) (we assert each type \(p_i\) is realized by \(c_i\) giving us the realization of uncountably many types).
 
To show that \(\Gamma\) is finitely satisfiable, it will suffice to show that sets of the form \(\Gamma_0 = T \cup \eldiag(\mcM) \cup \bigcup_{i \in I}\bigcup_{\phi(c_i) \in (p_i)_0(c_i)} \phi(c_i)\) are satisfiable, where \((p_i)_0 \subseteq p_i\) are finite.  
\(\Gamma_0\) is satisfiable iff \(\Delta = T \cup \eldiag(\mcM) \cup \bigcup_{i \in I}\E x \bigwedge_{\phi(c_i) \in (p_i)_0(c_i)} \phi(x)\) is satisfiable.
For each of these existential sentences, \(\psi\), \(\mcM\) is either a model of \(\psi\) or of \(\neg \psi\). 
As these types are, by definition, consistent with \(\eldiag(\mcM)\), it must be that \(\mcM \models \psi\). 
Therefore, \(\mcM \models \Delta\) and \(\Gamma_0\) is satisfiable. 
By compactness, we have a model of \(\Gamma\) of cardinality \(|\mcL^*| \leq \kappa\) and by the upward L\"owenheim-Skolem theorem, a model of cardinality \(\kappa\).
\(\Gamma\) asserts that uncountably many types are realized over \(A\), as desired. 
\end{proof}

\subsection{Indiscernables}
Now, we must show the existence of a model of size \(\kappa\) which realizes only countably many types over every countable set. 
To do so, we will have to construct some machinery.
This will be our guiding intuition: to ensure our model is of size \(\kappa\), we will start out with \(\kappa\) many virtually identical (indiscernable) elements. 
To ensure our model is indeed a model of \(T\), we will add elements to whitness existentials. 
We will see that this method of construction will realize few types.

\begin{definition}\label{definition_order_indiscernibles}
Let \(\mcM\) be a model.
We say that a set \(\{x_i \in M \mid i \in I\}\) where \(I\) is an infinite ordered set is a sequence of \textit{order indiscernibles (of order type \(I\))} if, given two sequences from \(I\), 
\(i_1 < i_2 < \ldots < i_n\) and \(j_1 < j_2 < \ldots < j_n\), we will have \(\mcM \models \phi(i_1, \ldots, i_n) \iff \mcM \models \phi(j_1, \ldots, j_n)\). 
Note that the order for which \(i_1 < i_2\) is an ordering on \(M\).
The existence of this ordering on \(M\) in no way requires the binary relation symbol ``\(<\)'' to be in our language \(\mcL\). 
Intuitively, these elements cannot be distinguised based on first order properties, modulo their relative positions in \(M\).  
\end{definition}

Recall that the theory of dense linear orders without endpoints admits \qe and it's easy to see that any increasing sequence from such an order is a sequence of indiscernibles. 
Here, our ordering on the elements of the universe of our model agrees with the ordering present in the languge. 
If we removed the ordering from our language (i.e. we work in the language of sets) our order indiscernibles are still order indiscernibles. 

\begin{theorem}\label{theorem_order_indiscernibles}
Given an ordered set \(I\) and an \(\mcL\)-theory, \(T\), which has infinite models, there always exists an \(\mcM \models T\) containing a set of order insicernibles of order type \(I\). 
\end{theorem}

\begin{proof}
Let \(\mcL_I \supset \mcL\) be an extension of our language including a constant symbol corresponding to each \(i \in I\), that is \(\mcL' = \mcL \cup \{c_i \mid i \in I\}\).
We note that we can present a first order \(\mcL'\)-theory, \(T'\), such that all \(\mcM \models T'\) are models of \(T\) with a set of order indiscernibles of order type \(I\).
\(T'\) includes \(T\), \(\{c_i \neq c_j \mid i \neq j\}\) and \(\{\phi(c_{i_1}, \ldots, c_{i_n}) \iff \phi(c_{j_1}, \ldots, c_{j_n})\}\) for all \(\mcL\)-formulae \(\phi\), and all sequences \(i_1 < \ldots < i_n, j_1 < \ldots < j_n\). 
It suffices to show that finite subsets of \(T'\) are satisfiable. 
That this is true is an application of Ramsey's Theorem, which states: 
If \(S\) is an infinite set whose subsets of size \(n\) are colored with \(c\) different colors, there exists a \(T \subseteq S\) for which every subset of size \(n\) of \(T\) has the same color. 
(We will prove Ramsey's theorem, theorem \ref{ramseys_theorem}, in the appendix.)

Let \(\Phi\) be a finite set consisting of \(\mcL\)-formulae in at most \(k\) free variables. 
Let \(\mcM\) be an infinite model of \(T\). 
Let \(\mcL^* = \mcL \cup \{c_i | i \in \N\}\) where the \(c_i\)s are constant symbols. .
It suffices to show that \(\mcM\) is a model of the theory \(T^*\) including: \(T\), \(\{c_i \neq c_j \mid i \neq j\}\) and \(\{\phi(c_{i_1}, \ldots, c_{i_n}) \iff \phi(c_{j_1}, \ldots, c_{j_n})\}\) for all \(\phi \in \Phi\), and all sequences \(i_1 < \ldots < i_n, j_1 < \ldots < j_n\) of length at most \(k\). 
(All finite subsets of \(T'\) only include our indiscernability condition for finitely many formulae and finitely many constant symbols).

In applying Ramsey's Theorem, we let \(S = M\). 
We take \(n = k\) (that is we color subsets of size \(k\)).
We let our set of \(2^{|\Phi|}\) colors be the subsets of \(\Phi\). 
If \(N\) is a subset of \(S\) of size \(n\) let \(a_1 < \ldots < a_n\) enumerate its elements. 
Assign the set \(N\) the color \(\{\phi \in \Phi \mid \mcM \models \phi(\bar{a})\}\). 
If \(\phi\) has only \(m\) free variables and \(m < n\) we interpret \(\phi(\bar{a}\) as \(\phi(\bar{b})\) where \(\bar{b}\) is the first \(m\) elements of \(\bar{a}\).
Interpreting the \(c_i\)s as the elements of \(T\) (whose existence is guarunteed by Ramsey's Theorem) is sufficient by construction.

Thus, \(T'\) is satisfiable and its model is the desired \(\mcM\).
\end{proof}

\subsection{Realizing Only Countably Many Types}

\begin{definition}\label{definition_skolem_functions}
We recall that an \(\mcL\)-theory, \(T\), is said to have \textit{built-in Skolem functions} when for all \(\mcL\)-formulae \(\phi(v, \bar{w})\) is a function symbol \(f_\phi \in \mcL\) such that \(T \models \A \bar{w} [(\E v \phi(v, \bar{w})) \to \phi(f_\phi(\bar{w}), \bar{w})]\), in other words, the Skolem functions provide whitnesses for all existentials. 
\end{definition}

\begin{theorem}\label{theorem_skolem_function_extension}
Note that, for any \(\mcL\)-theory, \(T\), we can find \(\mcL^* \supseteq \mcL\) and \(T^* \supseteq T\) such that \(T^*\) has built in skolem functions. 
Moreover, if \(\mcM \models T\) we can extend \(\mcM\) to an \(\mcL^*\)-structure, \(\mcM^*\), such that \(\mcM^* \models T^*\). 
\end{theorem}

\begin{proof}
We will construct \(\mcL^*, T^*\) inductively. 
Let \(\mcL_0 = \mcL\) and \(T_0 = T\). 
For \(i > 0\) let \(\mcL_i = \mcL_i \cup \{f_\phi \mid \phi \text{ is an } \mcL_{i-1}\text{-formula}\}\) where \(f_\phi\) is a function symbol.
If \(\phi\) is a formula in \(n\) free variables, the arity of \(f_\phi\) is \(n-1\). 
\(T_i\) is \(T_{i-1} \cup \{\A \bar{w} [(\E v \phi(v, \bar{w})) \to \phi(f_\phi(\bar{w}), \bar{w})] \mid \phi \text{ is an } \mcL_{i-1}\text{-formula}\}\).
Taking \(\mcL^* = \bigcup_{i \in \N} \mcL_i\) and \(T^* = \bigcup_{i \in \N} T_i\) will have built in skolem functions by construction.

To show that \(\mcM^*\) exists, it suffices to show that, if \(\mcM \models T_i\) then \(\mcM \models T_{i+1}\) given the correct interpretation of the function symbols in \(\mcL_{i+1}\).
If \(\mcM \models \E x \phi(x, \bar{w})\) then it must be that, for some \(b \in M\) that \(\mcM \models \phi(b, \bar{w})\), so we let \(f_\phi(\bar{w}) = b\). 
Otherwise, we let \(f_\phi(\bar{w})\) be an arbitrary element of \(M\). 

Note that if \(\mcL\) is countable, then \(\mcL^*\) is countable as well as we add only countably many function symbols at each of the countably many stages.
\end{proof}

\begin{theorem}\label{theorem_tarski_vaught_test}
If \(\mcM \subseteq \mcN\) then \(\mcM \prec \mcN\) if and only if for every formula \(\phi(v, \bar{w})\), \(\bar{a} \in M\) and \(b \in N\) such that \(\mcN \models \phi(b, \bar{a})\) there is a \(c \in M\) such that \(\mcM \models \phi(c, \bar{a})\).
\end{theorem}

\begin{proof}
If \(\mcM \prec \mcN\) then \(\mcM \models \E x \phi(x, \bar{a}) \iff \mcN \models \E x \phi(x, \bar{a})\). This suffices to show the ``only if'' direction.

In the other direction, we show that \(\mcM \models \phi(\bar{a}) \iff \mcN \models \phi(\bar{a})\) by induction on formulae. 
As \(\mcM \subseteq \mcN\), \(\mcM \models \phi(\bar{a}) \iff \mcN \models \phi(\bar{a})\) holds for all atomic \(\phi\).
Assume for all \(\bar{a}\) that  \(\mcM \models \phi(\bar{a}) \iff \mcN \models \phi(\bar{a})\) and \(\mcM \models \psi(\bar{a}) \iff \mcN \models \psi(\bar{a})\).
It suffices to show:
\begin{enumerate}
\item \(\mcM \models \neg \phi(\bar{a}) \iff \mcN \models \neg \phi(\bar{a})\). By assumption \(\mcM \not \models \phi(\bar{a}) \iff  \mcN \not \models \phi(\bar{a})\) which gives us our desired result. 
\item \(\mcM \models (\phi\land\psi)(\bar{a}) \iff \mcN \models (\phi\land\psi)(\bar{a})\). If \(\mcM \models \phi(\bar{a})\) and \(\mcM \models \psi(\bar{a})\). By the inductive hypothesis,  \(\mcN \models \phi(\bar{a})\) and \(\mcN \models \psi(\bar{a})\). So \(\mcM \models (\phi \land \psi)(\bar{a}) \implies \mcN \models (\phi \land \psi)(\bar{a})\). The same argument applies in the other direction. 
\item \(\mcM \models \E v \phi(v, \bar{a}) \iff \mcN \models \E v \phi(v, \bar{a})\). If \(\mcM \models \E v \phi(v, \bar{a})\) then for some \(b\) \(\mcM \models \phi(b, \bar{a}) \iff \mcN \models \phi(b, \bar{a})\) by the induction hypothesis. If \(\mcN \models \E v \phi(v, \bar{a})\) then for some \(b\) we have \(\mcN \models \phi(b, \bar{a})\) and, by hypothesis, there is a \(c \in M\) for which \(\mcM \models \phi(c, \bar{a})\) so \(\mcM \models \E v \phi(v, \bar{a})\) as desired.    
\end{enumerate}
\end{proof}

\begin{cor}\label{closure_under_skolem_functions}
If \(T\) has built in skolem functions and \(\mcM \subseteq \mcN \models T\) then \(\mcM \prec \mcN\). 
As \(\mcM\) is closed under the skolem functions, we can whitness existentials in accordance with the hypothesis of our previous theorem. 
\end{cor}

\begin{definition}\label{definition_skolem_hull}
Let \(\mcM \models T\) where \(T\) has built in skolem functions. 
If \(A \subset M\) then we can take \(\mathcal{H}(A)\) which is the smallest submodel of \(\mcM\) containing \(A\).
As \(\mathcal{H}(A)\) is a submodel of \(\mcM\) it is closed under skolem functions, so \(\mathcal{H}(A) \prec \mcM\).
We call \(\mathcal{H}(A)\) the ``skolem hull'' of \(A\).
\end{definition}

\begin{definition}\label{definition_ehrenfeuct_mostowski_model}
If \(T\) is an \(\mcL\)-theory with build in skolem functions and infinite models, given any infinite ordered set \(I\), we find a \(\mcM \models T\) such that \(I \subset M\) is an infinte set of order indiscernibles. 
We call \(\mathcal{H}(I)\) an Ehrenfeucht-Mostowski model. 
Note that if \(\mcL\) is countable \(|\mathcal{H}(I)| = |I|\) as every element of \(\mathcal{H}(I)\) is an \(\mcL_{I}\)-term, of which there are \(|\mcL|+|I| = |I|\).
\end{definition}

\begin{theorem}\label{theorem_ehrenfeuct_mostowski_types}
Let \(T\) be an arbitrary theory and \((\kappa, <)\) be a set of order indiscernibles of order type \(\kappa\).
Let \(\mcM = \mathcal{H}((\kappa, <))\models T\). 
\(\mcM\) is a model of size \(\kappa\) realizing countably many types over any countable \(A\).
\end{theorem}

\begin{proof}
As every element of \(M\) (and of \(A\)) must be a Skolem term (i.e. of the form \(f_\phi(\bar{x})\) for some vector of indiscernables), it suffices to look at the types over countable subsets of \(\kappa\).
Let \(X\) be this countable subset of \(\kappa\). 
It suffices to show that for each fixed \(n, \phi\), terms of the form \(f_\phi(y_1, \ldots, y_n)\) realize only countably many types. 
(We then have divided \(M\) into countably many sets each realizing at most countably many types).
We see that \(f_\phi(y_1, \ldots, y_n)\) and \(f_\phi(z_1, \ldots, z_n)\) realize the same type if, for each \(i \leq n\) there is no \(x \in X\) between \(y_i\) and \(z_i\) or that \(y_i\) and \(z_i\) determine the same cut in \(X\).  
This is because indiscernability tells us \(\mcM \models \psi(f_\phi(\bar{y}), \bar{x}) \iff \mcM \models \psi(f_\phi(\bar{z}), \bar{x})\) when the relative orders of the indiscernibles in \(\bar{y}\bar{x}\) and in \(\bar{z}\bar{x}\) are the same. 
It now suffices to note that, as \(\kappa\) is well ordered, we get only countably many cuts of \(X\). 
This deals with the 1-types. Dealing with the \(n\)-types only requires dealing with \(n\)-tuples of formulae instead of the single formula \(\phi\) we adressed here. 
\end{proof}

\begin{theorem}\label{theorem_omega_stability_categoricity}
Any theory which is not \omst cannot be categorical in any uncountable cardinal. Contrapositively, all uncountably categorical theories are \(\omega\)-stable
\end{theorem}

\begin{proof}
We have succeded in showing that there is a model of any \(\omega\)-unstable theory realizing only countably many types over any countable set.
This model cannot be isomorphic to one realizing uncountably many types over a countable set, which we showed must exist as well. % reference the theorems from before
\end{proof}


\section{Vaughtian Pairs}
\subsection{Definition}
The (only) obstacle to uncountable categoricity aside from \(\omega\)-instability is the existence of Vaughtian pairs. %Introduce V-pairs first
We will be able to use a Vaughtian pair of models of a theory to construct a model of size \(\kappa\) with a countable definable subset.  
Such models prevent uncountable categoricity because given any model of cardinality \(\kappa\) we can construct an elementarily equivalent model in which all infinite definable sets are of cardinality \(\kappa\).
We simply add \(\kappa\) many vectors of constant symbols for each formula \(\phi\) defining an infinite set, assert that none of the vectors of constant symbols are equal. This is clearly finitely satisfiable and its model has the desired property.
First, let's define what it means for two models to form a Vaughtian pair. % Possible Addition: Definition of kappa lambda models (definition in previous commits)

\begin{definition}\label{definition_vaughtian_pairs}
\((\mcN, \mcM)\) form a Vaughtian pair if \(\mcM \prec \mcN\) and there is an \(\mcL_\mcM\)-formula \(\phi(\bar{v})\) (in some number of free variables) for which
\(\{\bar{m} \in N^k \mid \mcN \models \phi(\bar{n})\}\) is an infinite subset of \(M^k\). %7-1 m? I don't get it (I don't get what's not to get?)
In other words, a formula with parameters in \(M\) defines an infinite subset of \(N^k\) containing no elements of \(N^k \setminus M^k\).
\end{definition}

Even though we write \(\mcM \prec \mcN\), we write the Vaughtian pair in the other order, that is as \((\mcN, \mcM)\), because we can think of this pair of models as a single model (\(\mcN\)) with a distinguished subset corresponding to \(\mcM\).
This will often be useful in proving theorems and lemmas about Vaughtian pairs, so we will flesh out the construction here. 
If \(\mcM, \mcN\) are \(\mcL\)-structures, let  \(\mcL^* = \mcL \cup \{U\}\). 
We look at (\(\mcN, \mcM\)) as an \(\mcL^*\)-structure which shares its underlying set and interpretations of \(\mcL\) with \(\mcN\) but where \(U\) is the subset \(M\). 
This predicate allows us to state that a formula \(\phi(\bar{v})\) is true in \(\mcM\) with slight modification to \(\phi\).
Specifically, we will have \((\mcN, \mcM) \models \phi_U(\bar{v}) \iff \mcM \models \phi(\bar{v})\).
For \(\phi(\bar{v})\) \qf we let \(\phi_U(\bar{v}) = \bigwedge U(v_i) \land \phi(\bar{v})\) and for \(\phi(\bar{v}) = \exists v \psi, \phi_U(\bar{v}) = \E v (U(v) \land \psi_U)\). 
In the other cases, do the natural thing. %Say this better
% We might want to show how we can also state that this \mcL^* structure is a Vaughtian pair as a first order thing (mostly). 

\begin{theorem}\label{thm_countable_vaughtian_pairs}
Assuming a theory \(T\) has a Vaughtian pair \(\mcN, \mcM\), it has a Vaughtian pair \((\mcN_0, \mcM_0)\) of countable models. 
\end{theorem}

\begin{proof}\label{proof_countable_vaughtian_pairs}
In order to show the existence of a countable model, we will want to use the L\"owenheim-Skolem theorm.
To do so, we will have to view the pair of models as a single model. 

Our goal is to demonstrate an \(\mcL^*\)-theory in a countable language asserting that \((\mcN, \mcM)\) is Vaughtian pair.
By the L\"owenheim-Skolem theorem, we will get out countable pair of models.
This comes in two parts: asserting \(\mcM \prec \mcN\) and asserting that we can isolate an infinite subset of \(M\). 
We assert the former with the formulae \((\bigwedge U(v_i) \land \psi(\bar{v})) \to \psi_U(\bar{v})\). 
Let \(\phi(\bar{v})\) be the \(\mcL_\mcM\)-formula which isolates an infinite subset of \(M^k\). 
We can assert that \(\phi\) has infinitely many realizations by \(\E \bar{v}_1, \E \bar{v}_2, \ldots \E \bar{v}_n \bigwedge \bar{v}_i \neq \bar{v}_j \land \bigwedge \phi(\bar{v}_i)\). %This argument got out of hand
We can assert that \(\phi\) only holds for vectors from \(M^k\) by \(\A \bar{v} (\phi(\bar{v}) \to U(\bar{v}))\).
Finally, we assert that \(\mcN\) is a proper extension with \(\E x \neg U(x)\). 
By construction, the countable model of this theory which is satisfied by \((\mcN, \mcM)\) will be our countable model.
\end{proof}

For the next few claims we will need to use a property called homogeneity which will allow us to show that certain countable models are isomorphic.

\begin{definition}\label{definition_homogeneity}
We call a countable model \(\mcM\) is homogeneous if all finite \(A \subset M\) and functions \(f: A \to M\) for which \(\mcM \models \phi(\bar{v}) \iff \mcM \models \phi(f(\bar{v}))\) -- we call this property being ``partial elementary'' -- and every \(m \in M\), there is a \(f': A \cup \{m\} \to M\) extending \(f\) which is partial elementary. % Examples?
\end{definition}

The first interesting thing we will note about countable homogeneous models is that any partial elementary map \(f: A \to M\) (where \(A \subset M\) is finite) can be extended to an automorphism. 
To see this, note that partial elementary maps have partial elementary inverses. And \(f_0 = f\) %this bad
We will alternatively extend the domain and range of \(f_i\) by elements in \(M\). 
By construction, \(f^* = \bigcup_{<\omega}f_i\) has domain and range equal to all of \(M\). %something here
Additionally, it is easy to verify that \(f^*\) is a homomorphism. 

This is why homogeneity is a useful property. We can use it to build isomorphisms, which will be useful soon. 

\begin{lemma}\label{lemma_types_isomorphism}
The next thing we note about countable homogeneous models, \(\mcM\) and \(\mcN\), of the same complete theory, \(T\), is that if \(\mcM\) and \(\mcN\) realize the same types in \(S_n(T)\), they are isomorphic.
\end{lemma}

\begin{proof}\label{proof_types_isomorphism}
To show this, we will build a map in a way very similar to the way we built \(f^*\) above. 
Note that we need \(T\) to be complete in order for \(f_0=\emptyset\) to be partial elementary.
We will enumerate the elements in the universe of both models, and iteratively add elements to the domain and range of a partial elementary map. 
If we can do this, the union of this sequence of maps will be an isomorphism.
As the inverse of a partial elementary map is also partial elementary, it will suffice to show that we can extend a partial elementary map to include an arbitrary element in its range. 
Let \(dom(f) = \{m_1, \ldots, m_i\}, rng(f) = \{n_1, \ldots, n_i\}, f(m_j) = n_j\).
Now, choose \{m\} an arbitrary element of \(M\), our goal is to find \(g \supset f\) which is partial elementary and \(m \in dom(g)\).
Note that we can find a partial elementary \(h\) for which \(dom(f) \cup \{m\} \subset dom(h)\) by merely finding the realization of \(tp(\bar{m}, m)\) in \(\mcN\), which is guarunteed by \(\mcM, \mcN\) realizing the same types. 
Our one remaining obstacle is that \(h(m_i)\) may not be equal to \(n_i\). 
This can be rectified with homogeneity. We can easily see that the function which takes \(h(m_i) \mapsto n_i\) can be extended to an automorphism. 
By simply composing this automorphism with our \(h\) from above to form \(g\) we have found our desired function. 
\end{proof}

% Possible Clarification: Moving from types over models to types over theories (p.125)

\begin{theorem}\label{theorem_vaughtian_pairs_countable_isomorphic}
A Vaughtian pair of countable models, (\(\mcN_0, \mcM_0\)) can be extended to a Vaughtian pair of isomorphic countable models.
\end{theorem}

\begin{proof}
By above, it suffices to show a vaughtian pair \((\mcN, \mcM)\) can be extended to a vaughtian pair \((\mcN^*, \mcM^*)\) where \(\mcN^*, \mcM^*\) are homogeneous and realize the same types. 

We aim to construct a chain of Vaughtian pairs indexed by the natural numbers such that each pair is an elementary extension of the previous one and their union will be homogeneous and realize the same types. 
The zeroth pair is the countable pair (\(\mcN_0, \mcM_0\)) from above. 
For \(i > 0\) we construct (\(\mcN_i, \mcM_i\)) as follows (the details of constructions in each of these stages will follow):
\begin{enumerate}
\item  If \(i\) is a multiple of 3, then (\(\mcN_i, \mcM_i\)) is an elementary extension of (\(\mcN_{i-1}, \mcM_{i-1}\)) such that every type realized in \(\mcN_{i-1}\) is realized in \(\mcM_i\)
\item  If \(i\) is one more than a mutliple of 3, then (\(\mcN_i, \mcM_i\)) is an elementary extension of (\(\mcN_{i-1}, \mcM_{i-1}\)) such that if \(\bar{a}, \bar{b}\) realize the same type in \(\mcM_{i-1}\) and \(c \in \mcM_{i-1}\), then there is a \(d \in \mcM_i\) such that \(\bar{a}c\) and \(\bar{b}d\) realize the same type in \(\mcM_i\). 
\item If \(i\) is two more than a mutliple of 3, then (\(\mcN_i, \mcM_i\)) is an elementary extension of (\(\mcN_{i-1}, \mcM_{i-1}\)) such that if \(\bar{a}, \bar{b}\) realize the same type in \(\mcN_{i-1}\) and \(c \in \mcN_{i-1}\), then there is a \(d \in \mcN_i\) such that \(\bar{a}c\) and \(\bar{b}d\) realize the same type in \(\mcN_i\).
\end{enumerate}
If we let \((\mcN, \mcM) = \bigcup_{i \in \N}(\mcN_i, \mcM_i)\), we see that every type realized in \(\mcN\) is realized in some \(\mcN_i\) and therefore in \(\mcM_{i+3}\) (as we must have done the first stage of our construction once in the interim) and thus in \(\mcM\). 
Similarly, any type realized in \(\mcM\) must be realized in some \(\mcM_i\) and in \(\mcN_i\) as well (as \(\mcM \prec \mcN\)) and thus in \(\mcN\).
Thus, \(\mcM\) and  \(\mcN\) realize the same types.  
Given a partial elementary map \(f:\mcM \to \mcM\) with finite domain, we have \(\text{dom}(f) \in \mcM_i\) for some \(i\), and stage two of our construction guaruntees that it can be extended. 
Similarly for homogeneity of \(\mcN\) by the third stage of our construction. 

At this point it would suffice to show how to construct the elementary extensions mentioned in out three stages. 
% TODO 1: Proof of 4.3.38
\textcolor{red}{Insert Proof Here \ldots}

\end{proof}

% claim?
\begin{theorem}\label{theorem_aleph_one_vaightian_pairs}
Given a Vaughtian pair of countable isomorphic models, \((\mcN_1, \mcN_0) \models T\), we can construct a model \(\mcN^*\models T\) which is of cardinality \(\aleph_1\) and has a definable subset which is countable.  
\end{theorem}

\begin{proof}
% We have \(\mcN_0 \prec \mcN_1\) so \(\iota: \mcN_0 \to \mcN_1\) is the map including \(\mcN_0\) into \(\mcN_1\). 
% Additionally, we have an isomorphism \(f: \mcN_0 \to \mcN_1\). 
% TODO 2: Proof of 4.3.34
\textcolor{red}{Insert Proof Here \ldots}
\end{proof}

\begin{theorem}\label{theorem_uncountable_vaightian_pairs}
Given a Vaughtian pair \((\mcN, \mcM)\), \(\mcN, \mcM \models T\) where \(|N| = \aleph_1\), \(\mcM\) is countable and \(T\) is \(\omega\)-stable, for every uncountable cardinal \(\kappa\), there is a model of \(T\) of cardinality \(\kappa\) with a countable definable subset.  
\end{theorem}

\begin{proof}
% TODO 2b: Proof of 4.3.41 
Let \(\phi\) be the formula defining a countable subset. 
It suffices to construct an elementary chain of models indexed by ordinals less than \(\kappa\) such that \(\mcM_\alpha \prec \mcM_{\alpha+1}\) and \(\mcM_\alpha \neq \mcM_{\alpha+1}\) but \(\phi(\mcM_\alpha) = \phi(\mcM_{\alpha+1}\). 
The model given by \(\bigcup_{\alpha<\kappa} \mcM_\alpha\) will be of size \(\kappa\) (as there are \(\kappa\) many stages, each adding fewer than \(\kappa\) many elements but at least one).
By construction, \(\phi\) will still define a countable subset. 

\textcolor{red}{Insert Proof Here \ldots} 
\end{proof}

\begin{theorem}\label{theorem_vaughtian_pairs_categoricity}
Any theory which has Vaughtian pairs cannot be \(\kappa\)-categorical for any uncountable \(\kappa\).
\end{theorem}

\begin{proof}
As we stated above, we can also create a model of \(T\) of size \(\kappa\) where all infinite definable sets are of size \(\kappa\). 
Such a model wouldn't be isomorphic to the model we just created which has a countable definable subset. 
\end{proof}


\section{Proof of Morley's Categoricity Theorem}
At this point we've made good progress towards the proof of Morley's Categoricity theorem.
It suffices to show that uncountable categoricity (for any uncountable cardinal) is equivalent to \(\omega\)-stability and lacking Vaughtian pairs.
As this property doesn't depend on the uncountable cardinality, this suffices. 

We've already shown that \(\omega\)-stability and lacking Vaughtian pairs are necessary for uncountable categoricity.
To show these conditions are also sufficient we will define a notion of ``dimension'' for elementary extensions.
\(\osmega\)-stability already tells us that we can view any two models as elementary extensions of the same prime model by Theorem %.
Because of their cardinalities, any equinumerous models will have the same dimension over this prime model.
Having the same dimension over this prime model will suffice to show that these two models are isomorphic.
This will complete the proof. 

\subsection{Strong Minimality and Vaughtian Pairs}

Now, assume we have a theory which is \omst and has no Vaughtian-pairs. 
As it is \omst it has a prime model.
By \(\omega\)-stability, that prime model has a minimal formula. 
For our proof of Morley's categoricity theorem it will be important that this minimal formula is strongly minimal, so that we can use dimension to find an isomorphism as described above. 
A theory lacking Vaughtian pairs will have the property that minimal formula are strongly minimal. 

\begin{theorem}\label{theorem_minimal_vaughtian_pair}
If \(T\) has no Vaughtian pairs and \(\phi\) is a minimal formula for \(\mcM \models T\) then \(\phi\) is strongly-minimal. 
\end{theorem}

\begin{proof}
If \(\phi\) is not strongly minimal, there is \(\mcM \prec \mcN\), \(\bar{b} \in \mcN\) and \(\psi(\bar{x}, \bar{b})\) such that \(\psi(\mcN, \bar{b})\) is infinite and co-infinite. 
We see that for every natural number \(n\), we know \(\mcN \models \E \bar{b} (\theta_{\psi, \bar{b}, n} \land \theta_{\neg \psi, \bar{b}, n})\) where \(\theta_{\psi, \bar{b},n}\) asserts there are at least \(n\) vectors \(\bar{x}\) for which \(\psi(\bar{x}, \bar{b})\) holds. 
As \(\mcM \prec \mcN\), we know that each of these formulae hold in \(\mcM\) as well. 
We can use this fact, together with compactness to show that \(T\) has a Vaughtian pair.
Note that for each \(\bar{b}\) either \(\psi(\mcM, \bar{b})\) is infinite or it is co-infinite. 
One of these cases must happen for arbitrarily large \(n\).  % 10-1 in M?
Assume without loos of generality that \(\psi(\mcM, \bar{b})\) is co-infinite and arbitrarily large. 
We will represent model pairs as in the previous section with an additional predicate \(U\) defining an elementary submodel. 
We will show that the following set of formulae is satisfiable: 
\(\E x (\neg U(x))\), \(U(\bar{w})\), \(\{\theta_{\psi, \bar{w}, n}\}_{n\in \N}\), \(\psi(\bar{v}, \bar{w}) \to U(\bar{v})\). 
% Possible Clarification: We took the \(\psi(\mcM, \bar{n})\) co-infinite so we could have a proper elementary submodel. Elaborate more on this. 
When taking finite subsets of these formulae, it suffices to take \(n\) sufficiently large and let \(\bar{w}\) be the \(\bar{b}\) which we know exists.
In this model, \(\{\bar{v}\mid\psi(\bar{w}, \bar{v})\}\) defines an infinite subset, for which \(U(\bar{v})\) holds, in other words, a Vaughtian pair. % Explain more

With these past few lemmata we can conclude that any theory which lack Vaughtian pairs and is \omst has a strongly minimal formula with parameters from a countable prime model. 
We will develop the theory of dimensions of elementary extensions and see that any two extensions of cardinality \(\kappa > \aleph_0\) will have dimension \(\kappa\) over the prime model.
From this fact, we will be able to show they are isomorphic, providing us, finally, with \(\kappa\) categoricity and completing the proof of Morley's categoricity theorem. 
\end{proof}

\subsection{Algebraic Closure and Dimension}

\begin{definition}\label{definition_algebraic_closure}
Given a model \(\mcM\) and an \(A \subseteq M\), we define the algebraic closure of \(A\), denoted \(\acl(A)\) to be all elements of \(M\) which are in a finite set definable by an \(\mcL_A\)-formula. 
(Note that by quantifier elimination over algebraically closed fields, this notion of ``algebraic closure'' agrees with the familiar one.)
\end{definition}

The algebraic closure has the nice properties that \(\acl(\acl(A)) = \acl(A)\), that \(A \subseteq B \implies \acl(A) \subseteq \acl(B)\) and that if \(a \in \acl(A)\) then there is a finite subset \(A_0 \subseteq A\) such that \(a \in \acl(A_0)\).
%This almost gives us that algebraic closure is a pregeometry. % Definition?
We would like the exchange property to be true, namely that if \(a, b \notin \acl(A), a \in \acl(A \cup \{b\})\) then \(b \in \acl(A \cup \{a\})\) 
This is only true when we take \(A\) to be a subset of some strongly minimal set.

\begin{theorem}\label{theorem_exchange}
Let \(S\) be a strongly minimal set. If \(A \subseteq S, a, b \in S\) and \(a, b \notin \acl(A)\) but \(a \in \acl(A \cup \{b\})\) then \(b \in \acl(A \cup \{a\})\) 
\end{theorem}

% Possible Clarification: Limiting \acl to a set D

\begin{proof}
% Possible Improvement: Better presentation/intuition including pairs definable by \phi(x, y)
As \(a \in \acl(A \cup \{b\})\), we have \(\mcM \models \phi(a, b)\) for some \(\mcL_A\)-formula 
% Possible Clarification: Make sure notation of \mcL_A is familiar/defined
\(\phi\) where \(\{x \in S \mid \phi(x, b)\}\) contains \(n\) elements. 
Consider the set defined by \(\psi(y)\), where \(\psi(y)\) asserts that \(\{x \in S \mid \phi(x, y)\}\) contains exactly \(n\) elements. 
% Book mentions \psi(\mcM) is cofinite, why?
We would very much like for \(X = \{y \in S \mid \psi(y) \land \phi(a, y)\}\) to be finite, making \(b \in \acl(A \cup \{a\})\). 
We will show this must be the case by contradiction.
If this fails and \(X\) is infinite, then \(X\) is cofinite. 
Specifically, \(\{y \in S \mid \neg(\psi(y) \land \phi(a, y))\}\) contains \(m\) elements.  
We let \(\theta(x)\) assert that \(\{y \in S \mid \neg(\psi(y) \land \phi(x, y))\}\) contains exactly \(m\) elements.
\(\theta\) also defines a cofinite set, as otherwise \(a \in \acl(A)\).   
Let \(a_1, \ldots, a_{n+1}\) be such that for each \(i\), \(\theta(a_i)\) holds.
As \(\theta(a_i)\) holds, each set \(\{y \in S \mid \psi(y) \land \phi(a_i, y)\}\) is cofinite, as is their intersection. 
For any element, \(b'\) in the intersection, we have \(\psi(b')\) and \(a_i \in \{x \in S \mid \phi(x, b')\}\) for each \(a_i\) by construction.
This is a contradiction as \(\psi(b')\) asserts there are only \(n\) such elements.  
\end{proof}

\begin{definition}\label{definition_independence}
As above let \(D\) be a minimal set. \(A \subseteq D\) is independent if for all \(a \in A\), \(a \notin \acl(A \setminus \{a\})\). 
% Possible Addition: Define ``independent over''. \(A \subseteq D\) is independent over \(C \subseteq D\) if, for all \(a \in A\), \(a \notin \acl(C \cup (A \setminus \{a\}))\). 
\(B\) is a basis for \(A\) when \(\acl(A) = \acl(B)\) and \(B\) is independent.
We will show that all bases of a set \(A\)  have the same cardinality, which we will call the dimension of \(A\).
\end{definition}

\begin{lemma}\label{lemma_dimension}
Given two disjoint independent sets, \(S, T\), 
%whose union is also independent?
 if \(S \cup \{s\}\) is independent, there is an element \(t \in T\) such that \((S \cup \{s\}) \cup (T \setminus \{t\})\) is independent and has the same algebraic closure as \(S \cup T\). 
\end{lemma}

\begin{proof}
Let \(T_0 \subseteq T\) be of the smallest cardinality such that \(s \in \acl(S \cup T_0)\). 
Clearly, \(T_0\) is non-empty, so choose \(t \in T_0\). 
\(s \in \acl(S \cup T_0)\) and by exchange \(t \in \acl(S \cup \{s\} \cup T_0 \setminus \{t\}) \subseteq \acl(S \cup \{s\} \cup T \setminus \{t\})\).
We can conclude \(\acl(S \cup \{s\} \cup T \setminus \{t\}) = \acl(S \cup T)\) as the former contains \(S \cup T\) and is algebraically closed and \(s \in \acl(S \cup T)\). 
It suffices to show \(S \cup \{s\} \cup T \setminus \{t\}\) is independent. 
We already know that \(S \cup T\) and therefore \(S \cup T \setminus \{t\}\) are independent. 
It suffices to show (by exchange) that \(s \notin \acl(S \cup T \setminus \{t\})\).
If this were the case, by exchange we would have \(t \in \acl(S \cup T \setminus \{t\})\) as that set is equal to \(\acl(S \cup \{s\} \cup T \setminus\{t\})\).
But this would contradict the independence of \(S \cup T\). 
\end{proof}

\begin{theorem}\label{theorem_dimension}
The dimension of a set (which is a subset of a strongly minimal set) is well defined.
\end{theorem}

\begin{proof}
To prove the uniqueness of dimension, it suffices to show that if \(A\) and \(B\) are bases for \(|C|\), then \(|A| \leq |B|\). 
The first case we show is when \(|B|\) is finite, which we will show by contradiction.  
Let \(S_0 = \emptyset\) and \(T_0 = A\). We iteratively expand \(S_i\) by elements of \(B = \{b_1, \ldots, b_n\}\) and remove elements of \(T_i\) preserving the algebraic closure of their union. 
Ultimately, we see that \(\acl(A) = \acl(S_i \cup T_i) = \acl(B \cup (A \setminus \{a_1, \ldots, a_n\})) = \acl(C)\)
If \(A \setminus \{a_1, \ldots, a_n\}\) is non-empty and contains some element \(a\), then \(a \notin \acl(B)\) as \(B \cup (A \setminus \{a_1, \ldots, a_n\})\) is independent. 
But \(a \in \acl(A) = \acl(C)\) so \(\acl(B)\) cannot equal \(\acl(C)\) and cannot be a basis. 

In the case where \(B\) is infinite, \(|B| = \kappa\), we note that there are \(\kappa\) many formula with parameters from \(B\) and at most \(\kappa\) many elements of \(\acl(B) = \acl(C) \supset C\). 
It is impossible to take mare than \(\kappa\) many elements of \(C\), so there cannot be a subset \(A\) of \(C\) with \(|A| > |B|\) so there cannot be such a basis. 
\end{proof}

\begin{theorem}\label{partial_elementary_bijection}
Let \(\mcM, \mcN_1, \mcN_1 \models T\) and \(\mcM \prec \mcN_1, \mcM \prec \mcN_2\) and \(\phi(v)\) be a strongly minimal formula with parameters from \(M\).
If \(\phi(\mcN_1)\) has the same dimension as \(\phi(\mcN_2)\), then there is a partial elementary map \(f: \phi(\mcN_1) \to \phi(\mcN_2)\) which is a bijection. 
\end{theorem}

\begin{proof}
Let \(B_1\) be a basis for \(\phi(\mcN_1)\) and \(B_2\) be a basis for \(\phi(\mcN_2)\).
As \(|B_1| = |B_2|\), we have a bijection \(f: B_1 \to B_2\), which we claim is partial elementary. 

% check this
Otherwise, for some \(\psi\) an \(\mcL_\mcM\)-formula and \(\bar{a} \in B_1\) we have \(\mcN_1 \models \psi(\bar{a})\) and \(\mcN_2 \not\models \psi(f(\bar{a}))\). 

% We show that this doesn't happen by induction on length of \(\bar{a}\).
Begin with the case \(\mcN_1 \models \psi(a)\) and \(\mcN_2 \not \models \psi(f(a))\). 
\(a\) is not in any finite definable subset as \(B_1\) is independent and \(a \notin \acl(B_1 \setminus\{a\})\). 
So \(\mcN_1 \models |\{x \mid \phi \land \neg \psi(x)\}| = n\) so \(\mcM \models |\{x \mid \phi \land \neg \psi(x)\}| = n\)
and \(\mcN_2 \models |\{x \mid \phi \land \neg \psi(x)\}| = n\).
\(f(a)\) is not in any finite definable subset as \(B_2\) is independent and \(f(a) \notin \acl(B_2 \setminus\{f(a)\})\). 
So \(\mcN_2 \models \psi(f(a)\) as desired. 

This argument easily generalizes to all \(\bar{a}\). 

Now that we know \(f\) is partial elementary, we consider extensions of \(f\) which are partial elementary. 
By Zorn's Lemma, it suffices to show that any partial elementary \(g\) extending \(f\) which doesn't have all of \(\mcN_1\) as its domain is not maximal (applying this to its inverse tells us it must be surjective as well).
Let \(b \notin \text{dom}(g)\). \(b\) is algebraic over \(\text{dom}(g)\). %why
So, it must be that the type of \(b\) over \(\text{dom}(g)\) is isolated by some \(\phi(v, \bar{d})\).
As \(g\) is partial elementary, \(\phi(v, g(\bar{d}))\) holds for some \(c \in \phi(\mcN_1)\).  % something about still isolating the type?
We must have \(tp(b/\text{dom}(g)) = tp (c/\text{rng}(g))\) so \(g \cup \{(b, c)\}\) must be partial elementary as desired. 

Thus, our maximal element guarunteed by Zorn's Lemma is a bijective partial elementary map between \(\phi(\mcN_1)\) and  \(\phi(\mcN_2)\)
\end{proof}

\subsection{Proof of Morley's Categoricity Theorem}

We can now present a proof of Morley's Categoricity Theorem, Theorem \ref{theorem_morleys_categoricity}.

\begin{proof}
In the previous sections we have succeeded in showing theories categorical in any uncountable cardinal are \omst and lack Vaughtian pairs. 
We now show the converse: two models \(\mcM, \mcN\)  of a complete \omst theory, \(T\), in a countable language lacking Vaughtian pairs we can show \(\mcM \isom \mcN\). 
As \(T\) is \omst it has a prime model, \(\mcM_0\), in which we have a minimal formula \(\phi\) (again as \(T\) is \(\omega\)-stable).
Moreover, as \(T\) has no Vaughtian pairs, \(\phi\) is strongly minimal. 
We begin by constructing a partial elementary map \(f:\phi(\mcM) \to \phi(\mcN)\) which is a bijection, which exists as per Theorem \ref{partial_elementary_bijection}.
Both \(\phi(\mcM), \phi(\mcN)\) have cardinality \(\kappa\) and therefore dimension \(\kappa\). 
As \(\phi(\mcM)\) is infinite and definable and \(T\) has no Vaughtian pairs and is \(\omega\)-stable, \(\mcM\) is prime over \(\phi(\mcM)\).
We can extend \(f\) to an \(f'\) which is partial elementary and has domain \(\mcM\), \(\mcM\) is prime over \(\phi(\mcM)\).
The range of \(f'\) is an elementary submodel of \(\mcN\) (as it's domain is isomorphic to its range) containing \(\phi(\mcN)\). 
If the range of \(f'\) isn't all of \(\mcN\), then \((\mcN, f'(\mcM))\) forms a vaughtian pair (\(\phi(\mcN)\) is infinite and includes no element of \(\mcN \setminus f'(\mcM)\)).  
So \(f'\) is injective (as it is partial elementary) and surjective (as we have no Vaughtian pairs) and is our desired isomorphism. 
\end{proof}

% Possible Addition: general review of proof.


\section{Further Questions}
We begin by restating Morley's categoricity theorem in a way which will allow us to generalize or extend it naturally.

\begin{definition}\label{definition_number_models}
If \(T\) be a complete first order theory \(I(T, \kappa)\) is the number of models of \(T\) of size \(\kappa\) up to isomorphism.
\end{definition}

Morley's categoricity theorem stated that if \(T\) is a theory in a countable language and \(\kappa, \lambda > \aleph_0\) then \(I(T, \kappa) = 1 \iff I(T, \lambda) = 1\). 

We'd like to ask some more general questions, some of which we will be able to answer fully in this section and some of which will only address in a limited sense. 
In the following, we assume that \(T\) is a theory in a countable language. 

\begin{question}\label{question_countable_models_uncountably_categorical}
If \(\lambda > \aleph_0, I(T, \lambda) = 1\) what could \(I(T, \aleph_0)\) be? (could it be \(2^{\aleph_0}\)?)
\end{question}

\begin{question}\label{question_los_conjecture_uncountable_languages}
Is there a natural analogue to Morley's Categoricity Theorem when we no longer assume that \(T\) is a theory in a countable language?
\end{question}

\begin{question}\label{question_morleys_conjecture}
Is it possible to have \(\kappa > \lambda > \aleph_0\) but \(I(T, \kappa) < I(T, \lambda)\)?
\end{question}

An easy extension of our argument for Morley's Categoricity Theorem shows us that for uncountably categorical \(T\), \(I(T, \aleph_0)\leq \aleph_0\).
Up to isomorphism, these models are determined by their dimension, which must not be greater than \(\aleph_0\) leaving only countably many possibilities. 
This is an answer to question \ref{question_countable_models_uncountably_categorical}.

To adress Question \ref{question_los_conjecture_uncountable_languages}, \L o\'s' Conjecture for uncountable languages (of cardinality \(\kappa\)) states that if \(\lambda, \lambda' >\kappa\) then \(I(T, \lambda) = 1 \iff I(T, \lambda') = 1\). 
A complete proof of this theorem is again beyond the scope of this paper, but we reference \cite{shelahUncountable}.

We can answer question \ref{question_morleys_conjecture} in the negative. This is known as Morley's Conjecture. 
We direct the interested reader to Hart\cite{hart}.


\appendix
\section{Appendix}
\subsection{A Proof of Ramsey's Theorem}

\begin{theorem}\label{ramseys_theorem}
If \(S\) is an infinite set whose subsets of size \(n\) are colored with a finite set of colors, there exists a \(T \subseteq S\) for which every subset of size \(n\) of \(T\) has the same color.
\end{theorem}

\begin{proof}
Assume without loss of generality that \(S = \N\).
When \(n=1\), Ramsey's Theorem states we cannot partition \(\N\) into finitely many finite sets, which is clear.
Now, we will induct on \(n\).
We will construct a \(T \subseteq \N\) with the following property: the color of every subset of size \(n\) of \(T\) is determined by its least element.
Once we have that set \(T\), we can color the elements \(t\) of \(T\) by the color of all \(n\)-element sets with \(t\) as their element.
By the pidgeonhole principle, there will be one color which appears infinitely often.
If we take \(T' \subseteq T\) consisting of those elements, every \(n\) element subset of \(T'\) will have a least element and by construction, that subset will have the color we chose.
\(T'\) will be the desired set.

We now construct a \(T \subseteq \N\) such that the color of every subset of size \(n\) of \(T\) is determined by its least element.
We can color subsets, \(A\) of size \(n-1\) of \(\N \setminus \{0\}\) by the color assigned to \(A \cup \{0\}\).
By the inductive hypothesis, there is \(T_0 \subseteq \N \setminus \{0\}\) such that every \(n\)-element subset of \(T_0 \cup \{0\}\) with 0 as its least element is the same color.
We now take the least element \(t_0 \in T_0\) and similarly apply the inductive hypothesis there to get an infinite \(T_1 \subseteq T_0\) such that every subset of \(T_1 \cup \{0, t_0\}\) with \(t_0\) as i\
ts least element has the same color.
As \(T_1 \subseteq T_0\) the same is still true for subsets with \(0\) as their least element.
Continuing inductively, we take \(T = \bigcap\limits_{i \in \N}T_i = \{0, t_0, t_1, \ldots\}\) which will have this property.
\end{proof}




\begin{thebibliography} \small
%
\bibitem[Mar]{mar} David Marker, Model Theory: An Introduction, 2002

\bibitem[Sh1]{shelahUncountable} Shelah, Saharon. Categoricity of uncountable theories -- Proc Tarski Symposium (Univ. of California, Berkeley, Calif., 1971) (1974) 187--203

\bibitem[Sh2]{shelahUnsuperstable} Shelah, Saharon. "Why there are many nonisomorphic models for unsuperstable theories." Proceedings of the International Congress of Mathematicians, Vancouver. 1974.

\bibitem[Har]{hart}Hart, Bradd. A Proof of Morley's Conjecture. Journal of Symbolic Logic 54 (1989), no. 4, 1346--1358. http://projecteuclid.org/euclid.jsl/1183743103.
%
\end{thebibliography}
\end{document}

