\documentclass[12pt]{article}
\usepackage[centertags]{amsmath}
\usepackage{amsfonts,amssymb,amsthm,color}

%
%MY SHORTCUTS
%
\newcommand{\mcE}{\mathcal{E}}
\newcommand{\mcC}{\mathcal{C}}
\newcommand{\mcA}{\mathcal{A}}
\newcommand{\mcB}{\mathcal{B}}
\newcommand{\mcM}{\mathcal{M}}
\newcommand{\mcN}{\mathcal{N}}
\newcommand{\mcL}{\mathcal{L}}
\newcommand{\R}{\mathbb{R}}
\newcommand{\C}{\mathbb{C}}
\newcommand{\Z}{\mathbb{Z}}
\newcommand{\N}{\mathbb{N}}
\newcommand{\Q}{\mathbb{Q}}
\newcommand{\satisfies}{\models}
\newcommand{\A}{\forall}
\newcommand{\E}{\exists}
\newcommand{\isom}{\cong}
\newcommand{\diag}{\text{Diag}}
\newcommand{\Th}{\text{Th}}
\newcommand{\eldiag}{\text{Diag}_\text{el}}
\newcommand{\qe}{quantifier elimination }
\newcommand{\qf}{quantifier-free }
\newcommand{\omst}{\(\omega\)-stable }
\newcommand{\acl}{\text{acl}}

%
%PAGE REFORMATTING
%
\setlength{\topmargin}{0 in}
\setlength{\headheight}{0.25 in}
\setlength{\oddsidemargin}{0.5 in}
\setlength{\textwidth}{6 in}
\setlength{\textheight}{8.5 in}
\setlength{\headsep}{0.25 in}
\newcommand{\singlespaced}{\renewcommand{\baselinestretch}{1}\normalfont}
\newcommand{\doublespaced}{\renewcommand{\baselinestretch}{2}\normalfont}
\singlespaced
\doublespaced

%
%THEOREM STYLES
%
\newtheorem{theorem}{Theorem}[section]
\newtheorem{lemma}[theorem]{Lemma}
\newtheorem{question}[theorem]{Question}

\theoremstyle{definition}
\newtheorem{definition}[theorem]{Definition}
\newtheorem{notation}[theorem]{Notation}
\newtheorem{remark}[theorem]{Remark}
\newtheorem{exercise}[theorem]{Exercise}
\newtheorem{example}[theorem]{Example}
\newtheorem{convention}[theorem]{Convention}

\theoremstyle{definition}
\newtheorem{defn}{Definition}[section]
\newtheorem{fig}{Figure}[section]
%
\theoremstyle{plain}
\newtheorem{res}[defn]{Result}
\newtheorem{thm}[defn]{Theorem}
\newtheorem{cor}[defn]{Corollary}
\newtheorem{prop}[defn]{Proposition}
\newtheorem{lem}[defn]{Lemma}
%
\theoremstyle{remark}
\newtheorem*{case}{Case}
\newtheorem{rem}[defn]{Remark}
\newtheorem{ex}[defn]{Example}
\newtheorem{exs}[defn]{Examples}

%
%EQ. NUMBERING
%
\numberwithin{equation}{section}
\renewcommand{\theequation}{\thesection.\arabic{equation}}
%
\def\thetitle{A Proof of Morley's Categoricity Theorem}
\def\theauthor{Adam Freilich}
\def\theadvisor{Advisor: Dr. Henry Towsner}
\def\theyear{2014}
%
\pagenumbering{roman}

\begin{document}
\large\newlength{\oldparskip}\setlength\oldparskip{\parskip}\parskip=.3in
\thispagestyle{empty} \doublespaced
\begin{center}
\vspace*{\fill} \thetitle

\theauthor


A THESIS

in

Mathematics

\singlespaced

\large Presented to the Faculties of the University of
Pennsylvania in Partial
 Fulfillment of the Requirements for the Degree of Master of
 Arts
 
\doublespaced

\large
\theyear
\end{center}

\singlespaced

\noindent\makebox[0in][l]{\rule[2ex]{3in}{.3mm}}
 Henry Towsner \\  \hspace*{.5mm} Supervisor of Thesis

\noindent\makebox[0in][l]{\rule[2ex]{3in}{.3mm}}  David Harbater \\  \hspace*{.5mm} Graduate Group Chairperson \vspace*{\fill}

\normalsize\parskip=\oldparskip

\doublespaced


%\include{Abstract}
%\vspace*{\fill} \begin{center} {\it
Dedicated to the winners and the losers.}
\end{center}
\vspace*{\fill}



\pagenumbering{arabic}
\pagestyle{myheadings} \markright{}

\section{Basic Notions}
\subsection{Defining Categoricity}

We begin by recalling the defition of categoricity in a cardinal \(\lambda\):

\begin{definition}\label{def_categoricity}
A theory \(T\) is \(\lambda\)-categorical if \(\mcM \isom \mcN\) whenever \(\mcM, \mcN \models T\) and \(|M| = |N| = \lambda\).
Alternatively, \(T\) is \(\lambda\)-categorical if it has a single model of cardinality \(\lambda\) up to isomorphism. 
\end{definition}

\begin{example}\label{example_categoricity_sets}
Consider the theory of infinite sets in the pure language of equality. 
This theory is \(\lambda\)-categorical for all \(\lambda\).
Consider two models \(\mcM, \mcN\) such that \(|M|=|N|=\lambda\). 
By hypothesis we have a bijection \(f: M \to N\). 
\(f\) is an isomorphism of sets, so \(M \isom N\).
\end{example}

Recall that a theory, \(T\), is complete if for all sentences \(\phi\) we have either \(T \models \phi\) or \(T \models \neg \phi\). 
Note that this theory must also be complete. 
More generally, Vaught's test tells us that theories without finite models that are categorical in some infinite cardinal \(\kappa\) are complete. 
(If such a theory were not complete, it would have models of cardinality of \(\lambda\) disagreeing on some \(\phi\) which couldn't then be isomorphic.)

\begin{example}\label{example_categoricity_equiv}
Consider the theory of an equivalence relation with exactly two infinite classes. 
This theory is ``\(\aleph_0\)-categorical''.
We will often call an \(\aleph_0\)-categorical theory ``countably categorical''.
(These terms are synonomyous.)  
Consider two countable models \(\mcM, \mcN\) of this theory. 
There must be countably many members of each equivalence class in any countable model. 
We can present a bijection \(f_1\) between members of the first equivalence class in  \(\mcM\) and those in \(\mcN\) and a similar bijection \(f_2\) for the second equivalence class. 
We see that \(f = f_1 \cup f_2\) is a bijection and actually an isomorphism. 

On the other hand, if \(\lambda > \aleph_0\) is an infinite cardinal, then this theory isn't \(\lambda\)-categorical. 
It suffices to note that we can have the two non-isomorphic cases: where the first equivalence class has cardinality \(\lambda\) and the second is countable and where both classes are of cardinality \(\lambda\). 
Vaught's theorem tells us that even though this theory isn't \(\aleph_1\) categorical, it is still complete. 
\end{example}

\begin{example}\label{example_categoricity_integers}
Consider the theory of (\(\Z, s\)), the integers with successor. 
Included in this theory is the fact that every element has a unique successor and predecessor and that there are no cycles (i.e. no number is its own successor, or the sucessor of its successor, etc.). 
Using this fact, we observe that each model of this theory must consist of some number of \(\Z\)-chains (i.e. submodels isomorphic to \(\Z\) itself). 
For uncountable cardinals \(\kappa\), we know that, up to isomorphism, the only model of cardinality \(\kappa\) consists of \(\kappa\) many \(\Z\)-chains. 
Making this theory categorical for all uncountable cardinals and complete. 
That being said, there are countably many non-isomorphic countable models, which are the familiar \(\Z\), \(n\) copies of \(\Z\) for any natural \(n\) or countably many copies of \(\Z\). 
This theory is not countably categorical, but it is \(\kappa\) categorical for all uncountable \(\kappa\). 
\end{example}

\begin{example}\label{example_categoricity_sequence}
We consider the language \(\mcL = \{<, c_1, c_2, c_3, \ldots\}\) which has countably many constant symbols indexed by the natural numbers. 
In our theory, \(T\), \(<\) is a dense linear order (this is a first order property) and \(c_i\) is a strictly increasing sequence. 

We show that \(T\) is complete.
Consider a sentence \(\phi\), we will show that either \(T \models \phi\) or \(T \models \neg \phi\). 
\(\phi\) can only include finitely many of the constant symbols, so assume without loss of generality that \(\phi\) is an \(\mcL_n\) formula where \(\mcL_n = \{<, c_1, \ldots, c_n\}\).
It would suffice to show that \(T_0 = \{\phi \in T \mid \phi \text{ is an } \mcL_n \text{ formula}\}\) is countably categorical and therefore complete. 
If \(\mcN, \mcM\) are countable models of this theory, we can build an isomorphism between them with the ``back and forth'' method. 
We define an isomorphism as a union of partial functions (``partial isomorphisms'') respecting the order. 
Let \(f_0\) be the function mapping \(c_i\) in \(\mcM\) to \(c_i\) in \(\mcN\) and undefined everywhere else. 
We can enumerate the elements of \(\mcM, \mcN\) by the natural numbers making \(M = \{m_1, \ldots\}, N = \{n_1, \ldots\}\). 
We will define \(f_i\) inductively for \(i > 0\). 
If \(i\) is even, we extend \(f_{i-1}\) to include \(m_{i/2}\) in its domain if \(m_{i/2} \notin \text{dom}(f_{i-1})\) and otherwise, let \(f_i = f_{i-1}\).
Note that as \(\text{dom}(f)\) is finite and doesn't include \(m_{i/2}\) there is a greatest element in \(\text{dom}(f)\) less than \(m_{i/2}\) and a least element greater than it, call them \(p\) and \(q\) respectively.
We can map \(m_{i/2}\) to an arbitrary element in \(N\) which is greater than \(f(p)\) but less than \(f(q)\). 
For odd \(i\) we similarly extend the range of \(f_{i-1}\). 
Letting \(f = \bigcup_{i \in \N}f(i)\) we will find that we have an isomorphism. 
Using the fact that dense linear orders have quantifier elimination \cite{mar} % specific in reference
makes verifying this fact rather straightforward, and the details are left to the reader.
 
This theory is never categorical.
Consider two models of this theory, both resemble \(\Q\) in their underlying set and order.
In \(\mcM\) we let \(c_i = i\), making our increasing sequence the natural numbers. 
In a second model, \(\mcN\), we let \(c_i = 1 - \frac{1}{i}\).
It suffices to note that any isomorphism \(f: \mcN \to \mcM\) must send \(1\) to some element of \(\Q\).
Inevitably \(f(1)\) will be less than \(n\) for some natural number \(n\) and therefore, \(\mcN \models f(1) < c_i\) for some \(i\).
As \(\mcM \models c_i < 1\) for all \(i\), \(f\) cannot be an isomorphism. 
We've shown two countable models of \(T\) which aren't isomorphic, showing that \(T\) isn't countably categorical. 
Taking these two models, and adding \(\kappa\) many elements greater than every element of \(\Q\) will give us two non-isomorphic models of cardinality \(\kappa\) by a similar argument. 

\(T\) is complete but never categorical. 
\end{example}

Morley's categoricity theorem will state that these are the only four cases.
That is, a theory \(T\) is never categorical, always categorical, \(\kappa\) categorical for all \(\kappa > \aleph_0\) or countably categorical but categorical in no uncountable cardinal.  
That theorem will be the main focus of this thesis. 

\subsection{Stating Morley's Categoricity Theorem}

\begin{theorem}\label{theorem_morleys_categoricity}
If a first order theory in a countable language is categorical in some uncountable cardinal, it is categorical in all uncountable cardinals. 
\end{theorem}

I present a very general sketch of the proof of this theorem which I will flesh out throughout the rest of this work. 

We will define two properties of theories: being \(\omega\)-stable and having Vaughtian pairs. 
It turns out that lacking \(\omega\)-stability or having Vaughtian pairs will prevent uncountable categoricity, so all uncountably categorical theories lack Vaughtian pairs and are \(\omega\)-stable. 

We then observe that these two conditions, lacking Vaughtian pairs and being \(\omega\)-stable, are sufficient to show categoricity in all uncountable cardinals. 
Hence, the categoricity of a theory in some uncountable cardinal must imply \(\omega\)-stability and the absence of Vaughtian pairs, which, in turn, imply categoricity in all other uncountable cardinals.

Once I complete this proof, I will seek to answer some related questions about how many non-isomorphic models there can be of a theory. 

%I will elaborate more on issues of stability in general, morley rank and forking.  

\section{\(\omega\)-stability}
In this section, our goal is to define stability (and \(\omega\)-stability in particular) and show that theories which aren't \(\omega\)-stable cannot be uncountably categorical for any uncountable cardinality. 

In order to define stability we need to recall a few definitions and facts about types. 

Let \(\mcM\) be an \(\mcL\)-structure and \(A \subset M\). 
It will often be natural to want to add constant symbols for every element of \(A\) to our language. 
We will call the resulting language \(\mcL_A\). \(\mcM\) has a natural definition as an \(\mcL_A\)-structure. 
In general, we call the set of \(\mcL_A\)-sentences true in \(\mcM\) ``\(\Th_A(\mcM)\)'' and, in the special case where \(A = M\) we call it \(\eldiag(\mcM)\) or the ``elementary diagram of \(\mcM\)''. 

\subsection{Defining Types}

We are now prepared to define types. 

\textbf{Definition.} Let \(A, \mcM, \mcL\) as above. 
An \(n\)-type, \(p\), is a set of \(\mcL_A\)-formulae in \(n\) free variables (\(v_1, \ldots, v_n\)) for which \(p \cup \Th_A(\mcM)\) is satisfiable.
We say that \(p\) is complete if, for every \(\mcL_A\)-formula \(\phi(v_1, \ldots, v_n)\), either \(p \ni \phi\) or \(p \ni \neg \phi\). 
We call the set of all complete \(n\) types (containing \(\mcL_A\) formulae satisfiable with \(\Th_A(\mcM)\)) ``\(S_n^\mcM(A)\)''. 

Let's consider a few simple examples of types when  \(\mcM = (\N,s)\), the natural numbers with successor and \(A = \{0\}\) (we've added a constant symbol for 0). 
One example or an incomplete type is when we take \(p = \{v_1 \neq 0\}\). 
As \(\N \models \phi(2)\) for all \(\phi \in p\), \(p \cup \Th_\emptyset(\N)\) must be satisfiable, and \(p\) is a type. 
Still, neither \(v_1 = s(s(s(0)))\) or its negation is in \(p\) so it is not complete. 

There is one clear source of complete types. 
Every element of \(\N\) has an associated complete type. 
Let \(p(v) = \{\phi(v) \mid \N \models \phi(1)\}\). 
As \(\N\) is a model realizing \(p \cup \Th_\emptyset(\N)\) when \(v_1\) is interpreted as 1, this set of formulae is a type. 
As every formula in one free variable is either true or false for 1, this type is complete. 
We say \(p\) is realized by 1. 
More generally, we say \(\bar{x} \in M^n\) realizes an \(n\)-type \(p\) if \(\mcM \models \phi(\bar{x})\) for all \(\phi \in p\).  

This, though, is not the only source of types. 
If we take \(p = \{v_1 \neq 0, v_1 \neq s^1(0), v_1 \neq s^2(0), \ldots\}\) we get a set of \(\mcL_{\{0\}}\)-formulae in one free variable. 
We note that \(p \cup \Th_{\{0\}}(\N)\) is satisfiable (as it is finitely satisfiable). 
We can extend \{p\} to a complete type, but that type cannot be realized by any natural number, since an element realizing \(p\) is greater than every natural number by construction. 
It isn't always the case that a type is realized. 

\subsection{Defining Stability}

We are now ready to define stability: 

\textbf{Definition.} Let \(T\) be a complete theory in a countable language and \(\kappa\) be an infinite cardinal. 
\(T\) is \(\kappa\)-stable if \(|S_n^\mcM(A)| = \kappa\) whenever \(\mcM \models T\) and \(|A| = \kappa\). 
Note: \(\aleph_0\)-stable theories are called \(\omega\)-stable for historical reasons. 

First, we will show that the theory of dense linear orders without endpoints is not \(\omega\)-stable. 
\(\mcM = (\R, <)\) is a model of this theory and \(A = \Q\) is a subset of size \(\aleph_0\). 
Still, every real number realizes a different type in \(S_1^\R(\Q)\) (as every two distinct real numbers have a rational number between them). 
So there must be at least as many types as there are real numbers; there are uncountably many types.  

Recall that we mentioned in the proof outline of Morley's Categoricity theorem that all uncountably categorical theories are \(\omega\)-stable. 
We mentioned the theory of infinite sets and the theory of (\(\Z, s\)) as uncountably categorical theories. 
Let's observe that both of these theories are \(\omega\)-stable. 

\textbf{Theorem.} The theory of infinite sets is \omst. 

First, consider the complete 1-types, \(p(v_1)\). \(p(v_1)\) must assert either that \(v_1 = a\) for exactly \(a \in A\) or that \(v_1 \neq a\) for all \(a \in A\). 
Moreover, as the theory of infinite sets admits \qe all formulae are equivalent to boolean combinations of such formulae. 
We have now identified all 1-types (there is one for each element in \(A\) and one more for when \(v_1\) isn't in \(A\)). 
This can be easily extended to \(n\)-types for \(n > 1\)

\textbf{Theorem.} The theory of \((\Z, s)\) is \omst. 

We will again consider 1-types and leave the general argument to the reader. 
We can assume, WLOG, that no two elements of \(A\) are in the same \(\Z\)-chain. 
Our 1-type either asserts that \(v_1\) is in the same \(\Z\)-chain as a single element of \(A\) and determines its position relative to that element, or asserts that \(v_1\) isn't in the same \(\Z\) chain as any element of \(A\). 
Every formula is equivalent to a \qf one, so as these are all of the possible complete atomic types, they are also all of the possible types.    

\subsection{A Theorem about \(\omega\)-stability}

We now aim to show that all theories, \(T\), which aren't \(\omega\)-stable cannot be uncountably categorical.

It suffices to show that for all uncountable \(\kappa\) we can present two non-isomorphic models of \(T\) of cardinality \(\kappa\). 
We will do so by finding one model realizing only countably many types, and one realizing uncountably many. 
Clearly, two such models will not be isomorphic, so \(T\) will not be categorical. 

As \(T\) isn't \omst, for some countable \(A\) we have that \(|S_n^\mcM(A)|\) is uncountable. 
We can realize all of these in some elementary extension of \(\mcM\) which can be taken to be of cardinality exactly \(\kappa\) for any desired uncountable \(\kappa\). 
Say \(\{p_i\}\) is a set of uncountably many but at most \(\kappa\) many types indexed by some set \(I\). 
Let \(c_i\) similar index that same number of constants in a new language \(\mcL^* = \mcL_\mcM \cup \{c_i\}\). 
Let \(\Gamma = T \cup \eldiag(\mcM) \cup \{p_i(c_i)\}\) (we assert each type \(p_i\) is realized by \(c_i\)). 
We claim that every finite subset of \(\Gamma\) is satisfiable (indeed, every finite subset is true in \(\mcM\)).
By compactness, we have a model of \(\Gamma\) of cardinality \(|\mcL^*| \leq \kappa\) and by the upward L\"owenheim-Skolem theorem, a model of cardinality \(\kappa\).
\(\Gamma\) asserts that uncountably many types are realized, as desired. 

Now, we must show the existence of a model of size \(\kappa\) which realizes only countably many types. To do so, we will have to construct some machinery.
This will be our guiding intuititon: to ensure our model is of size \(\kappa\), we will start out with \(\kappa\) many virtually identical (indiscernable) elements. 
To ensure our model is indeed a model of \(T\), we will close it under existentials by adding the right elements. 
We will be able to show that these ``right elements'' do not change the cardinality of the model and do not realize any types not found in out countable model.
Only countably many types can be realized in a countable model, so our constructed model will realize countably many types over any countable set, as desired. 

\textbf{Definition.} We say that a \(\{x_i \mid i \in I\}\) where \(I\) is an ordered set is a sequence of \textit{order indiscernibles} if, given two increasing sequences of length \(n\) from \(I\), 
\(i_1 < i_2 < \ldots < i_n\) and \(j_1 < j_2 < \ldots < j_n\), we will have \(\mcM \models \phi(i_1, \ldots, i_n) \iff \mcM \models \phi(j_1, \ldots, j_n)\). 
Intuitively, these elements cannot be distinguised based on first order properties, modulo their relative positions.  
Recall that the theory of dense linear orders without endpoints admits \qe and it's easy to see that any increasing sequence from such an order is a sequence of indiscernibles. 

% Perhaps add something about how < need not be present in the language?

Our first theorem about order indiscernibles for every underlying order \(I\), is that they always exist in a theory with infinite models. 
Augment our language with constants \(\{c_i \mid i \in I\}\).
Let \(\Gamma\) assert that these constants are order indiscernibles, more explicitly:
\begin{itemize}
\item \(\Gamma\) asserts that \(c_i \neq c_j\) whenever \(i \neq j\)
\item For every \(\mcL\)-formula, \(\phi(v_1, \ldots, v_n)\), and pair of sequences of \(n\) increasing elements of \(I\), \((i_1, \ldots, i_n)\) and \((j_1, \ldots, j_n)\), \(\Gamma\) asserts that \(\phi(c_{i_1}, \ldots, c_{i_n}) \leftrightarrow \phi(c_{j_1}, \ldots, c_{j_n})\)
\end{itemize}

Obviously, any model of \(T \cup \Gamma\) has a set of order indiscernibles indexed by \(I\). 
We claim that any finite subset of \(T \cup \Gamma\) is satisfied by any \(\mcM\) an infinite model of \(T\). 
\textcolor{red}{This proof is missing because I'm looking for a way to provide the infinite combinatorial details.}

It is natural to identify this set of order indiscernibles with \(I\), so that if \(\mcM \models \Gamma \cup T\), then \(I \subset M\).

\textcolor{red}{\(\ldots\)}

Now that we are familiar with order indiscernibles, we can use them to build \textit{Ehrenfeucht-Mostowski Models}, which are constructed as \textit{Skolem hulls} of order indiscernibles. 
We recall that an \(\mcL\)-theory is said to have \textit{built-in Skolem functions} when for all \(\mcL\)-formulae, \(\phi(v, \bar{w})\), there is a function symbol \(f_\phi\) for which 
\(T \models \A \bar{w} [(\E v \phi(v, \bar{w})) \to \phi(f_\phi(\bar{w}), \bar{w})]\), in other words, the Skolem functions provide whitnesses for all existentials. 
Given any \(\mcL\)-theory \(T\), we can extend \(\mcL\) to \(\mcL^*\) so that it includes Skolem-functions and we also extend \(T\) to \(T^*\) which asserts that \(\mcL^*\) has Skolem-functions. 
Moreover, there is a natural way to interpret any \(\mcM \models T\) as an \(\mcL^*\) structure modeling \(T^*\).
Given, then \(I \subset M\), we can look at the substructure of \(\mcM\) whose universe will be the smallest set containing \(I\) and closed under our Skolem functions. 
We will call this model ``\(\mathcal{H}(I)\)'' or the Skolem hull of \(I\).
Note that \(\mathcal{H}(I) \prec \mcM\).

We can now achieve our goal of constructing a model of any \(\omega\)-unstable theory which realizes only countably many types over any countable set. 
As above, simply take \(\mcM = \mathcal{H}((\kappa, <))\). \(M\) will have cardinality \(\kappa\).
Let \(A\) be our countable subset of \(M\). 
Every element of \(A\) (like every element of \(M\)) is of the form \(f_\phi(\bar{x})\) for order indiscernibles \(x\).  
The set of all indiscernibles appearing in one of these Skolem terms is still countable, so we can assume \(A\) contains only order indiscernibles.
\textcolor{red}{Fill in more details.}

\textcolor{red}{\(\ldots\)}


\section{Vaughtian Pairs}
\subsection{Definition}
Like \(\omega\)-instability, lacking Vaughtian pairs is a property of theories which prevents uncountable categoricity (we will prove this in Theorem \ref{theorem_vaughtian_pairs_categoricity}).
The content of Morley's Categoricity theorem (Theorem \ref{theorem_morleys_categoricity}) will be that these are the only two obstacles to uncountable categoricity. 

\begin{definition}\label{definition_definable_subset}
Let \(\mcM\) be an \(\mcL\)-structure and \(\phi\) an \(\mcL\) formula in some number of free variables.
We define \(\phi(\mcM) = \{\bar{m} \in M \mid \mcM \models \phi(\bar{m})\}\).
If \(S = \phi(\mcM)\) for some \(\phi\) we say that \(S\) is a definable subset of \(\mcM\) and that it is defined by \(\phi\).
\end{definition}

\begin{definition}\label{definition_vaughtian_pairs}
Two distinct \(\mcL\)-structures \((\mcN, \mcM)\) form a Vaughtian pair if \(\mcM \prec \mcN\) and for some \(\mcL_\mcM\)-formula \(\phi\), \(\phi(\mcM) = \phi(\mcN)\) is infinite. 
\end{definition}

Even though we write \(\mcM \prec \mcN\), we write the Vaughtian pair in the other order, that is as \((\mcN, \mcM)\), because we can think of this pair of models as the model \(\mcN\) with a distinguished subset corresponding to \(\mcM\), that is a subset defined by a unary predicate.
This will often be useful in proving theorems and lemmas about Vaughtian pairs, so we will flesh out the construction here. 

If \(\mcM, \mcN\) are \(\mcL\)-structures, let  \(\mcL^U = \mcL \cup \{U\}\). 
We look at (\(\mcN, \mcM\)) as an \(\mcL^U\)-structure which shares its underlying set and interpretations of \(\mcL\) with \(\mcN\) but where \(U\) is the subset \(M\). 
Note that, if \(\phi\) is an \(\mcL\)-formula \((\mcN, \mcM) \models \phi \iff \mcN \models \phi\).
Adding the unary predicate, \(U\), to our language allows us to find an \(\mcL^U\)-formula \(\phi_U\) corresponding to any \(\mcL\)-formula \(\phi\) such that \((\mcN, \mcM) \models \phi_U \iff \mcM \models \phi\).

We define \(\phi_U\) as follows:
\begin{enumerate}
\item If \(\phi(\bar{v})\) is atomic, we let \(\phi_U(\bar{v}) = \bigwedge\limits_{v \in \bar{v}}U(v) \land \phi(\bar{v})\)
\item If \(\phi(\bar{v}) = \neg \psi(\bar{v})\) then \(\phi_U(\bar{v}) = \neg \psi_U(\bar{v})\)
\item If \(\phi(\bar{v}) = (\psi \land \theta)(\bar{v})\) then \(\phi_U(\bar{v}) = \psi_U(\bar{v}) \land \theta_U(\bar{v})\)
\item If \(\phi(\bar{v}) = \E x \, \psi(x, \bar{v})\) then \(\phi_U(\bar{v}) = \E x\, (U(x) \land \psi_U(x, \bar{v}))\)
\end{enumerate}
A simple induction shows that \((\mcN, \mcM) \models \phi_U \iff \mcM \models \phi\).

The following is now an easy consequence of the downward L\"owenheim-Skolem theorem:

\begin{theorem}\label{theorem_countable_vaughtian_pairs}
If \(\mcL\) is countable and an \(\mcL\)-theory, \(T\), has a Vaughtian pair \(\mcN, \mcM\), it has a Vaughtian pair \((\mcN_0, \mcM_0)\) of countable models. 
\end{theorem}

\begin{proof}
It will suffice to show that there is an \(\mcL^U\)-theory, \(T'\), for which \((\mcA, \mcB) \models T'\) iff \((\mcA, \mcB)\) forms a Vaughtian pair for \(T\) with \(\phi(\mcB) = \phi(\mcB)\) an infinite set. 
As \((\mcN, \mcM)\) forms a Vaughtian pair, \(T'\) is satisfiable.
As \(\mcL^U\) is countable, it has a countable model, \((\mcN_0, \mcM_0)\), by the downward L\"owenheim-Skolem theorem.
This is the desired countable Vaughtian pair. 

Our goal is to demonstrate an \(\mcL_\mcM^U\)-theory, \(\Gamma\), asserting that \((\mcN, \mcM)\) is Vaughtian pair.

\(\Gamma\) is the union of the following sets of formulae:
\begin{itemize}
\item \(T\) and \(\{\phi_U \mid \phi \in T\}\). (That is \(\mcM, \mcN \models T\).)
\item \(\A \bar{v} \, (U(\bar{v}) \land \E x \, \phi(x, \bar{v})) \to \E x \, U(x) \land \phi(x, \bar{v})\) for all \(\phi\). 

(By Theorem \ref{theorem_tarski_vaught_test} this shows \(\mcM \prec \mcN\)) 

\item If \(\psi(\bar{v})\) is the \(\mcL_\mcM\)-formula which isolates an infinite subset of \(\mcM\), we take the set \(\{\psi_n \mid n \in \N\}\) where \(\psi_n = |\{\bar{v} \mid \psi(\bar{v})\}| > n\).\footnote{\(\psi_n\) is first order.
Specifically, \(\psi_n = \E x_1 \ldots \E x_n \, \left(\bigwedge\limits_{0 <i < j \leq n} x_i \neq x_j\right) \land \left(\bigwedge\limits_{0 < i \leq n} \psi(x_i)\right)\). Similarly, we can assert \(|\{\bar{v} \mid \psi(\bar{v})\}| = n\) as \((\psi_n \land \neg \psi_{n+1})\).}
That is \(\psi\) defines an infinite subset of \(\mcN\).

\item \(\A \bar{v} \, (\psi(\bar{v}) \to U(\bar{v}))\). That is, \(\psi(\mcN) \subseteq \psi(\mcM)\). As the other direction is trivial, this ensures \(\psi(\mcN) = \psi(\mcM)\). 
\item Finally, we assert that \(\mcN\) is a proper elementary extension of \(\mcM\) with \(\E x \, \neg U(x)\). 
\end{itemize}

By construction, \(\Gamma\) is an \(\mcL^U\)-theory, \(T'\), for which \((\mcA, \mcB) \models T'\) iff \((\mcA, \mcB)\) form a Vaughtian pair for \(T\) (where the required infinite set is defined by \(\psi\)). 
The L\"owenheim-Skolem theorem gives us our desired countable model. 
\end{proof}

\subsection{Intuition for Vaughtian Pairs and Categoricity}

Before we proceed, it will be useful to develop some intuition about why Vaughtian pairs prevent uncountable categoricity for theories in countable languages.

\begin{theorem}\label{theorem_vaughtian_pair_categoricity}
If \(\mcM \models T\), \(|\mcM| = \kappa > \aleph_0\) and \(\phi(\mcM)\) is countably infinite then \(T\) is not \(\kappa\)-categorical and \(T\) has a Vaughtian pair. 
\end{theorem}

\begin{proof}
We first show that there is a Vaughtian pair for \(T\).
We have established that we can assume without loss of generality that \(T\) has built in Skolem functions. 
If we take the submodel of \(\mcM\) generated by \(\phi(\mcM)\) this will be a countable (and therefore proper) elementary submodel of \(\mcM\) in which \(\phi\) will define the same subset, \(\phi(\mcM)\).
This provides the desired Vaughtian pair. 
Additionally, \(T\) cannot be \(\kappa\) categorical as we can construct an \(\mcN \models T\) with \(|\mcN| = \kappa\) and \(|\phi(\mcN)| = \kappa\) (an isomorphism \(f: \mcM \to \mcN\) would be bijection when restricted to \(\phi(\mcM)\) and by definition there is no bijection from a countable set to any uncountable \(\kappa\)).
We simply add constant symbols \(\{\bar{c}_i \mid i \in \kappa\}\) to our language. 
We can easily see that \(\Gamma = \text{Th}(\mcM) \cup \{\bar{c}_i \neq \bar{c}_j \mid i \neq j\} \cup \{\phi(\bar{c}_i) \mid i \in \kappa\}\) is finitely satisfiable (it suffices to find \(n\) distinct realizations of \(\phi\) in \(\mcM\) for all \(n \in \N\) which is possible as \(\phi(\mcM)\) is infinite). 
Any \(\mcN \models \Gamma\) will have \(|\phi(\mcN)| = \kappa\). 
\end{proof}

We see the connection between some Vaughtian pairs and lacking \(\kappa\)-categoricity, specifically Vaughtian pairs \((\mcN, \mcM)\) for which \(|\mcN| = \kappa\) and \(|\mcM| = \aleph_0\).
It will suffice to show that whenever there is a Vaughtian pair for an \omst theory \(T\), we can construct a model of \(T\) which has a countable definable subset of any uncountable cardinality \(\kappa\). 
If \(T\) is not \omst then Theorem \ref{theorem_omega_stability_categoricity} tells us \(T\) is not \(\kappa\)-categorical.
Otherwise, this construction, coupled with Theorem \ref{theorem_vaughtian_pair_categoricity} tells us \(T\) isn't \(\kappa\)-categorical.
The rest of this chapter will show this construction. 

\subsection{Homogeneity}

For the next few claims we will need to use a property called homogeneity which will allow us to show that certain countable models are isomorphic.

\begin{definition}\label{definition_homogeneity}
A countable model \(\mcM\) is homogeneous if, for all finite \(A \subseteq M\) and partial elementary functions \(f: A \to M\) and \(m \in M\), there is a partial elementary \(f' \supseteq f\) with domain \(A \cup \{m\}\) (taking values in \(M\)). 
\end{definition}

\begin{lemma}\label{theorem_partial_elementary_automorphism}
Any partial elementary map \(f: A \to M\) (where \(A \subseteq M\) is finite) can be extended to an automorphism. 
\end{lemma}

\begin{proof}
We define an isomorphism extending \(f\) inductively. 
Let \(f_0 = f\) be the first function in our chain.
We will define \(f_i\) for \(i > 0\) such that \(f_{i} \supseteq f_{i-1}\).
We will alternatively extend the domain and range of \(f_i\) by elements in \(M = \{m_1, m_2, \ldots\}\).
That is, if \(i\) is odd, \(m_i \in \text{dom}(f_{\frac{i-1}{2}})\) and if \(i\) is even \(m_i \in \text{rng}(f_{\frac{i}{2}})\).
The former is possible as homogeneity allows us to expand the domain of \(f_{i-1}\) to include \(m_i\).
For the case of \(i\) even, note that partial elementary maps have partial elementary inverses (if \(f\) is partial elementary, it is injective). 
That partial elementary functions have partial elementary inverses allows us to extend the range of \((f_{i-1})^{-1}\) to include \(m_i\). 
Taking the inverse of the resulting function gives us \(f_i\).
Let \(f^* = \bigcup\limits_{i \in \N}f_i\).
\(f^*\) has domain and range equal to all of \(M\) and is partial elementary, making \(f^*\) our desired automorphism.

This is a ``back and forth'' argument, much like the one we saw in Example \ref{example_categoricity_sequence}. 
In retrospect, we see that the construction in that example relied on homogeneity. 
\end{proof}

\begin{lemma}\label{lemma_types_isomorphism}
The next thing we note about countable homogeneous models, \(\mcM\) and \(\mcN\), of the same complete theory, \(T\), is that if \(\mcM\) and \(\mcN\) realize the same types in \(S_n(T)\), they are isomorphic.
\end{lemma}

\begin{proof}
To show this, we will build a map in a way very similar to the way we built \(f^*\) in the previous Lemma \ref{theorem_partial_elementary_automorphism}.
We start by letting \(f_0 = \emptyset\), which is partial elementary if \(T\) is complete.
We will enumerate the elements in the universe of both models, and iteratively add elements to the domain and range of a partial elementary map as above.
If we can do this, the union of this sequence of maps will be an isomorphism.

As the inverse of a partial elementary map is also partial elementary, it will suffice to show that, for an arbitrary  partial elementary map \(f\) and element \(n \in N\), we can find \(f'\supseteq f\) with \(m \in \text{dom}(f')\).
Let \(\text{dom}(f) = \{m_1, \ldots, m_i\}, \text{rng}(f) = \{n_1, \ldots, n_i\}, f(m_j) = n_j\).
Note that we can find a partial elementary \(h\) for which \(dom(f) \cup \{m\} =  dom(h)\) by merely finding the realization of \(\text{tp}(\bar{m}, m)\) in \(\mcN\), which must exist by hypothesis.
Our one remaining obstacle is that \(h(m_i)\) may not be equal to \(n_i\) for all \(i\), that is, it is quite possible that \(f \not \subseteq h\). 
This can be rectified with homogeneity. 
We can easily see that the function \(g: h(m_i) \mapsto n_i\) and undefined elsewhere, is a partial elementary map which can be extended to an automorphism by Lemma \ref{theorem_partial_elementary_automorphism}. 
Call this automorphism \(g'\). 
We take \(f' = g' \circ h\).
\(f'\) is partial elementary, extends \(f\) and has \(m\) in its domain.   

We alternate adding elements of \(M\) and \(N\) to domain and range of \(f_i\) as above and take the union of these functions.
Their union will be a partial elementary bijection defined on all of \(M\) and therefore an isomorphism.
\end{proof}

\subsection{Uncountable Models with Definable Subsets}

\begin{theorem}\label{theorem_countable_isomorphic_vaughtian_pair}
A Vaughtian pair of countable models, (\(\mcN_0, \mcM_0\)), can be extended to a Vaughtian pair of isomorphic countable models.
\end{theorem}

\begin{proof}
By Lemma \ref{lemma_types_isomorphism}, it suffices to show a countable Vaughtian pair \((\mcN, \mcM)\) can be extended to a countable Vaughtian pair \((\mcN^*, \mcM^*)\) where \(\mcN^*, \mcM^*\) are homogeneous and realize the same types. 

We aim to construct a chain of Vaughtian pairs \((\mcN_i, \mcM_i)\) such that \((\mcN_i, \mcM_i) \prec (\mcN_{i+1}, \mcM_{i+1})\) and their union, \(\bigcup\limits_{i\in \N} (\mcN_i, \mcM_i)\) will be homogeneous and realize the same types. 
The first pair in our chain is \((\mcN_0, \mcM_0) = (\mcN, \mcM)\) from above. 
For \(i > 0\) we construct (\(\mcN_i, \mcM_i\)) as follows (the details of constructions in each of these stages will follow):

\begin{enumerate}

  \item  \(i\) is a multiple of 3: then (\(\mcN_i, \mcM_i\)) is an elementary extension of (\(\mcN_{i-1}, \mcM_{i-1}\)) such that every type in \(S_n(T)\) realized in \(\mcN_{i-1}\) is realized in \(\mcM_i\). (This ensures \(\mcN^*, \mcM^*\) realize the same types.)

  \item  If \(i = 3k+1\) , then (\(\mcN_i, \mcM_i\)) is an elementary extension of (\(\mcN_{i-1}, \mcM_{i-1}\)) such that if \(\bar{a}, \bar{b}\) realize the same type in \(\mcM_{i-1}\) and \(c \in \mcM_{i-1}\), then there is a \(d \in \mcM_i\) such that \(\bar{a}c\) and \(\bar{b}d\) realize the same type in \(\mcM_i\). (This will ensure \(\mcM^*\) is homogeneous.) 

  \item If \(i = 3k + 2\): then (\(\mcN_i, \mcM_i\)) is an elementary extension of (\(\mcN_{i-1}, \mcM_{i-1}\)) such that if \(\bar{a}, \bar{b}\) realize the same type in \(\mcN_{i-1}\) and \(c \in \mcN_{i-1}\), then there is a \(d \in \mcN_i\) such that \(\bar{a}c\) and \(\bar{b}d\) realize the same type in \(\mcN_i\). (This will ensure \(\mcN^*\) is homogeneous.) 

\end{enumerate}

If we let \((\mcN^*, \mcM^*) = \bigcup\limits_{i \in \N}(\mcN_i, \mcM_i)\), we see that every type realized in \(\mcN^*\) is realized in some \(\mcN_i\) and therefore in \(\mcM_{i+3}\) (as we must have done the first stage of our construction once in the interim) and thus in \(\mcM^*\). 
Similarly, any type realized in \(\mcM^*\) must be realized in some \(\mcM_i\) and in \(\mcN_i\) as well (as \(\mcM_i \prec \mcN_i\)) and thus in \(\mcN^*\).
Thus, \(\mcM^*\) and  \(\mcN^*\) realize the same types.  
Given a partial elementary map \(f:\mcM^* \to \mcM^*\) with finite domain, we have \(\text{dom}(f) \in \mcM_i\) for some \(i\), and stage two of our construction guarantees that it can be extended. 
Similarly for homogeneity of \(\mcN^*\) by the third stage of our construction. 

At this point it will suffice to show how to construct the elementary extensions mentioned in our three stages. 

\begin{enumerate}
\item As we can enumerate the countably many types realized in any countable \(\mcN\) (there are countably many finite sequences from a countable set and each realizes just one type), it would suffice to show that if \(p(\bar{v})\) is a single type and \((\mcN, \mcM)\) is a Vaughtian pair with \(\mcN\) realizing \(p(\bar{v})\) that there is a \((\mcN, \mcM) \prec (\mcN', \mcM')\) with \(\mcM'\) realizing \(p(\bar{v})\). 
If we can realize a single type in an elementary extension, enumerating the types as \(p_1, p_2, \ldots\) and forming an elementary chain where each \(p_i\) is realized in \(\mcM\) in the \(i\)th model in the chain and taking the union will suffice.

Let \(p(\bar{v})\) be a given type. 
We can see that \(\Gamma = \eldiag((\mcN, \mcM)) \cup \{\phi_U(\bar{v}, \bar{a}) \mid \phi(\bar{v}, \bar{a}) \in  p(\bar{v})\}\) is satisfiable.  
Specifically, if \(\phi_1(\bar{v}, \bar{a}), \ldots, \phi_k(\bar{v}, \bar{a}) \in p\) are a finite subset of this type, we know that \(\mcN \models \bigwedge\limits_{j \leq k} \E \bar{x}\, \phi_j(\bar{x},\bar{a})\) and therefore that \(\mcM \models \bigwedge\limits_{j \leq k} \E \bar{x} \, (\phi_j)_U(\bar{x},\bar{a})\). 
This finite subset is realized in \((\mcN, \mcM)\) itself. 
Any model of \(\Gamma\) is an elementary extension of \((\mcN, \mcM)\) which we may call \((\mcN', \mcM')\).
We can also take \((\mcN', \mcM')\) to be countable as we did in Theorem \ref{theorem_countable_vaughtian_pairs}. 
As \((\mcN', \mcM') \models \Gamma\) we also have that \(p(\bar{v})\) is realized in \((\mcN', \mcM')\). 

\item The construction for this stage is very similar to the one covered just above for the first stage.
Again, it will suffice to show that just one such type can be realized (namely the type of \(c\) over \(\bar{a}\)) as we can add realizations for our countably many types one at a time. 
Here, we take \(\Gamma = \eldiag((\mcN, \mcM)) \cup \{\phi(\bar{b}, d) \mid \mcM \models \phi(\bar{a}, c)\}\). 
If some finite subset, \(\Phi\), were not realizable, we would have \(\mcM \models \exists x \bigwedge\limits_{\phi \in \Phi}\phi(\bar{a}, c)\) but this formula wouldn't hold for \(\bar{b}\), giving a contradiction to \(\bar{a}, \bar{b}\) having the same type. 

\item This case is very similar to the previous one. Here we let \(\Gamma = \eldiag((\mcN, \mcM)) \cup \{\phi(\bar{b}, d) \mid \mcN \models \phi(\bar{a}, c)\}\) where \(d\) is a new constant symbol.
If \(\Gamma\) is not finitely satisfiable that must mean that for some \(\phi\) we have \(\mcN \models \phi(\bar{a}, c) \land \neg \E x \, \phi(\bar{b}, x)\), 
so \(\bar{a}, \bar{b}\) cannot have the same type, a contradiction.
\end{enumerate}
This completes the proof. 
\end{proof}

\begin{lemma}\label{lemma_extend_isomorphically}
Let \((\mcN, \mcM)\) be a Vaughtian pair of countable models, \(\mcN'\) a countable model and \(f:\mcN' \to \mcM\) is an isomorphism. 
We claim that there must exist an \(\mcN''\) such that \((\mcN'', \mcN') \isom (\mcN, \mcM)\) and that, moreover, we can take \(N' \subseteq N''\). 
\end{lemma}

\begin{proof}
Let \(N'' = N' \sqcup (N \setminus M)\) (note that this ensures \(N' \subseteq N''\)).
Define \(f'': N'' \to N\) as follows: if \(x \in N'\) then \(f'(x) = f(x)\). 
Otherwise, \(f'(x) = x\). 
By construction, \(f'\) is a bijection between \(N''\) and \(N\) which, when restricted to \(N'\) is a bijection between  \(N'\) and \(M\). 
For atomic \(\mcL^U\)-formulae \(\phi\), let \((\mcN'', \mcN') \models \phi(\bar{v}) \iff (\mcN, \mcM) \models \phi(f'(\bar{v}))\).
This gives us interpretations of \(\mcL\) for \(\mcN''\) such that \(f'\) is an \(\mcL\)-isomorphism by construction. 
\end{proof}

\begin{theorem}\label{theorem_aleph_one_vaughtian_pair}
Given a Vaughtian pair of countable isomorphic models, \((\mcN, \mcM) \models T\), we can construct a model \(\mcN^* \models T\) which is of cardinality \(\aleph_1\) and has a definable subset which is countable.  
\end{theorem}

\begin{proof}
We will construct \(\mcN^*\) as the union of an elementary chain of models \(\mcN_\alpha\) indexed by \(\omega_1\).

Let \(\mcN_0 = \mcN\). 
We'd like to maintain \(\mcN_\alpha \isom \mcN\) and \(\phi(\mcN_\alpha) = \phi(\mcN)\) and \(\mcN_\alpha\) is countable. 
This is true by hypothesis for \(\mcN\).

If \(\alpha\) is not a limit ordinal, let \(\mcN_{\alpha}\) be a countable model of \(T\) such that \((\mcN_{\alpha}, \mcN_{\alpha-1}) \isom (\mcN, \mcM)\) and \(N_{\alpha-1} \subseteq N_\alpha\) which exists by Lemma \ref{lemma_extend_isomorphically}. 
As \((\mcN_{\alpha}, \mcN_{\alpha-1})\) is a Vaughtian pair, we've ensured that \(\mcN_{\alpha-1} \prec \mcN_{\alpha}\) and \(\mcN_{\alpha-1} \neq \mcN_{\alpha}\).
We have \(\phi(\mcN_{\alpha}) = \phi(\mcN_{\alpha-1})\) and by hypothesis \(\phi(\mcN_{\alpha-1}) = \phi(\mcM)\). 
Similarly,  \(\mcN_{\alpha} \isom \mcN_{\alpha-1} \isom \mcN\). 

To define \(\mcN_\alpha\) where \(\alpha\) is a limit ordinal, let \(\mcN_\alpha = \bigcup\limits_{\beta < \alpha}\mcN_\beta\). 
As \(\mcN_\alpha\) is the union of countably many countable sets, it is countable. 
If \(\bar{a} \in \mcN_\alpha\) then \(\bar{a} \in \mcN_\beta\) for some \(\beta < \alpha\). 
As \(\mcN_\beta \prec \mcN_\alpha\), \(\text{tp}^{\mcN_\beta}(\bar{a}) = \text{tp}^{\mcN_\alpha}(\bar{a})\). 
As \(\mcN_\beta \isom \mcM\), this type is realized in \(\mcN\).
Thus, every from \(S_n(T)\) realized in \(\mcN_\alpha\) is realized in \(\mcN\). 
The converse follows from \(\mcN \prec \mcN_\alpha\). 
Thus \(\mcN \isom \mcN_\alpha\).

If \(\mcN^* = \bigcup\limits_{\alpha < \omega_1}\mcN_\alpha\) we know that \(\phi(\mcN^*) = \phi(\mcN) \subseteq \mcM\) which is countable.
As \(\mcN \prec \mcN^*\) we have \(\mcN^* \models T\).  
As we add at most countably many but at least one element at each of the \(\aleph_1\) stages, we have \(|\mcN^*| = \aleph_1\).
\end{proof}

\subsection{Vaughtian Pairs and Categoricity}

\begin{theorem}\label{theorem_uncountable_vaughtian_pairs}
Given \(\mcM \models T\) where \(|M| = \aleph_1\), \(T\) is \(\omega\)-stable and \(\phi(\mcM)\) is countably infinite, for every uncountable cardinal \(\kappa\), there is a model of \(T\) of cardinality \(\kappa\) with a countably infinite definable subset.  
\end{theorem}

\begin{proof}
As in Theorem \ref{theorem_aleph_one_vaughtian_pair}, it suffices to show that if \(\mcM\) is uncountable and has a countably infinite definable subset defined by \(\phi\), that there is an \(\mcM \prec \mcN\) which is a strict elementary extension such that \(\phi(\mcN) = \phi(\mcM)\) and \(|\mcM| = |\mcN|\).

First, we show there is an \(\mcL_\mcM\)-formula, \(\psi\) such that there are uncountably many realizations of \(\psi(v)\) in \(\mcM\) and for all \(\mcL_\mcM\)-formulae, \(\theta\), either \(\psi \land \theta(v)\) or \(\psi \land \neg\theta(v)\) has only countably many realizations. 

For contradiction, we build an infinite binary tree of formulae such that all paths in the tree are types over a countably set, and all are satisfiable, contradicting \(\omega\)-stability as we did in Theorem \ref{thm_omst_prime}.  
The root of the tree is \(v = v\), which has uncountably many realizations. 
As this formula is not our desired \(\psi\), there is a \(\theta\) such that \((v = v) \land \theta(v)\) and \((v = v) \land \neg \theta(v)\) have uncountably many realizations. 
Every path in this tree is a finitely satisfiable type over the countable set of constants from \(\mcM\) appearing in the tree as in
Theorem \ref{thm_omst_prime}.
This contradicts \(\omega\)-stability. 

Given \(\psi\) as above, we now construct a type over \(M\), \(p\), the type of an element not in any of the countable definable subsets of \(\psi(\mcM)\). 
That is \(p = \{\theta(v) \mid |(\theta \land \psi (\mcM))| \geq \aleph_1\}\).
This type is finitely satisfiable. If we take \(\theta_1, \ldots, \theta_n\) from \(p\) then each fails in on at most countably many elements. 
Their conjunction also fails to hold of at most countably many elements. 
\(p\) is actually a complete type as a formula is in this type  exactly when it is true of uncountably many elements of \(\mcM\) and it's negation is in this type otherwise. 
Note that \(\mcM\) cannot contain a realization of \(p\) (for each element \(m \in M\), \(p\) contains \(v \neq m\)). 
We realize \(p\) in \(\mcM'\) such that \(\mcM \prec \mcM'\). 
Let \(c\) be the realization of \(p\). 
By Theorem \ref{thm_omst_prime} we can find a \(\mcM_0 \prec \mcM'\) prime over \(M \cup \{c\}\). 
Additionally, \(\mcM \prec \mcM_0\). 
This \(\mcM_0\) will be our desired model. 

% THIS PART IS TRICKY
%%%%%%%%%%%%%%%%%%%%%%%%%%%%%%%%%%%%%%%%%%%%%%%%%%%%%%%%%%%%%%%%%%%%%%%%%%%%%%%%
All that remains is to show that \(\mcM_0\) has the desired property, that \(\phi(\mcM) = \phi(\mcM_0)\).   
It suffices to show that the type \(\Gamma(v) = \{\phi(v)\} \cup \{v \neq m \mid m \in \phi(\mcM)\}\) is not realized. 
Assume for the purposes of contradiction that it is realized in \(\mcN\) by some \(v\), and we will show \(\Gamma\) must be realized in \(\mcM\) as well. 
There must be an \(\mcL_\mcM\) formula \(\psi(x, y)\) such that \(\psi(v, c)\) isolates \(\text{tp}(v/M \cup \{c\})\). % Why? Do something with the prime models
Note that the formula \(\E x \, \psi(x, c)\) is true of \(c\) and is therefore in \(p(c)\). 
Moreover, if \(\gamma(v) \in \Gamma(v)\) then \(\A v (\psi(v, c) \to \gamma(v)) \in p(c)\). 
Let \(\Delta(c) = \{\A v \, (\psi(v, c) \to \gamma(v)) \mid \gamma \in \Gamma\} \cup \{\E v \, \psi(v, c)\}\).
\(\Delta\) is a subset of \(p\) so it is realized by \(c\). 
If another \(d\) realizes \(\Delta\) then \(\E w \, \psi(w, d)\) and \(w\) realizes \(\Gamma\).
Let \(\{\delta_0(c), \delta_1(c), \ldots\} = \Delta(c)\). 
\(\{x \in M \mid \psi(x)\}\) must be uncountable by construction. 
For every \(n\), though, \(|\{x \in M \mid \psi(x) \land \neg (\bigwedge\limits_{i=0}^n\delta_i)\}|\) is countable (by the definition of \(p \ni \delta_i\)).
So only countably many \(x\) realizing \(\psi\) fail to realize \(\Delta\), let \(d\) be one of the uncountably many elements realizing \(\Delta\) in \(M\).
As above, \(w\) realizes \(\Gamma\) in \(\mcM\). This is our contradiction.  
\end{proof}
%%%%%%%%%%%%%%%%%%%%%%%%%%%%%%%%%%%%%%%%%%%%%%%%%%%%%%%%%%%%%%%%%%%%%%%%%%%%%%%%

\begin{theorem}\label{theorem_vaughtian_pairs_categoricity}
Any theory, \(T\), which has Vaughtian pairs cannot be \(\kappa\)-categorical for any uncountable \(\kappa\).
\end{theorem}

\begin{proof}
If \(T\) is not \omst it cannot be categorical for any uncountable \(\kappa\) by Theorem \ref{theorem_omega_stability_categoricity}.
Otherwise, as \(T\) has a Vaughtian pair, by Theorem \ref{theorem_countable_vaughtian_pairs} it has a Vaughtian pair of countable models. 
By Theorem \ref{theorem_countable_isomorphic_vaughtian_pair} it has a Vaughtian pair of countable isomorphic models.
By Theorem \ref{theorem_aleph_one_vaughtian_pair} it has a model of size \(\aleph_1\) with a countable definable subset. 
By Theorem \ref{theorem_uncountable_vaughtian_pairs} it has a model of size \(\kappa\) which has a countable subset.  
Theorem \ref{theorem_vaughtian_pair_categoricity} tells us that \(T\) cannot be \(\kappa\)-categorical. 
\end{proof}


\section{Proof of Morley's Categoricity Theorem}
At this point we've made good progress towards the proof of Morley's Categoricity theorem.
It suffices to show that uncountable categoricity (for any uncountable cardinal) is equivalent to \(\omega\)-stability and lacking Vaughtian pairs.
As this property doesn't depend on the uncountable cardinality, this suffices. 

We've already shown that \(\omega\)-stability and lacking Vaughtian pairs are necessary for uncountable categoricity.
To show these conditions are also sufficient we will define a notion of ``dimension'' for elementary extensions.
\(\omega\)-stability will tell us that we can view any two models as elementary extensions of the same \textit{prime model}.
The fact that, because of their cardinalities, any equinumerous models will have the same dimension over this prime model.
Having the same dimension over this prime model will suffice to show that these two models are isomorphic.
This will complete the proof. 

\subsection{Prime Models}

\textbf{Definition.} We say that a theory, \(T\),  has a prime model, \(\mcM \models T\) if, for every \(\mcN \models T\), we have \(\mcM \prec \mcN\). 
Clearly, prime models are unique up to isomorphism. Easy examples of prime models are \(\Q\) for fields of characteristic 0 and \(\N\) for \(\Th(\N, 0, s)\). % Examples?

\textbf{Theorem.} \omst theories have prime models.
 
% Proof: 4.2.20 p 136 (a transfinite induction)
{\color{red}Prove this \(\ldots\)}
% Strongly minimal formula with params from prime model

\subsection{Minimality}

\textbf{Definition.} We say that a \(\mcL_\mcM\)-formula \(\phi\) is \textit{minimal} if \(\phi(\mcM) = \{\bar{m} \in M \mid \mcM \models \phi(\bar{m})\) is infinite and has no infinite and co-infinite definable subsets. 
We may also call \(\phi(\mcM)\) minimal. 
We will call \(\phi\) and the set \(\phi(\mcM) \subset M\) % -- This could use some work 
\textit{strongly minimal} if they are minimal in any elementary extension of \(\mcM\).
Moreover, a thepry is strongly minimal if the set \(M\) (defined by the formula \(v = v\)) is strongly minimal. 

Now, assume we have a theory which is \omst and has no Vaughtian-pairs. 
As it is \omst it has a prime model.
We will show that it's being \omst also tells us that there is a minimal formula in its prime model.
We will also show this formula is strongly minimal since the theory has no vaughtian pairs. 

\textbf{Lemma.} If \(T\) is \omst there is a minimal formula in all \(\mcM \models T\).

We will build an infinite binary tree of formulae such that 
every node corresponds to an infinite definable subset of our model (and the formula defining it), 
and such that every node is a subset (or consequence) of its parent. 
We will see that every (infinite) path in this tree is a distinct type over a countable set and that there are \(2^{\aleph_0}\) such types. 
This contradicts the \(\omega\)-stability of \(T\).

It is easy to construct this tree. Let the root of the tree be the entire universe, \(M\), of our model which is defined by the formula \(v = v\).
Given a node in our tree defined by a formula \(\phi\) which defines an infinite subset, there must be a formula \(\psi\) such that both \(\phi \land \psi\) and \(\phi \land \neg \psi\) define infinite subsets, as \(\phi\) is not minimal. 
These two formulae/definable sets will be the children of \(\phi\) in our tree. 
As no formula is minimal, we can repeat this ad infinitum. 

Consider a path in this tree. We can consider the corresponding countable set of formulae consisting of the nodes on this path. 
Every finite subset contains a node of maximum depth, which defines an infinite subset of \(M\) and is therefore satisfiable. 
Our set of formulae corresponding to this path can therefore be completed to a type.
No two distinct paths can be completed to the same type. 
Specifically, if two paths diverge after a node \(\phi\), then they will disagree on the formula \(\psi\) as defined above. 

Moreover, there are only countably many nodes in this tree. 
Each formula in this tree contains finitely many constants from \(M\).
Therefore, all of these types can be defined over countably many constants from \(M\). 
We have presented \(2^{\aleph_0}\) types over a finite subset, contradicting the \(\omega\)-stability of \(T\).

\textbf{Lemma.} If \(T\) has no Vaughtian pairs and \(\phi\) is a minimal formula for \(\mcM \models T\) then \(\phi\) is strongly-minimal. 

{\color{red}\(\ldots\) PROOF????} % Do This one

With these past few lemmata we can conclude that any theory which lack Vaughtian pairs and is \omst has a strongly minimal formula with parameters from a countable prime model. 
We will develop the theory of dimensions of elementary extensions and see that any two extensions of cardinality \(\kappa > \aleph_0\) will have dimension \(\kappa\) over the prime model.
From this fact, we will be able to show they are isomorphic, providing us, finally, with \(\kappa\) categoricity and completing the proof of Morley's categoricity theorem. 

\subsection{Algebraic Closure and Dimension}

\textbf{Definition.} Given a model \(\mcM\) and an \(A \subset M\), we define the algebraic closure of \(A\), denoted \(\acl(A)\) to be all elements of \(M\) which are in a finite set definable by an \(\mcL_A\)-formula. 
(Note that by quantifier elimination over algebraically closed fields, this notion of ``algebraic closure'' agrees with the familiar one.)

The algebraic closure has the nice properties that \(\acl(\acl(A)) = \acl(A)\), that \(A \subset B \implies \acl(A) \subset \acl(B)\) and that if \(a \in \acl(A)\) then there is a finite subset \(A_0 \subset A\) such that \(a \in \acl(A_0)\).
This almost gives us that algebraic closure is a pregeometry. 
The only othe fact we would have to know is exchange, namely that if \(a, b \notin \acl(A), a \in \acl(A \cup \{b\})\) then \(b \in \acl(A \cup \{a\})\) 
This is only true when we take \(A\) to be a subset of some strongly minimal set.

\textbf{Theorem.} Let \(S\) be a strongly minimal set. If \(A \subset S, a, b \in S\) and \(a, b \notin \acl(A)\) but \(a \in \acl(A \cup \{b\})\) then \(b \in \acl(A \cup \{a\})\) 

% Clarify: Limiting \acl to a set D


% I think there might be a better, more intuititve presentation for this proof. (something about pairs for which \phi(x, y))
As \(a \in \acl(A \cup \{b\})\), we have \(\mcM \models \phi(a, b)\) for some \(\mcL_A\)-formula % Make sure we use this notation
\(\phi\) where \(\{x \in S \mid \phi(x, b)\}\) contains \(n\) elements. 
Consider the set defined by \(\psi(y)\) asserting \(\{x \in S \mid \phi(x, y)\}\) contains exactly \(n\) elements.
% As neither \(a\) nor \(b\) are in \(\acl(A)\), the set defined by \(\psi\) must not be finite. 
% By minimality, we know that \(\psi(\mcM)\) is cofinite. % Make sure we use this notation.   
We would very much like for \(X = \{y \in S \mid \psi(y) \land \phi(a, y)\}\) to be finite, making \(b \in \acl(A \cup \{a\})\). 
We will show this must be the case by contradiction.  % WRONG?: As \(\mcM \models \psi(b)\), it will suffice to exhibit \(a_1, \ldots, a_{n+1}\) for which \(\phi(a_i, b)\) holds. 
If this fails and \(X\) is infinite, then \(X\) is cofinite. 
Specifically, \(\{y \in S \mid \neg(\psi(y) \land \phi(a, y))\}\) contains \(m\) elements.  
We let \(\theta(x)\) assert that \(\{y \in S \mid \neg(\psi(y) \land \phi(x, y))\}\) contains exactly \(m\) elements.
\(\theta\) also defines a cofinite set, as otherwise \(a \in \acl(A)\).   
Let \(a_1, \ldots, a_{n+1}\) be such that for each \(i\), \(\theta(a_i)\) holds.
As \(\theta(a_i)\) holds, each set \(\{y \in S \mid \psi(y) \land \phi(a_i, y)\}\) is cofinite, as is their intersection. 
For any element, \(b'\) in the intersection, we have \(\psi(b')\) and \(a_i \in \{x \in S \mid \phi(x, b')\} for each \(a_i\) by construction.
This is a contradiction as \(\psi(b')\) asserts there are only \(n\) such elements.  


\textbf{Definition.} We say that % ???
Independent % something something
Basis % something something
Dimension

% 6.1.11
\textbf{Theorem.} There is a partial elementary map from the set defined by \(\phi\) in \(\mcM\) to that set in \(\mcN\). % This isn't what I'd seriously suggest as the statement. 

% define ``Prime over''
% 6.1.17
\textbf{Theorem.} \(\mcM\) is prime over \(\phi(\mcM)\).

\subsection{Proof of Morley's Categoricity Theorem}

In the previous sections we have succeeded in showing theories categorical in any uncountable cardinal are \omst and lack Vaughtian pairs. 
We now show the converse: two models \(\mcM, \mcN\)  of a complete \omst theory, \(T\), in a countable language lacking Vaughtian pairs we can show \(\mcM \isom \mcN\). 
As \(T\) is \omst it has a prime model, \(\mcM_0\), in which we have a minimal formula \(\phi\) (again as \(T\) is \(\omega\)-stable).
Moreover, as \(T\) has no Vaughtian pairs, \(\phi\) is strongly minimal. 
We begin by constructing a partial elementary map \(f:\phi(\mcM) \to \phi(\mcN)\) which is a bijection, which exists as per our theorem above.
% Both \(\phi(\mcM), \phi(\mcN)\) have cardinality \(\kappa\) and therefore dimension \(\kappa\). Maybe some more explanation. 

% We could use some work here:
Something something \(\mcM\) prime over something. Extending \(f\). \(f'\) is surjective. \(f'\) is an isomorphism finally giving us that \(\mcM \isom \mcN\). 


\section{Further Questions}
We begin by restating Morley's categoricity theorem in a way which will allow us to generalize or extend it naturally.

\begin{definition}\label{definition_number_models}
If \(T\) be a complete first order theory \(I(T, \kappa)\) is the number of models of \(T\) of size \(\kappa\) up to isomorphism.
\end{definition}

Morley's categoricity theorem stated that if \(T\) is a theory in a countable language and \(\kappa, \lambda > \aleph_0\) then \(I(T, \kappa) = 1 \iff I(T, \lambda) = 1\). 

We'd like to ask some more general questions, some of which we will be able to answer fully in this section and some of which will only adress in a limited sense. 
In the following, we assume that \(T\) is a theory in a countable language. 

\begin{question}\label{question_countable_models_uncountably_categorical}
If \(\lambda > \aleph_0, I(T, \lambda) = 1\) what can we conclude about \(I(T, \aleph_0)\)? (could it be \(2^{\aleph_0}\)?)
\end{question}

\begin{question}\label{question_finite_spectra}
For which natural numbers, \(n\), are there theories for which  \(I(T, \aleph_0) = n\)?
\end{question}

% \item More generally, what possible values can \(I(T, \aleph_0)\) take on?

\begin{question}\label{question_morleys_conjecture}
Is it possible to have \(\kappa > \lambda > \aleph_0\) but \(I(T, \kappa) < I(T, \lambda)\)?
\end{question}

\begin{question}\label{question_los_conjecture_uncountable_languages}
Is there a natural analogue to Morley's Categoricity Theorem when we no longer assume that \(T\) is a theory in a countable language?
\end{question}

% \item What if we no longer assume that \(T\) is a first order theory?

An easy extension of our argument for Morley's Categoricity Theorem shows us that for uncountably categorical \(T\), \(I(T, \aleph_0)\leq \aleph_0\).
Up to isomorphism, these models are determined by their dimension, which must not be greater than \(\aleph_0\) leaving only countably many possibilities. 
This is an answer to question \ref{question_countable_models_uncountably_categorical}.

With regards to question \ref{question_finite_spectra}, we will exhibit theories which have \(n\) models up to isomorphism for all \(n \in \N\) with the exception of \(2\).
Moreover, we can show that a theory with only two countable models up to isomorphism must not exist. 

% TODO 4?
% You have to actually do this

We can answer question \ref{question_morleys_conjecture} in the negative. This is known as Morley's Conjecture. A full proof is beyond the scope of this paper, but we will present some of the ideas from the proof. 

To adress Question \ref{question_los_conjecture_uncountable_languages}, \L o\'s' Conjecture for uncountable languages (of cardinality \(\kappa\)) states that if \(\lambda, \lambda' >\kappa\) then \(I(T, \lambda) = 1 \iff I(T, \lambda') = 1\). 
A complete proof of this theorem is again beyond the scope of this paper, but we reference {\color{red}someplace or other.} %insert some reference here 

\appendix
\section{Appendix}
this is the appendix


\begin{thebibliography} \small
%
\bibitem[Mar]{mar} David Marker, Model Theory: An Introduction, 2002

\bibitem[Sh1]{shelahUncountable} Shelah, Saharon. Categoricity of uncountable theories -- Proc Tarski Symposium (Univ. of California, Berkeley, Calif., 1971) (1974) 187--203

\bibitem[Sh2]{shelahUnsuperstable} Shelah, Saharon. "Why there are many nonisomorphic models for unsuperstable theories." Proceedings of the International Congress of Mathematicians, Vancouver. 1974.

\bibitem[Har]{hart}Hart, Bradd. A Proof of Morley's Conjecture. Journal of Symbolic Logic 54 (1989), no. 4, 1346--1358. http://projecteuclid.org/euclid.jsl/1183743103.
%
\end{thebibliography}
\end{document}

